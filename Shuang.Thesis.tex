%%%%%%%%%%%%%%%%%%%%%%%%%%%%%%%%%%%%%%%%%%%%%%%%%%%%%%%%%%%%%%%%%%%%%%%%%%%%%%%%
% University of Western Ontario Thesis Template
% By: Justin Quinn Veenstra, 2010
% With thanks to Mr. (soon to be Dr.) Will Robertson.


\documentclass[12pt,twoside]{report}
%% Decomment next line to use PostScript fonts
%%\UsePackage{times}
%%%%%%%%%%%%%%%%%%%%%%%%%%%%%%%%%%%%%%%%%%%%%%%%%%%%%%%%%%%%%%%%%%%%%%%%
%%                                                                    %%
%%                    ***   I M P O R T A N T   ***                   %%
%%                                                                    %%
%% Fill in the following fields with the required information:        %%
%%  - \department{...}  name of the graduate department               %%
%%  - \degree{...}      name of the degree obtained                   %%
%%  - \author{...}      name of the author                            %%
%%  - \title{...}       title of the thesis                           %%
%%  - \gyear{...}       year of graduation                            %%
%%  - \super{...}    supervisor
%%  - \firstname, \middlename, \lastname... there is additional documentation by the actual fields, so I'll leave it at that
%%%%%%%%%%%%%%%%%%%%%%%%%%%%%%%%%%%%%%%%%%%%%%%%%%%%%%%%%%%%%%%%%%%%%%%%
\setcounter{secnumdepth}{4}
\usepackage{appendix}
\usepackage{graphicx}
\usepackage{amsmath}
\usepackage[byname]{smartref}
\usepackage{multirow}
%\usepackage{hyperref} %comment out for hardcopy
\usepackage{txfonts}
\usepackage{lscape}
\usepackage{tocloft}
\usepackage{graphicx}
\usepackage{subfig}
%\usepackage{subfigure}
\usepackage{lipsum}
\usepackage{array}
\usepackage{algorithm,amsmath}
\usepackage{algorithmic}
\usepackage{color,soul}
\usepackage{epstopdf}
\usepackage{subfig}
\usepackage{stmaryrd}
\newcolumntype{L}[1]{>{\raggedright\let\newline\\\arraybackslash\hspace{0pt}}m{#1}}
\newcolumntype{C}[1]{>{\centering\let\newline\\\arraybackslash\hspace{0pt}}m{#1}}
\newcolumntype{R}[1]{>{\raggedleft\let\newline\\\arraybackslash\hspace{0pt}}m{#1}}
\renewcommand{\algorithmicrequire}{\textbf{Input:}}
\renewcommand{\algorithmicensure}{\textbf{Output:}}
\newcommand{\algorithmicbreak}{\textbf{break}}
%\newtheorem{theorem}{Theorem}
%\newtheorem{corollary}{Corollary}
%\newtheorem{lemma}{Lemma}
\newtheorem{defi}{Definition}


\makeatletter
\numberwithin{figure}{chapter}
\newenvironment{acknowledgements}%
{\clearemptydoublepage
	\begin{center}
		\section*{Acknowledgements}
	\end{center}
	\begingroup
}{\newpage\endgroup}

\newenvironment{dedication}%
{\clearemptydoublepage 
	\begin{center}
		\section*{Dedication}
	\end{center}
	\begingroup
}{\newpage\endgroup}

\newenvironment{preliminary}%
{\pagestyle{plain}\pagenumbering{roman}}%
{\pagenumbering{arabic}}

\addtoreflist{chapter}
\newtheorem{theorem}{Theorem}[section]
\newtheorem{lemma}[theorem]{Lemma}
\newtheorem{proposition}[theorem]{Proposition}
\newtheorem{corollary}[theorem]{Corollary}

\newenvironment{proof}[1][Proof]{\begin{trivlist}
		\item[\hskip \labelsep {\bfseries #1}]}{\end{trivlist}}
\newenvironment{definition}[1][Definition]{\begin{trivlist}
		\item[\hskip \labelsep {\bfseries #1}]}{\end{trivlist}}
\newenvironment{example}[1][Example]{\begin{trivlist}
		\item[\hskip \labelsep {\bfseries #1}]}{\end{trivlist}}
\newenvironment{remark}[1][Remark]{\begin{trivlist}
		\item[\hskip \labelsep {\bfseries #1}]}{\end{trivlist}}

\newcommand{\qed}{\nobreak \ifvmode \relax \else
	\ifdim\lastskip<1.5em \hskip-\lastskip
	\hskip1.5em plus0em minus0.5em \fi \nobreak
	\vrule height0.75em width0.5em depth0.25em\fi}

% Default values for title page.

%% To produce output with the desired line spacing, the argument of
%% \spacing should be multiplied by 5/6 = 0.8333, so that 1 1/4 spaced
%% corresponds to \spacing{1.5} and double spaced is \spacing{1.66}.
\def\normalspacing{1.25} % default line spacing
\linespread{\normalspacing}

%% Define the "thesis" page style.
\if@twoside % If two-sided printing.
\def\ps@thesis{\let\@mkboth\markboth
	\def\@oddfoot{}
	\let\@evenfoot\@oddfoot
	\def\@oddhead{
		{\sc\rightmark} \hfil \rm\thepage
	}
	\def\@evenhead{
		\rm\thepage \hfil {\sc\leftmark}
	}
	\def\chaptermark##1{\markboth{\ifnum \c@secnumdepth >\m@ne
			Chapter\ \thechapter. \ \fi ##1}{}}
	\def\sectionmark##1{\markright{\ifnum \c@secnumdepth >\z@
			\thesection. \ \fi ##1}}}
\else % If one-sided printing.
\def\ps@thesis{\let\@mkboth\markboth
	\def\@oddfoot{}
	\def\@oddhead{
		{\sc\rightmark} \hfil \rm\thepage
	}
	\def\chaptermark##1{\markright{\ifnum \c@secnumdepth >\m@ne
			Chapter\ \thechapter. \ \fi ##1}}}
\fi

\pagestyle{thesis}
% Set up page layout.
\setlength{\textheight}{9in} % Height of the main body of the text
\setlength{\topmargin}{-.5in} % .5" margin on top of page
\setlength{\headsep}{.5in}  % space between header and top of body
\addtolength{\headsep}{-\headheight} % See The LaTeX Companion, p 85
\setlength{\footskip}{.5in}  % space between footer and bottom of body
\setlength{\textwidth}{6.25in} % width of the body of the text
\setlength{\oddsidemargin}{.25in} % 1.25" margin on the left for odd pages
\setlength{\evensidemargin}{0in} % 1.25"  margin on the right for even pages

% Marginal notes
\setlength{\marginparwidth}{.75in} % width of marginal notes
\setlength{\marginparsep}{.125in} % space between marginal notes and text

% Make each page fill up the entire page. comment this out if you
% prefer. 
\flushbottom

\setcounter{tocdepth}{3} % Number the subsubsections 
\def\normalspacing{1.25} % default line spacing

\newcommand\isco[1]{%
	\edef\@tempa{#1}%
	\def\@tempb{}%
	\ifx\@tempa\@tempb
	\else \\\underline{Co-Supervisor:}\vspace{0.35in}\\\dots\dots\dots\dots\dots\dots\dots\\{#1}\\
	\fi
}

\newcommand\isjoint[1]{%
	\edef\@tempa{#1}%
	\def\@tempb{}%
	\ifx\@tempa\@tempb
	\else \\\underline{Joint Supervisor:}\vspace{0.35in}\\\dots\dots\dots\dots\dots\dots\dots\\{#1}\\
	\fi
}

\newcommand\isalt[1]{%
	\edef\@tempa{#1}%
	\def\@tempb{}%
	\ifx\@tempa\@tempb
	\else \\\underline{Alternate Supervisor:}\vspace{0.35in}\\\dots\dots\dots\dots\dots\dots\dots\\{#1}\\
	\fi
}

\newcommand\isdefinedsig[1]{%
	\edef\@tempa{#1}%
	\def\@tempb{}%
	\ifx\@tempa\@tempb
	\else \\ \dots\dots\dots\dots\dots\dots\dots\\{#1}\\
	\fi
}
\newcommand\isdefinedspinetitle[1]{%
	\edef\@tempa{#1}%
	\def\@tempb{}%
	\ifx\@tempa\@tempb
	\else (Spine title: #1)\\
	\fi
}
\newcommand\coauthor[1]{%
	\edef\@tempa{#1}%
	\def\@tempb{}%
	\ifx\@tempa\@tempb
	\else \newpage \Large Co-Authorship Statement\normalsize\\\indent\\#1\\
	\fi
}

\newcommand\acknowlege[1]{%
	\edef\@tempa{#1}%
	\def\@tempb{}%
	\ifx\@tempa\@tempb
	\else \newpage \Large Acknowlegements\normalsize\\\indent\\#1\newpage
	\fi
}

%\renewcommand{\appendixtocname}{\Huge \textbf{List of Appendices} \normalsize}
\newcommand{\blank}{\hspace{-2mm}}
\newcommand{\super}{Dr. Charles X. Ling} %supervisor
\newcommand{\superj}{} %joint supervisor, if there is one, leave blank if not (lbin)... only one of the three.
\newcommand{\superc}{} %co-supervisor, if there is one, leave blank if not (lbin)
\newcommand{\supera}{} %alternate supervisor, if there is one, leave blank if not (lbin)
\newcommand{\sco}{}  %member of supervisory committee
\newcommand{\sct}{}  %other member of supervisory committee (lbin)
\newcommand{\examo}{Dr. Huajie Zhang}  %examining committee (up to four, if less leave blank)
\newcommand{\examt}{Dr. Dan Lizotte}
\newcommand{\examth}{Dr. Olga Veksler}
\newcommand{\examf}{Dr. Luiz Fernando}
\newcommand{\department}{Computer Science}
\newcommand{\degree}{Doctor of Philosophy}
\newcommand{\firstname}{Shuang}
\newcommand{\middlename}{}
\newcommand{\lastname}{Ao}
%\renewcommand{\author}[1]{\ifx\empty#1\else\gdef\@author{#1}\fi} 
\newcommand{\authorname}{{\firstname} {\middlename} {\lastname}}
\newcommand{\titl}{Visual Transfer Learning in the Absence of the Source Data}
\newcommand{\spinetitle}{}%only if the above is more than 60 characters
\newcommand{\thesisformat}{Monograph} %or Integrated Article
\newcommand{\gyear}{\number\year}
\newcommand{\makecoauthor}{
	%Type information about coauthorship here/
	I would like to acknowlege my imaginary friend, Jummi for doing all the work. 
}
\newcommand{\makeacknowlege} {
	%Type in acknowlegements here % % % % % % % % % % % % % % % % % % % % % % % % % % % % % % % % % % % % % % % %
	Firstly, I would like to thank my advisor Prof Charles X. Ling for his patience and insightful guidance. He was always eager
	to discuss new ideas at any time help me understand the research and inspired me to thinking critically for the topic I was working on.
	
	Thanks are also given to my colleagues: Yan Luo, Xiao Li, Bin Gu, Chang Liu, Robin Liu and Jun Wang for their collaboration and valuable discussions. Many thanks to Xiang Li. We worked together with many ideas and he is very helpful in many details of the papers we published together.
	
	The Last gratitude is given to my families: my parents Dingan Ao and Liwen Yang, and my wife Jinglang Hu. Without their encouragements and supports, I am not able to pursuing my PhD degree in Western and finish this dissertation.
	
	My research is supported by NSERC Grants and scholarship from the University’s the
	School of Graduate Studies. This thesis would not have been possible without the generous
	resources provided by the Department of Computer Science.
}
\newcommand{\listappendixname}{List of Appendices}
\newlistof{myappendices}{app}{\listappendixname}
\newcommand{\myappendices}[1]{%
	\addcontentsline{app}{myappendices}{#1}\par}

\renewcommand{\maketitle}
{\begin{titlepage}
		\setcounter{page}{1}
		%% Set the line spacing to 1 for the title page.
		%\begin{spacing}{1} 
		\begin{large}
			\begin{center}
				\mbox{}
				\vfill
				{\MakeUppercase{\titl}}\\
				\isdefinedspinetitle{\spinetitle}
				(Thesis format: \thesisformat)\\
				\vfill
				by \\
				\vfill
				{\firstname} \underline{\lastname}\\
				\vfill
				Graduate Program in {\department}\\
				\vfill
				A thesis submitted in partial fulfillment\\
				of the requirements for the degree of\\
				\degree\\
				\vfill
				The School of Graduate and Postdoctoral Studies\\
				The University of Western Ontario\\
				London, Ontario, Canada\\
				\vfill
				{\copyright} {\authorname} {\gyear}  \\
				\vspace*{.2in}
			\end{center}
		\end{large}
		%   \end{spacing}
	\end{titlepage}
	
}%\maketitle

\newcommand{\makecert}{
	\setcounter{page}{2}
	\vfill
	\begin{center}
		\large
		THE UNIVERSITY OF WESTERN ONTARIO\\
		School of Graduate and Postdoctoral Studies\\
		\vfill
		\textbf{CERTIFICATE OF EXAMINATION}
	\end{center}
	
	\vfill
	\begin{table}[ht]
		\begin{minipage}[t]{0.5\linewidth} %tabular instead?
			\begin{tabular}{l}
				\underline{Supervisor:}\vspace{0.35in}
				\isdefinedsig{\super}
				\isco{\superc}
				\isjoint{\superj}
				\isalt{\supera}
				\\
				\underline{Supervisory Committee:}\vspace{0.35in}
				\isdefinedsig{\sco}\vspace{0.15in}
				\isdefinedsig{\sct}
			\end{tabular}
			\vfill
		\end{minipage}
		\hspace{0.5in}
		\begin{minipage}[t]{0.5\linewidth}
			\begin{tabular}{l}
				\underline{Examiners:} \\\vspace{.5cm}
				\isdefinedsig{\examo}\\
				\isdefinedsig{\examt}\\
				\isdefinedsig{\examth}\\
				\isdefinedsig{\examf}
			\end{tabular}
			\vfill
		\end{minipage}
		\vfill
	\end{table}
	\vfill
	\begin{center}
		The thesis by \\ \vfill
		\textbf{\firstname{} \middlename{} \underline{\lastname}}\\
		\vfill
		entitled:\\\vfill
		\textbf{\titl}\\\vfill
		is accepted in partial fulfillment of the \\
		requirements for the degree of\\
		\degree\\
	\end{center}
	\begin{table}[ht]
		\begin{minipage}[t]{0.5\linewidth}
			\begin{tabular}{l}
				\dots\dots\dots\dots\dots\\
				Date
			\end{tabular}
		\end{minipage}
		\hspace{0.5in}
		\begin{minipage}[t]{0.5\linewidth}
			\begin{tabular}{l}
				\dots\dots\dots\dots\dots\dots\dots\dots\dots\dots\\
				Chair of the Thesis Examination Board
			\end{tabular}
		\end{minipage}
	\end{table}
	
}

\makeatother
\begin{document}
	
	%% ***   NOTE   ***
	%% You should put all of your '\newcommand', '\newenvironment', and
	%% '\newtheorem's (in other words, all the global definitions that you
	%% will need throughout your thesis) in a separate file and use
	%% "\input{filename}" to input it here.
	
	
	%% This sets the page style and numbering for preliminary sections.
	\begin{preliminary}
		
		%% This generates the title page from the information given above.
		\maketitle
		\addcontentsline{toc}{chapter}{Certificate of Examination}
		\makecert
		\newpage
		%\addcontentsline{toc}{chapter}{Co-Authorship Statement}
		%\coauthor{\makecoauthor}  %comment this out if none
		%\newpage
		\addcontentsline{toc}{chapter}{Acknowlegements}
		\acknowlege{\makeacknowlege}    %as above
		
		\addcontentsline{toc}{chapter}{Abstract}
		\Large\begin{center}\textbf{Abstract}\end{center}\normalsize
		%%  ***  Put your Abstract here.   ***
		%% (150 words for M.Sc. and 350 words for Ph.D.)
		
		Image recognition has become one of the most popular topics in machine learning. With the development of Deep Convolutional Neural Networks (CNN) and the help of the large scale labeled image database such as ImageNet, modern image recognition models can achieve competitive performance compared to human annotation in some general image recognition tasks. Many IT companies have adopted it to improve their visual related tasks. However, training these large scale deep neural networks requires thousands or even millions of labeled images, which is an obstacle when applying it to a specific visual task with limited training data. Visual transfer learning is proposed to solve this problem. Visual transfer learning aims at transferring the knowledge from a source visual task to a target visual task. Typically, the target task is related to the source task, and the training data in the target task is relatively small.
		In visual transfer learning, the majority of existing methods assume that the source data is freely available and use the source data to measure the discrepancy between the source and target task to help the transfer process. However, in many real applications, source data are often a subject of legal, technical and contractual constraints between data owners and data customers. Beyond privacy and disclosure obligations, customers are often reluctant to share their data. When operating customer care, collected data may include information on recent technical problems which is a highly sensitive topic that companies are not willing to share. This scenario is often called Hypothesis Transfer Learning (\textbf{HTL}) where the source data is absent. Therefore, these previous methods cannot be applied to many real visual transfer learning problems.
		
		In this thesis, we investigate the visual transfer learning problem under HTL setting. Instead of using the source data to measure the discrepancy, we use the source model as the proxy to transfer the knowledge from the source task to the target task. Compared to the source data, the well-trained source model is usually freely accessible in many tasks and contains equivalent source knowledge as well. Specifically, in this thesis, we investigate the visual transfer learning in two scenarios: domain adaptation and learning new categories. In contrast to the previous methods in HTL, our methods can both leverage knowledge from more types of source models and achieve better transfer performance.
		
		In chapter 3, we investigate the visual domain adaptation problem under the setting of Hypothesis Transfer Learning. We propose Effective Multiclass Transfer Learning (\textbf{EMTLe}) that can effectively transfer the knowledge when the size of the target set is small.  Specifically, EMTLe can effectively transfer the knowledge using the outputs of the source models as the auxiliary bias to adjust the prediction in the target task. Experiment results show that EMTLe can outperform other baselines under the setting of HTL.
		
		In chapter 4, we investigate the semi-supervised domain adaptation scenario under the setting of HTL and propose our framework {Generalized Distillation Semi-supervised Domain Adaptation} (\textbf{GDSDA}). Specifically, we show that GDSDA can effectively transfer the knowledge using the unlabeled data. We also demonstrate that the imitation parameter, the hyperparameter in GDSDA that balances the knowledge from source and target task, is important to the transfer performance. Then we propose GDSDA-SVM which uses SVMs as the base classifier in GDSDA. We show that GDSDA-SVM can determine the imitation parameter in GDSDA autonomously. Compared to previous methods, whose imitation parameter can only be determined by either brutal force search or background knowledge, GDSDA-SVM is more effective in real applications.
		
		In chapter 5, we investigate the problem of fine-tuning the deep CNN to learn new food categories using the large ImageNet database as our source. Without accessing to the source data, i.e. the ImageNet dataset, we show that by fine-tuning the parameters of the source model with our target food dataset, we can achieve better performance compared to those previous methods.
		
		To conclude, the main contribution of is that we investigate the visual transfer learning problem under the HTL setting. We propose several methods to transfer the knowledge from the source task in supervised and semi-supervised learning scenarios. Extensive experiments results show that without accessing to any source data, our methods can outperform previous work.
		
		
		\vfill
		\textbf{Keywords:} Visual Transfer Learning, Hypothesis Transfer Learning, Supervised Learning, Semi-supervised Learning
		
		\newpage
		\tableofcontents\newpage
		\newpage
		\addcontentsline{toc}{chapter}{List of Figures}
		\listoffigures
		\newpage
		\addcontentsline{toc}{chapter}{List of Tables}
		\listoftables\newpage
		%\addcontentsline{toc}{chapter}{List of Appendices}
		%\listofmyappendices\newpage
		%\addcontentsline{toc}{chapter}{List of Abbreviations, Symbols, and Nomenclature}
		%\large List of Abbreviations, Symbols, and Nomenclature \normalsize
		%\newpage
	\end{preliminary}
	%% End of the preliminary sections: reset page style and numbering.
	
	%%%%%%%%%%%%%%%%%%%%%%%%%%%%%%%%%%%%%%%%%%%%%%%%%%%%%%%%%%%%%%%%%%%%%%%%
	%%                                                                    %%
	%%                    ***   I M P O R T A N T   ***                   %%
	%%                                                                    %%
	%% Put your Chapters here; the easiest way to do this is to keep each %%
	%% chapter in a separate file and \include all the files right here.  %%
	%% Note that each chapter file should start with the line             %%
	%% "\chapter{ChapterName}".  Note that using "\include" instead of    %%
	%% "\input" makes each chapter start on a new page.                   %%
	%%%%%%%%%%%%%%%%%%%%%%%%%%%%%%%%%%%%%%%%%%%%%%%%%%%%%%%%%%%%%%%%%%%%%%%%
	\chapter{Introduction}\label{sec:intro}
With the explosive image resources people uploaded every day, image recognition becomes a very hot topic and has draw many attentions in recent years. Every year, there are many inspiring results in the ImageNet Large Scale Visual Recognition Challenge (ILSVRC). 
With development of recognition technology, many IT companies want to use image recognition techniques to serve their customers and some interesting applications have been developed, such as HowOld from Microsoft and Im2Calories from Google.

In order to successfully capture the diversity of different objects around us, many recognition models contain thousands or sometimes even millions of parameters and require large amount of training images to tune these parameters as well.
Unfortunately, for some real applications, it is often difficult and cost to collect large set of training images. Moreover, most algorithms require that the training examples should be aligned with a prototype, which is commonly done by hand. In the real applications, collecting and fully annotating these images can be extremely expensive and could have a significant impact on the over cost of the whole system. As more companies pay attention to the individual customer experience, personalized service system become more important in recent years. For image recognition, one application of personalization can be learning the categories defined by individuals. The challenge of the self-defined categories learning comes from two aspects:
\begin{enumerate}
	\item Flexibility: people could define any categories they want. The recognition algorithm should be more dynamic to adapt the new categories.
	\item Rare learning examples: it is almost impossible to get abundant images from the self-defined categories and the algorithm should be able to learn the new categories from just a few examples.
\end{enumerate} 

On the other hand, recognition is one of the most important part of our human visual system. We can recognize various kinds of materials (apple, orange, grape), objects (vehicles, buildings) and natural scenes (forests, mountain). At the age of six, human can recognize about $10^4$ object categories\cite{biederman1987recognition}. 
Our human can learn and recognize a new object with just a glance, which means we can capture the diversity of forms and appearances of a objects with just a handful examples. It could be ideal if we can find a way to train a new category with few examples.

\section{Overview for Image Recogntion}\label{sec:intro:over}
In this section, we review the major procedures for image recognition. 
A general image recognition method consists of three parts: image preprocess, feature extraction and classification.

\begin{figure}
	\centering
	\includegraphics[scale=.8]{introduction/fig/IRflow.png}
	\caption{Major procedure for image recognition.}\label{fig:intro:irflow}
\end{figure}
\subsection{Preprocess}
The digital camera can capture the optical property of an object through its optical sensor and generate raw digital data.
After receiving the raw data of a image from the sensor, preprocess is to generate a new image from the source image. This new image is similar to the source image, but differs from it considering certain apsects, e.g. the new image has smoother edge, better contrast and less noise. 
Here, some \textit{pixel operations} and \textit{local operations} are used to improve the contrast and remove the noise.  

Another important operation of preprecess is segmentation according to the object, i.e. finding the region of interest. Images used for recognition should be aligned, making the target object appear in the central of the image and remove those irrelevant area.

The result of preprocess has great impact on the final result of the recognition. Clear and noise free images can make the feature extraction more effective and significantly improve the final classification accuracy.

\subsection{Feature Extraction}
 Feature extraction is a type of dimensionality reduction that efficiently represents interesting parts of an image as a compact feature vector. The feature vector is then used for either training the classifier or recognition. Therefore, feature extraction is the most important part for image recognition. The quality of the features extracted from a image have great impact on the recognition result. There are two major streams for feature extraction: the hand engineered method and representation learning method.
\subsubsection{Hand Engineered Feature}
Hand engineered features are typically low level and local features.
Low level features are extracted according to some optical properties of an image. These features are low level / local features. There is a widely agreement that local features are an efficient tool for object representation due to their robustness with respect to occlusion and geometrical transformations \cite{van2006coloring}. Common low level hand engineered features include Histogram of Oriented Gradients (HOG) \cite{dalal2005histograms}, Scale Invariant Feature Transform (SIFT) \cite{lowe1999object}, Speeded Up Robust Features (SURF) \cite{bay2006surf}, Local Binary Patterns (LBP) \cite{ojala2002multiresolution}, and color histograms \cite{birchfield1998elliptical}. Feature descriptors obtain from these low level features refer to a pattern or distinct structure found in an image, such as a point, edge, or small image patch. They are usually associated with an image patch that differs from its immediate surroundings by texture, color, or intensity. What the feature actually represents does not matter, just that it is distinct from its surroundings.
\begin{figure}[h]
	\centering
	\includegraphics[scale=.6]{introduction/fig/sift.jpg}
	\caption{Feature extraction using SIFT.}\label{fig:intro:sift}
\end{figure}
These low level features can be used directly for recognition. However, since they just represent certain local properties of an image and are not discriminative enough for recognition, discriminative high level features can be further learned by combining the low level features. 

\subsubsection{Representation Learning}
Representation learning is mainly described by Deep Learning 
algorithms\cite{krizhevsky2012imagenet} or Auto Encoders\cite{bengio2007scaling}. The ideas is to learn a group of filters that are able to capture various kinds of features to discern one category of images from the another category with some supervised or unsupervised algorithm. Typically in representation learning, features are learned hierarchically from low-level features to high level ones. 
Learning representation from an image can start from either low level features (Auto Encoders) or raw pixels of an image (Deep Learning). It is generally considered that learning the good feature representations is the most important part in image recognition.
\begin{figure}
	\centering
	\includegraphics[scale=.3]{introduction/fig/sparsecoding.png}
	\caption{General Scheme of Auto Encoders. L1 is the input layer, possibly raw-pixel intensities. L2 is the compressed learned latent representation and L3 is the reconstruction of the given L1 layer from L2 layer. AutoEncoders tries to minimize the difference between L1 and L3 layers with some sparsity constraint.}\label{fig:intro:sparse}
\end{figure}

\textbf{Auto Encoders} are widely used to combine different types of low level feature. The outputs of the Auto Encoders are some latent representations. These latent representations are learned from the given images that have lowest possible reconstruction error. Even though the high level representations from Auto Encoders are learned by minimizing the reconstruction errors, they are still not robust enough to handle all kinds of variance of the objects.

\textbf{Deep Learning} is the most popular approach for learning representations. It has been widely used for all kinds of image recognition tasks and achieved the state-of-the-art performance on some large scale image recognition tasks, such as ILSVRC and The PASCAL Visual Object Classes Challenge (PASCAL VOC). 
Convolutional Neural Networks (CNN) is the most popular deep learning model for the image recognition tasks\footnote{Deep CNNs are sometimes considered as the end-to-end classifier while learning the feature representation and discriminative classifier simultaneously. However, the feature representation learned from deep CNNs can still achieve good results with other classifiers and here we consider it as a feature extractor rather than a classifier.}. The first deep CNN that had great success on image recognition is the LeNet proposed by Y.LeCun in 1989 \cite{lecun1989backpropagation}. Backpropagation was applied to Convolutional Neural networks with adaptive connections. This combination, incorporating with Max-Pooling and speeding up on graphics cards has become an important part for  many modern, competition-winning, feedforward, visual Deep Learners. Deep CNNs have been widely used as the feature extractor for all kinds of images recognition tasks and proven to be the most powerful method for feature extraction.
\begin{figure}
	\centering
	\includegraphics[scale=.3]{introduction/fig/alexnet.png}
	\caption{The architecture of ALEXNET (adapted from \cite{krizhevsky2012imagenet}).}\label{fig:intro:alex}
\end{figure}

\subsection{Classification}
After extracting feature representation from the images, a classifier should be used to train a recognition model as well as for predicting the new coming images. A supervised model is always used for training the recognition model. Discriminative classifier such as Support Vector Machine (SVM) is widely used as the classifier for recognition \cite{cristianini2000introduction}. As we mentioned before, in order to capture different variances of the images for one category, the size of the feature representation for a image is usually very large. In order to avoid overfitting, the size of the training set should be at least the same size of the size of the feature representation as well. Some classifiers such as Bayesian method or decision tree require to consider the correlations between each feature and the class labels and suffer from the large feature dimension. However, Discriminative Models\cite{bottou2010large} are more convenient for training and can be effectively optimized with stochastic gradient descent which is suitable for very large training set. 

However, obtaining a good recognition model requires abundant data when we learning a model from scratch. With the limited training data, it is difficult to achieve a good classification performance. Transfer learning is an effective way to solve this problem by utilizing the knowledge from previous tasks. In this thesis, we focus on how to transfer the knowledge from the source domain for recognition tasks. The methods proposed in this thesis are mainly focus on the stage of classification. %More specifically, we investigate the problem of how to build classifier to leverage the source knowledge when we can only visit the source model. 




\section{Our Scenario (to be refined)}
\subsection{Importance of learning self-defined categories}
Nowadays, more and more companies provide individual service for their clients. Personalized recommendation system has been widely used in many electronic commerce, such as Amazon and eBay. The requirement of personalized service is growing every year. Personalized system means it is possible to handle the variance of the request from individuals. For image recognition, many interesting applications have been proposed. However, unlike other personalized system, it is difficult to train personalized recognition system for individuals. There could be enormous variances of one object and it is always difficult for a model to capture these variances.

Even though, the state-of-the-art recognition system can do as well as a human, its great success is achieved based on the fact that millions of labeled training images are used for training. Selecting large amount of images for the new categories is always a tedious job.
Moreover, an important property of self-defined category images is that the source of the data is inherently scarce. Therefore, it is not possible to obtain abundant training data. For example, users of Google's im2calories can track their nutrition of every meal by taking a pictures of their foods via image recognition techniques. The system firstly check the category of each food item and find their nutritions in the database. However, the existing model can only recognize the general food categories which means it is not possible for a user to track the nutrition of his/her daily home-made meals. These home-made foods can be similar to some existing food categories (home-made veggie burger can be similar to burger in McDonald), but they may have different nutritions. This scenario can be applied to learning any exclusive category in our life. Therefore, a model that can learn these self-defined categories can be important.

\begin{figure}
	\centering
	\includegraphics[scale=.6]{introduction/fig/scenario.jpg}
	\caption{Different nutrition facts between the burgers in McDonald and home-made.}\label{fig:intro:scenario}
\end{figure}

Our human is good at learning the new objects. For our human, all the information acquired is stored in our memory. These information are organized according to the properties. When we see a new concept, we don't treat it isolated, but connect it to certain previous knowledge we stored in our memory. By comparing a new concept with the organized information in our memory, we can capture the property of a new concept effectively. When referring to visual tasks, several examples can be given to show this cognitive ability. For instance, when we describe the animal "zebra", we would probably say that: zebra is a horse with while and black striped coats. People who never see a zebra could instantly have a rough idea of a zebra. 

This means to learn a new category effectively, we should be able to make use of the gained knowledge instead of learn it from scratch. This process is commonly refer to as transfer learning. Traditional machine learning methods work under the common assumption: training data and testing data are drawn from the same feature space and same distribution. When considering adding classifying the data from the new categories, the distribution of the data changes and a new model should be rebuilt from newly collected data.Those data from the original distribution is called source data and data from the new distribution is called target data. Transfer learning is used to utilize the source knowledge from the source data to help training the new model to classify the target data. 

%\hl{We can keep learning new categories throughout our whole life and become more proficient as we learn more. Human doesn't need large amount of training data to adapt a new object. In most scessnario, we can still recognize the new object well by a few examples, e.g. taking a glancing at the object. This indicates that it is possible for a recognition model to adapt by just a few training examples from the new categories.}
In this thesis, we try to use transfer learning approaches to solve the problem of learning self-defined categories. We can find many examples in knowledge engineering where transfer learning does benefit the learning process \cite{pan2010survey}. This implies that, by leveraging the learned knowledge properly, we can learn the self-defined categories with a few examples. 
In this thesis, we also assume that the self-defined categories are not isolated and similar to certain existed categories. Therefore, we can use leverage the knowledge from those existed categories to help us learn the self-defined categories. Then, we split the self-defined categories into two groups according to their relationship to the exited categories: 
\begin{itemize}
	\item One source category, a category that is very similar to one existed category. For example, my veggie burger can be very similar to general burger and less similar to other categories.
	\item Multi-source category, a category that doesn't have an explicit similarity to one category but share some properties with several existed categories (see figure \ref{fig:intro:multi} for instance).
\end{itemize} 
Due to different properties of these two groups, we should design different strategies to adapt them.

\begin{figure}
	\centering
	\includegraphics[scale=.6]{introduction/fig/multiple.jpg}
	\caption{Multi-source category case: an okapi can be roughly described as the combination of a body of a horse, legs of the zebra and a head of giraffe.}\label{fig:intro:multi}
\end{figure}

\subsection{Assumption of the source knowledge}
In transfer learning, the source knowledge can be presented in two different ways \cite{pan2010survey}: 
\begin{itemize}
	\item Instance transfer learning. Even though the source data can not be re-used directly, certain part of the data can be used incorporating with a few labeled target data to train the model.
	\item Parameter transfer learning. Instead of utilizing the source data, parameter transfer learning approaches re-use the parameters of the model trained from source data (called the source model). 
\end{itemize}

In this thesis, we try to explore a method to learn self-defined categories via the parameter transfer learning approach and try to utilizing the source data in a hard way, by assuming that the source data is access prohibited and we can only access the trained model from the source data.
This assumption is made based on the following two facts: (1) In some situations, we may not be able to access the source data and only the model trained from the source data is available. There are many credential datasets. Therefore, it is not always possible to fully access the source data. (2) The source data could be very large and it could be tedious to determine which part of the data could benefit the transfer learning. On the other hand, the trained model from the source data can be as informative as the source data itself. For example, the information extracted from the support vectors (SVs) of a SVM model trained from a dataset could be as much as the information of the whole dataset. Some results from recent work also show that it is possible to obtain a good model by even just utilize the source models and learn new categories from a few examples \cite{fei2006one} \cite{tommasi2010safety} \cite{tommasi2014learning}.

In this section, we discuss the scenario of the learning self-defined category problem. We conclude that learning self-defined categories can be achieved by transfer learning. We also make an assumption that we can only access the model trained from the source data and the source data itself is prohibited. \hl{In the next section, we discuss the challenges of our problem.}

%In this thesis, we propose a novel transfer learning method, called \textbf{SMITLe} (\textbf{S}afety \textbf{M}ulticlass \textbf{I}ncremental \textbf{T}ransfer \textbf{Le}arning) that is able to learn new categories from a few examples.
\section{Challeges}
In this section, we discuss several major challenges that we may encounter when solving the learning self-defined category problem.
To solve the learning self-defined category problem, there are the following major problems:
\begin{enumerate}
	\item What's the classifier and parameter we should use to train the model that can transfer the knowledge from the source data effectively?
	\item How are the knowledges transferred from the source task to target one and control the amount of the knowledge that should be transferred? 
	\item How to guarantee the result of transfer model? 
\end{enumerate}

The first challenge is to determine the classifier we use for the source data as well as the target data. There are two criterion for us to choose the classifier: (1) This classifier should be popular and has been used in many tasks. We would expect our method to be general enough and can be applied to many different scenarios. (2) Because we can only access the trained source model, this source model should be able to restore as much information as its training data to make sure that there is enough knowledge to be leveraged for the transferred model.  
%Therefore, in this thesis we choose \textbf{S}upport \textbf{V}ector \textbf{M}achines (SVM) as the classifier to train the source model as well as the target one. \cite{cristianini2000introduction}. SVM has been widely used in various of machine learning and image recognition tasks due to its performance and generalization ability. Another advantage of using SVM as the classifier is that a SVM model can provide as much information as its train data \hl{(we will discuss it later!!!!)}.
Then we have to choose the parameters of the model to be used for transfer learning. As we discussed above, the parameter should be informative enough to represent the source data. 
%In this thesis, we choose the hyperplane parameter $W$ of the SVM model as the transfer parameter.

The second challenge of out problem is to determine the way of the knowledge transferred from source model to target one. To learn a new category by transfer learning, according to our human experience, we would firstly select several known categories that are similar to the new one and then describe the new category as a combination of these known ones (see figure \ref{fig:intro:multi}). It is worthy to note that despite of the knowledge from the source models, we will also extract some new knowledge from the examples of the new category. This indicates that how to combine those knowledge from learned categories (e.g. the parameters of the source models) and the knowledge from the new category is the key for our task. 

The last challenge is to guarantee the performance of our transfer method. A baseline criterion is that after leveraging the knowledge from the source model, the performance of the transfer model to distinguish the self-defined categories will be improved or at least won't get worse when there is little useful knowledge in the source model. This is an important criterion for transfer learning. We should always expect to leverage knowledge of the source model to help training the target model. However, inappropriately leveraging the knowledge of the source models would decrease the performance of the target model. For example, we could never expect to obtain a good model to distinguish apple and other fruits by relying too much on the knowledge of distinguishing human and animal.
	\chapter{Related Work}\label{sec:works}
In this chapter, we review some previous work related to ours. We first review the classifiers used in this thesis, providing the general concept and principle of how they work. Then we review the types of transfer learning for visual recognition from two different views and then discuss the previous work of how to alleviate negative transfer. Finally, we review some methods related to the 3 methods used in this thesis.
\section{Classifiers for Image Recognition}\label{sec:relat:linear}

In our scenario, we have to face the problem of image recognition. Due to the large dimension of the feature representation for each image as well as the size of training image, manual classification is hopeless. As we mentioned in Section \ref{sec:intro:over}, a recognition model is used to distinguish the objects from different categories automatically trained by supervised learning. In this section, we introduce the classifiers we used in this thesis.

\subsection{Binary Classification and Multi-class Classification}

In image recognition, we train a recognition model from a set of training images along with their labels provided. The labels are predefined in a category space. Thus, the task of image recognition is to classify each image as one predefined category. If there are only two categories, this recognition task is called binary classification. For the task recognizing the objects from more than two categories, the recognition task is called multi-class classification \cite{aytar2011tabula} \cite{krizhevsky2012imagenet}.  

Here, we give a formal definition of the scenario for binary and multi-class classification. Generally, we can decompose the multi-class learning task into a set of binary scenarios by training a binary classifier for each class, e.g. One-VS-Rest strategy (see figure \ref{fig:related:ovsa})\cite{rifkin2004defense} \cite{tsoumakas2006multi}.

The binary scenario for classification can be defined as follow: given a dataset from domain $\mathcal{X} \times \mathcal{Y}$ where $\mathcal{X}$ is the input feature representation and $\mathcal{Y}$ is the binary label set $\{1,-1\}$ (for some classifier, $\{1,0\}$ is also used). We usually use the label $1$ to denote the examples belong to one certain category and -1 to denote examples not belong to that category. 
We assume that the training image set $D_{train}=\{(x_i,y_i)\} \subset \mathcal{X} \times \mathcal{Y}$ and the test image set $D_{test}=\{(x^t_i,y^t_i)\}\subset \mathcal{X} \times \mathcal{Y}$ are given and separated from each other. Each pair $(x_i,y_i)$ denotes the input feature representation $x_i$ and its corresponding label $y_i$ for the $i$\textit{th} image in the both set. Our goal of the classification problem is to learn a decision function $f:\mathcal{X} \rightarrow \mathcal{Y}$ from the training set $D_{train}$ such that $f$ can achieve good performance on both $D_{train}$ and $D_{test}$. 

\begin{figure}
	\centering
	\includegraphics[scale=.8]{relatedwork/fig/ovsa.png}
	\caption{One-vs-Rest strategy for multi-class scenario. A three classes problem can be decomposed into 3 binary classification sub-problems.}\label{fig:related:ovsa}
\end{figure}
\subsection{Softmax Classifier}
In this subsection, we will introduce a widely used linear classifier \textbf{Softmax classifier} in image recognition. Linear classifier is commonly used as the classification model for image recognition. Linear classifier achieves this by making a classification decision based on the value of a linear combination of the input feature representations of a image. A linear classifier consists of two parts: a score function and a loss function\cite{vapnik1999overview}. The score function maps the input data into the class scores and the loss function that quantifies the agreement between the predicted scores and the ground truth labels. Linear classifier often works very well when the number of dimensions of the input is large. Therefore, it is widely used as the classifier for image recognition, especially as the classifier for Convolutional Nueral Networks \cite{lecun1989backpropagation}. 

Typically, Softmax classifier is widely used for multi-class image classification.
Softmax classifier (also called multinomial logistic regression) is a generational form of logistic regression for the multi-class scenario. As logistic regression can only handle the binary classification scenario, Softmax classifier adapts the one-vs-rest strategy where several logistic regression models are trained for each class. 

Given a training set $\{(x_1,y_1),(x_2,y_2),...,(x_m,y_m)\}$, we assume the label $y_i \in \{1,0\}$ and the input feature $x_i \in \mathcal{R}^n$. For each binary logistic regression model, The score function takes the form:
\begin{equation}
{f_w}(x) = \frac{1}{{1 + \exp ( - {w^T}x)}}
\end{equation}
and the parameters $w$ are optimized to minimize the following loss function:
\begin{equation}
l(w) =  - [\sum\limits_{i = 1}^m {{y_i}\log {f_w}({x_i}) + (1 - {y_i})} \log (1 - {f_w}({x_i}))]\label{eq:logistic:loss}
\end{equation}

For Softmax classifier, it is used to handle the multi-class classification problem and suppose there are $N$ classes. Therefore, $y$ can take from $N$ different values $\{1,2,3,...,N\}$ instead of just two. For a given test example $x$, the score function estimate the probability $P(y=n|x)$ for each value of $n = 1,2,3,...,N$, i.e. estimate the probability that each score assigns the input $x$ to the $N$ classes associated to the $N$ different possible values. Thus, the score function will output a $N$-dimensional vector providing $N$ estimated probabilities whose sum of elements is 1. The $N$ dimensional output of the score function can be generated according to the following form:
\begin{equation}
{f_w}\left( x \right) = \left[ \begin{array}{l}
P\left( {y = 1|x;w} \right)\\
P\left( {y = 2|x;w} \right)\\
...\\
P\left( {y = n|x;w} \right)
\end{array} \right] = \frac{1}{{\sum\nolimits_{j = 1}^N {\exp \left( {{w^{(j)T}}x} \right)} }}\left[ \begin{array}{l}
\exp \left( {{w^{(1)T}}x} \right)\\
\exp \left( {{w^{(2)T}}x} \right)\\
...\\
\exp \left( {{w^{(N)T}}x} \right)
\end{array} \right]
\end{equation}
Here $w^{(1)T},w^{({2})T},...,w^{({N})T}$ are the parameters of the Softmax classifier model. The term $\frac{1}{{\sum\nolimits_{j = 1}^N {\exp \left( {{w^{(j)T}}x} \right)} }}$ is called the normalization term so that the $N$ distributions sum to 1.

To optimize the parameters $w^{(1)T},w^{({2})T},...,w^{({N})T}$, the cross-entropy loss used for the Softmax classifier is defined as:
\begin{equation}
J(w) =  - \left[ {\sum\limits_{i = 1}^l {\sum\limits_{k = 1}^N {\ell\left\{ {{y_i} = k} \right\}\log \frac{{\exp \left( {{w^{(k)T}}x_i} \right)}}{{\sum\nolimits_{j = 1}^N {\exp \left( {{w^{(j)T}}x_i} \right)} }}} } } \right] \label{eq:softmax:loss}
\end{equation}
Here $\ell{x}$ is the 0-1 loss function:
\begin{equation}
\ell \{ x\}  = \left\{ {\begin{array}{*{20}{c}}
	1&\text{$x$ is true}\\
	0&\text{$x$ is false}
	\end{array}} \right. 
\end{equation}
It is noted that Eq \eqref{eq:softmax:loss} is a generalized form of Eq \eqref{eq:logistic:loss}. Minimizing Eq \eqref{eq:softmax:loss} can be interpreted as minimizing the negative log likelihood of the correct class, which is equivalent to performing Maximum Likelihood Estimation (MLE) \cite{johansen1990maximum}.

The minimum of $J(w)$ can be obtained by gradient descent method while taking the gradient:
\begin{equation}
\nabla J({w^{(n)}}) =  - \sum\limits_{i = 1}^l {\left[ {{x_i}\left( {\ell \left\{ {{y_i} = n} \right\} - \frac{{\exp \left( {{w^{(k)T}}x_i} \right)}}{{\sum\nolimits_{j = 1}^N {\exp \left( {{w^{(j)T}}x_i} \right)} }}} \right)} \right]} 
\end{equation}

\subsection{Support Vector Machines}
In this subsection, we will review another widely used discriminant classifier, \textbf{Support Vector Machine} (SVM) \cite{cristianini2000introduction}. SVM is another classifier that has been adoptive in many image recognition tasks \cite{coates2011analysis} \cite{schuldt2004recognizing} \cite{yang2009linear}. In this thesis, we also use a classifier based on SVM. We will give a detailed description of SVM. 

As we mentioned before, the linear classifier consists of two parts: the score function and loss function. SVM classifier can be divided into two categories based on their score function: linear SVM that uses a linear discriminant function and kernel SVM that uses kernel function. Kernel SVM can be considered as a extension version of linear SVM where kernels are used for calculate the scores of the inputs. Another difference for between linear and kernel SVM is linear SVM can be solved on the primal problem while kernel SVM is mostly optimized on its dual \cite{cristianini2000introduction} \cite{shalev2011pegasos}.

First, we will introduce the linear SVM. Linear SVM uses the simplest representation of a score function by taking the linear combination of the input vector:
\begin{equation}\label{eq:relation:score}
f(x) = w^Tx+b
\end{equation}
where $w$ is called the weight vector and $b$ is called the bias. In a binary scenario, for the input example $x$ and the class labels $y \in \{c1,c2\}$, $x$ is assigned to class $c1$ if $f(x) \geq 0$ and $c2$  otherwise. Therefore, the corresponding decision surface is defined by $f(x)=0$. For two points $x_1$ and $x_2$ lie on the decision surface, we have $f(x_1)=f(x_2)=0$. Then we can have $w^T(x_1-x_2)=0$ and hence the weight vector $w$ is orthogonal to every point lying within the decision surface, i.e $w$ determines the orientation of the decision surface. Similarly, if $x$ lies on the decision surface, the normal distance from the origin to the decision surface is given by:
\begin{equation}
\frac{w^Tx}{||w||}=-\frac{b}{||w||}
\end{equation}
Therefore, we can see that, the location of the decision surface is determined by the bias $b$.
\subsubsection{Hard Margin SVM}
Hard margin SVM is used to find the optimal solution for the data sets that are linear separable. 
Given a set of n training points $(x_1,y_1),(x_2,y_2),...,(x_n,y_n)$ where $y_i \in \{1,-1\}$, we expect to find a decision surface that can separate the data from two classes. However, when the data from these two classes can be linearly separated, there could be several decision surfaces that can separate the data. The idea of SVM is to choose the maximal margin decision surface so that the distance between these two classes is as large as possible (called maximal margin hyperplane). For example, in figure \ref{fig:relate:svm}\subref{fig:relate:svma}, $H2$ and $H3$ are two candidate decision surface that can separate the data. However, $H3$ is has the largest distance to all the data from two classes and SVM will choose $H3$ as the optimal hyperplane. 

\begin{figure}
	\centering
	\subfloat[Different sparating hyperplanes]{\includegraphics[width = 0.45\textwidth]{relatedwork/fig/svm2.png}\label{fig:relate:svma}}
	\subfloat[Max-Margin hyperplanes]{\includegraphics[width = 0.4\textwidth]{relatedwork/fig/svm1.png}}
	\caption{Support Vector Machine}\label{fig:relate:svm}
\end{figure}
The optimal hyperplane can be found by minimizing the following objective function:
\begin{equation}
\begin{aligned}
\min \qquad & ||w||^2\\
\text{s.t.}\qquad & y_i(w^Tx_i+b) \geq 1 \quad \text{for all } 1\leq i \leq n
\end{aligned}
\end{equation}
\subsubsection{Soft Margin SVM}
In real world application, most data are not linear separable. Therefore, SVM introduce the concept of slack variable to handle this situation. Slack variable is defined as:
\begin{equation}
\xi_i  = hinge(x_i) = \max (0,1-y_i(w^Tx_i+b))
\end{equation}
The value of slack variable is 0 if the example $x_i$ lies on the correct side of the margin. For those data on the wrong side, its value is proportional to the distance from the correct margin.
\begin{figure}
	\centering
	\includegraphics[scale =1.2]{relatedwork/fig/slack.png}
	\caption{Slack variables for soft-margin SVM}
\end{figure}

Therefore, to find the parameters of hyperplane, i.e. weight vector $w$ and bias $b$, soft-margin SVM minimize the following loss function:
\begin{equation}\label{eq:related:softsvm}
\begin{aligned}
\min \qquad &  \frac{\lambda}{2}||w||^2+\frac{1}{n}\sum_{i}^{n}\xi_i\\
\text{s.t.}\qquad & y_i(w^Tx_i+b) \geq 1-\xi_i \\
& \xi_i \geq 0     \quad \text{for all } 1\leq i \leq n
\end{aligned}
\end{equation}
The objective function is called primal of SVM. A stochastic sub-gradient descent can be used to find the optimal solution for eq. \eqref{eq:related:softsvm} effectively \cite{shalev2011pegasos}. 
\subsubsection{Kernel SVM}
Linear soft-margin SVM works well when number of features is larger than number of training examples. However, when the size of the training example is larger than the features, kernel SVM (such as Gaussian Kernel) with proper parameters outperforms linear SVM \cite{keerthi2003asymptotic}.
\begin{figure}
	\centering
	{\includegraphics[width = 0.8\textwidth]{relatedwork/fig/non-linear2.png}}
	\caption{The hyperplane of SVM with RBF kernel for non-linear separable data.}\label{fig:relate:nonlinear}
\end{figure}

The idea of kernel was first introduced into pattern recognition by Aizerman \textit{et. al.} \cite{aizerman1964probability}. When the size of the training examples are significantly larger than the dimension of the input features and the distribution become more complex, these data can not be easily separated by a straight line in the feature space.
Instead of obtaining the optimal hyperspace in the input feature space, kernel SVM tries to map the inputs into a high-dimensional feature spaces and find the optimal hyperplane in the high-dimensional feature spaces. Given a feature space mapping $\phi(x)$, the score function for kernel SVM can be re-written as:
\begin{equation}\label{eq:relation:kernel}
f(x) = w^T\phi(x)+b
\end{equation}
From eq \eqref{eq:relation:kernel} we can see that, when we take the identity mapping $\phi(x) = x$, the kernel SVM becomes a linear SVM. Therefore, the linear SVM can be considered as a special case of kernel SVM where identity mapping is used as the kernel. The loss function of kernel SVM is almost identical to eq. \eqref{eq:related:softsvm} except for replacing the term $x$ with $\phi(x)$: 
\begin{equation}\label{eq:related:primal}
\begin{aligned}
\min \qquad &  \frac{\lambda}{2}||w||^2+\frac{1}{n}\sum_{i}^{n}\xi_i\\
\text{s.t.}\qquad & y_i(w^T\phi(x)_i+b) \geq 1-\xi_i \quad \\
& \xi_i \geq 0     \quad \text{for all } 1\leq i \leq n
\end{aligned}
\end{equation}
By introducing the Lagrangian term to the primal \eqref{eq:related:primal} and some transformation, we obtain the dual of kernel SVM function:
\begin{equation} \label{eq:related:dual}
\begin{aligned}
\max \qquad& \sum_{i}^{n}\alpha_i-\sum\limits_i^n {\sum\limits_j^n {{y_i}{y_j}{\alpha _i}} } {\alpha _j}\phi ({x_i})\phi ({x_j})\\
\text{s.t.} \qquad & 0 \leq \alpha_i \leq \frac{1}{\lambda}\\
& \sum_{i}^{n}y_i\alpha_i=0 \quad \text{for all } 1\leq i \leq n
\end{aligned}
\end{equation}
We can obtain the solution of \eqref{eq:related:dual} by Sequential Minimal Optimization (SMO) \cite{platt1998sequential} or Dual Coordinate Descent \cite{hsieh2008dual}.

There are several advantages of using SVM as the classifier: 
\begin{itemize}
	\item \textbf{Generalization ability.} SVM provides good generalization ability by maximizing the margin between the examples of the two classes. By setting the proper parameters and generalization grade, SVM can overcome some bias from the training set. Therefore, SVM is able to make correct prediction for unseen data. This ability can be very useful for image recognition as there is no image dataset that can cover all the transformation of the objects. Moreover,the idea of soft-margin makes it robust against noisy data.
	\item \textbf{Kernel transformation.} By introducing the non-linear transformation of the input, SVM can model complex non-linear distributed data. The kernel trick can greatly improve the computational efficiency.   
	\item \textbf{Unique solution.} The objective function of SVM is  convex. Compared to other methods, such as Neural Networks, which are non-convex and have many local minima, SVM can deliver a unique solution for any given training set and can be solved with efficient methods, like sub-SGD \cite{shalev2011pegasos} or SMO \cite{platt1998sequential}. 
\end{itemize}


\subsection{Convolutional Neural Networks}
Convolutional Neural Networks \cite{lecun1998gradient} is the most popular and powerful method for image recognition task. In this part, we introduce the history of Convolutional Neural Networks. The detail description of the Convolutional Neural Networks layers will be included in chapter \ref{sec:cnn}.

\subsubsection{Early Work with Convolutional Neural Networks}
The first simple version of Neural Networks (NNs) trained with supervised learning was proposed in 1960s  \cite{rosenblatt1958perceptron}\cite{rosenblatt1962principles}. Networks trained by the Group Method of Data Handling (GMDH) could be the first DL systems of the Feedforward Multilayer Perceptron type \cite{ivakhnenko1965cybernetic}\cite{Schmidhuber14}. Later, there have been many applications of GMDH-style
nets \cite{farlow1984self} \cite{ikeda1976sequential} \cite{kondo2008multi} \cite{witczak2006gmdh}.

Apart from deep GMDH networks, the Neocognitron, a hierarchical, multilayered artificial neural network, was perhaps the first artificial NN to incorporate the
neurophysiological insights \cite{fukushima1980neocognitron}. Inspired by Neocognitron, Convolutional NNs (CNNs) was proposed where the rectangular receptive field of a convolutional unit with given weight vector is shifted step by step across a 2-dimensional array of input values, such as the pixels of an image (usually there are several such filters). The results of the previous unit can provide inputs to higher-level units, and so on. Because of its massive weight replication,  relatively few parameters may be necessary to describe its behavior.

In 1989, backpropagation \cite{lecun1989backpropagation}\cite{lecun1998gradient} was applied to Convolutional Neural networks with adaptive connections \cite{lecun1989backpropagation}. This combination, incorporating with Max-Pooling and speeding up on graphics cards has become an important part for many modern, competition-winning, feedforward, visual Deep Learners. Later, CNNs achieved good performance on many practical tasks such as MNIST and fingerprint recognition and was commercially used in these fields in 1990s \cite{baldi1993neural} \cite{le1990handwritten}. 

In the early 2000s, even though GPU-MPCNNs wons several official contests, many practical and commercial pattern recognition applications were dominated by non-neural machine learning methods such as Support Vector Machines (SVMs).

\subsubsection{Recent Achievements with Convlutional Neural Networks}
In 2006, CNN trained with backpropagation set a new MNIST record of 0.39\% without using unsupervised pre-training \cite{marc2006efficient}. Also in 2006, an early GPU-based CNN implementation was introduced which was up to 4 times faster than CPU-CNNs \cite{chellapilla2006high}. Since then, GPUs or graphics cards have become more and more essential for CNNs in recent years. In 2012, a GPU implemented Max-Pooling CNNs (GPU-MPCNNs) was also the first method to achieve human-competitive performance (around 0.2\%) on MNIST \cite{ciresan2012multi}.

In 2012, an ensemble of GPU-MPCNNs (called AlexNet) achieved best results (top-5 accuracy at 83\%) on the ImageNet classification benchmark (ILSVRC2012), which contains 1000 classes and 1.2 million images \cite{krizhevsky2012imagenet}. After that, excellent results have been achieved by GPU-MPCNNs in image recognition and classification. Many attempts have been made to improve the architecture of AlexNet. With the help of high performance computing systems, such as GPUs and large scale distributed cluster, some improvements have been made by either making the network deeper or increasing the size of the training data  (with extra training example and data argumentation). By reducing the size of the receptive field and stride, Zeiler and Fergus improve AlexNet by 1.7\% on top 5 accuracy \cite{zeiler2014visualizing}. By both adding extra convolutional layers between two pooling layers and reducing the receptive field size, Simonyan and Zisserman built a 19 layer very deep CNN and achieved 92.5\% top-5 accuracy \cite{simonyan2014very}. After the AlexNet-like deep CNNs won ILSVRC2012 and ILSVRC2013, Szegedy et al. built a 22-layer deep network, called GoogLeNet and won the 1st prize on ILSVRC2014 for 93.33\% top-5 accuracy, almost as good as human annotation\cite{szegedy2014going}. Different from AlexNet-like architecture, GoogLeNet shows another trend of design, utilizing many $1\times 1$ receptive field. Recently, Wu et. al present an image recognition system by aggressive data augmentation on the training data, achieving a top-5 error rate of 5.33\% on ImageNet dataset\cite{wu2015deep}. Searchers from Google successfully trained an ingredient detector system based on GoogLeNet with 220 million images harvested from Google Images and Flickr \cite{malmaud2015s}.  

Besides its impressive performance on those huge datasets, MPCNNs shows some impressive results by fine-tuning the existing models on small datasets.Zeiler et al. applied their pre-trained model on Caltech-256 with just 15 instances per class and improved the previous state-of-the-art in which about 60 instances were used, by almost 10\% \cite{zeiler2014visualizing}. Chatfield et al. used their pre-trained model on VOC2007 dataset and outperformed the previous state-of-the-art by 0.9\% \cite{Chatfield14}. Zhou et al. trained AlexNet for Scene Recognition across two datasets with identical categories and provided the state-of-the-art performance using our deep features on all the current scene benchmarks \cite{NIPS2014_Zhou}. Hoffman et al. fine-tuned the MPCNNs trained from ImageNet with one example per class, showing that it is possible to use a hybrid approach where one uses different feature representations for the various domains
and produces a combined adapted model \cite{hoffman2013one}.

To summarize, in this section, we have briefly reviewed three types of classifier for image recognition, namely Softmax classifier, SVM and CNNs. Softmax classifier is typically used as the last layer of CNNs. SVM is a more general method for the classification task. Moreover, The generalization ability of SVM classifier can reduce the bias from the training data.

\section{An Overview of Visual Transfer Learning}
\begin{figure}
	\centering
	\includegraphics[scale =.7]{relatedwork/fig/transfer.png}
	\caption{Apart from the standard machine learning, transfer learning can leverage the information from an additional source: knoweldge from one or more related tasks.}
\end{figure}

Traditional machine learning algorithms try to build the classifiers from a set of training data and apply to the test data with the same distribution to the training data. In contrast, transfer learning attempts to change this by transfer the learned knowledge from one or several tasks (called \textbf{source tasks}) to improve a related new task (called \textbf{target task}). According to the situations of the source and target tasks, transfer learning can be categorized as 3 types: inductive transfer learning, transductive transfer learning and {unsupervised transfer learning}. On the other hand, from the types of the source knowledge, transfer learning can be classified as: instance transfer, feature representation transfer and parameter transfer. 

% Table generated by Excel2LaTeX from sheet 'Sheet1'
\begin{table}[htbp]
	\centering
	\caption{Categories of our learning scenarios}
	\begin{tabular}{|c|c|c|}
		\hline
		& Situation of Task     & Source Knowledge \\
		\hline
		Learning New Categories & Inductive Transfer & \multirow{2}[0]{*}{Parameter Transfer} \\
		\cline{1-2}
		Domain Adaptation & Transductive Transfer &  \\
		\hline
	\end{tabular}%
\end{table}%


\subsection{Types of Transfer Learning from the Situations of Tasks}
Transfer learning can be categorized into 3 sub-settings: \textbf{inductive transfer learning, transductive transfer learning} and \textbf{unsupervised transfer learning} based on the different situations of the source and target domains and tasks. \cite{pan2010survey}. We compared the differences of these three sub-categories and show them in Table \ref{tab:related:transfersetting}. 

\begin{table}[htbp]
	\centering
	\caption{Relationship between traditional machine learning and different transfer learning settings}
	\begin{tabular}{|c|c|c|c|}
		\hline
		\multicolumn{2}{|c|}{Learning settings} & Source target domain & Source target task \\
		\hline
		\multicolumn{2}{|c|}{Traditional machine learning} & the same & the same \\\hline
		\multirow{3}{*}{Transfer learning} & \textbf{Inductive transfer learning} & the same & different but related \\\cline{2-4}
		& Unsupervised transfer learning & different but related & different but related \\\cline{2-4}
		& Transductive transfer learning & different but related & the same \\\hline    
	\end{tabular}%
	\label{tab:related:transfercmp}%
\end{table}%

\begin{table}[htbp]
	\centering
	\caption{Various settings of transfer learning}
	\begin{tabular}{|c|C{4cm}|c|c|}
		\hline
		& Related areas & Source data & Target data \\
		\hline
		\multirow{2}[4]{*}{Inductive transfer learning} & self-taught learning & unlabeled & labeled \\\cline{2-4}
		& multi-task learning & labeled & labeled \\\hline
		Transductive transfer learning& \textbf{domain adaptation}, Sample selection bias & labeled & labeled/unlabeled \\\hline
		Unsupervised transfer learning &       & unlabeled & unlabeled \\
		\hline
	\end{tabular}%
	\label{tab:related:transfersetting}%
\end{table}%

\subsection{Types of Transfer Learning from the Aspect of Source Knowledge}
According to the type of the source knowledge comes from, transfer learning can be split into 3 major streams: instance transfer, feature representation transfer and parameter transfer.

The core idea of instance transfer learning is to select some useful data from the source task to help learning the target task. Dai et al. \cite{dai2007boosting} propose a method (called TrAdaBoost) that can select the most useful examples from the source task as the additional training examples for the target task. These useful examples are iteratively re-weighted according to the classification results of some base classifiers. Jiang et al. \cite{jiang2007instance} proposed a method that can ignore the "misleading" examples from the source data based on the conditional probabilities on the source task $P(y_t|x_t)$ and target task $P(y_s|x_s)$. Liao et al. \cite{liao2005logistic} proposed a active learning method that selects and labels the unlabeled data from the target data with the help of the source data. Ben-David et al. \cite{ben2010theory} provided a theoretical analysis the lowest target test error for different source data combination strategies when the source data is large and target training set is small.   

Feature representation transfer aims to find a good feature representations to reduce the gap between the source and target domains. According to the size of labeled examples in the source data, feature representation transfer consists of two approaches: supervised feature construction and unsupervised feature construction. When the source data are labeled, supervised feature transfer learning is used to find the feature representations shared in related tasks to reduce the difference between the source and target tasks. Evgeniou et al. \cite{evgeniou2007multi} proposed a method that can learn sparse low-dimension feature representations that can share between different tasks. Jie et al. \cite{jie2011multiclass} reconstructed the feature representations for the target data by using the outputs of the source models as the auxiliary feature representations. In unsupervised feature representation transfer learning, Daume III \cite{daume2009frustratingly} proposed a simple feature reconstruction method for both source and target data so that source and target data are triple augmented and a SVM model is trained on both source and target data.

Parameter transfer assumes that there should be some parameters or prior distribution of the hyperparameters in the individual models of related tasks. Most of the approaches are designed under the multi-task learning scenario. Therefore, in this thesis, we also focus on the parameter transfer approach to leverage the knowledge from the source data. In parameter transfer learning, there are three major frameworks: a regularization framework, a Bayesian framework and a neural network framework.
\begin{figure}
	\centering
	\includegraphics[scale =1]{relatedwork/fig/parameters.png}
	\caption{Two steps for parameter transfer learning. In the first step multi-source and single source combination are usually used to generate the regularization term. The hyperplane for the transfer model can be obtained by either minimizing training error or cross-validation error on the target training data.}
\end{figure}
\begin{itemize}
	\item Regularization framework: In the regularization framework, some researchers propose to transfer the parameters of the SVM following the assumption that the hyperplane for the target task should be related to the hyperplane of the source models. Evgeniou et al. \cite{evgeniou2004regularized} proposed an idea that the hyperplane of the SVM for the target task should be separate into two terms: a common term shared over tasks and a specific term related to the individual task. Inspired by this idea, some researchers propose different strategies to combine these two terms for transfer learning \cite{aytar2011tabula} \cite{tommasi2010safety} \cite{yang2007cross}. Most of these work contains two steps. In the first step, a SVM objective function with a biased regularization term for the target model is build. Then another objective function is build to reduce the empirical error of the target model on the target data.
	\item Neural network framework. In neural network framework, the idea is to use the parameters of a CNN pre-trained from a very large dataset as an initialization to reduce the bias when the target data is small. Yosinski et al. \cite{yosinski2014transferable} show that the high level layer parameters are more related to a specific task while the low level layer ones are more general and transferable. This framework is widely used for image recognition task. By re-using and fine-tuning the parameters of some layers in the pre-trained model, the bias of the target task can be greatly alleviated \cite{Chatfield14} \cite{hoffman2013one}  \cite{zeiler2014visualizing} \cite{NIPS2014_Zhou}. 
	\item Bayesian framework. In Bayesian framework, one or several posterior probabilities of the source data or parameters of the source model can be used to generate a prior probability for the target task. With this prior probability, a posterior probability for the target task can be obtained with the target data.
	Li et al. \cite{fei2006one} used a prior probability density function to model the knowledge from the source and modify it with the data from target to generate posterior density for detection and recognition. 
	Rosenstein et al. \cite{rosenstein2005transfer} used hierarchical Bayesian method to estimate the posterior distribution for all the parameters and the overall model can decide the similarities of the source and target tasks. 
\end{itemize}

\subsection{Special Issues in Avoiding Negative Transfer}
In transfer learning, for a given target task, the performance of a transfer method depends on two aspects: the quality of the source task and the transfer ability of the transfer algorithm. The quality of the source task refers to how the source and target tasks are related. If there exists a strong relationship between the source and target, with a proper transfer method, the performance in the target task can be significantly improved. However, if the source and target tasks are not sufficiently related, despite of the transfer ability of the transfer algorithm, the performance in the target task may fail to be improved or even decrease. In transfer learning, negative transfer refers to the degraded performance compare to a method without using any knowledge from the source \cite{pan2010survey}. How to avoid negative transfer is still an open question for researchers. 
For example, we can use a teacher-student diagram to illustrate the procedure of transfer learning. The student (target model) would like to learn the new knowledge (target task) with the assistance of a teacher (source knowledge). If the teacher can provide helpful knowledge (related knowledge), the student can learn the new knowledge very quickly (positive transfer). If the teacher can only provide useless knowledge, the student could not learn the new knowledge effectively or even get confused (negative transfer).

\begin{figure}
	\centering
	\includegraphics[scale=.7]{relatedwork/fig/negative.png}
	\caption{Positive transfer VS Negative transfer.}
\end{figure}

Another important aspect that affect the learning performance is the transfer ability of the algorithm used for the target task. An ideal transfer algorithm would be able to produce positive transfer on related tasks while avoiding negative transfer on unrelated tasks. However, in practice, it is not easy to achieve these two goals simultaneously. Approaches that can avoid negative transfer often bring some affects on positive transfer due to their caution. On the other hand, approaches using aggressive transfer strategies often have little or no protection against negative transfer \cite{torrey2009transfer}. Even though voiding negative transfer is an important issue in transfer learning, how to avoid negative transfer has not been widely addressed \cite{lu2015transfer} \cite{pan2010survey}. Previous work show suggest that negative transfer can be alleviated through 3 approaches \cite{torrey2009transfer}: 
\begin{itemize}
	\item Rejecting unrelated source information. A important approach to avoid negative transfer is to recognize and reject unrelated and harmful source knowledge. The goal of this approach is to minimize the impact of the unrelated source, so that the transfer model performs no worse than the learned model without transfer. Therefore, in some extreme situation, the transfer model is allowed to completely ignore the source knowledge. 
	Torrey et al. \cite{torrey2005using} proposed a method using advice-taking algorithm to reject the unrelated source knowledge. 
	Rosenstein et al. \cite{rosenstein2005transfer} presented an approach that use naive Bayes classifier to detect and reject the unrelated source.
	\item Choosing correct source task. When the source knowledge come from more than one candidate source, it is possible for the transfer model to select the best source knowledge of the candidates. In this scenario, leverage the knowledge from the best candidate may be effective against negative transfer as long as the best source knowledge is sufficiently related. 
	Talvitie et al. \cite{talvitie2007experts} proposed a method that can iteratively evaluate the candidate sources through a trail-and-error approach and select the best one to transfer. Kuhlmann et al. \cite{kuhlmann2007graph} constructed a kernel function from certain sources for the target task by estimating the bias from a set of candidate sources whose relationship to the target task is unknown.
	\item Measuring task similarity. To achieve a better transfer performance, it is reasonable for a transfer method to transfer the knowledge from multiple sources instead of just choosing a single source. In this approach, some methods try to involve all the source knowledge without considering the explicit relationship between the source and target. The other methods try to model the relationships between the source and target tasks and use the information as a part of their transfer methods which can significantly reduce the affect of negative transfer.
	Bakker et al. \cite{bakker2003task} proposed a method to provide guidance on how to avoid negative transfer by using clustering and Bayesian approach to estimate the similarities between the target task and multiple source tasks.
	Tommasi et al. \cite{tommasi2014learning} constructed the transfer model by using some transfer parameters to measure the relationships between each source and the target tasks and the transfer parameters are optimized by minimizing the cross-validation error of the transfer model. Similar approaches can be found in \cite{jie2011multiclass} \cite{kuzborskij2013n}.
	Kuzborskij et al. \cite{kuzborskij2013stability} provided some theoretical analysis of transfer learning and show that regularized least square SVM with truncation function and leave-one-out cross-validation for source task measurement can reduce negative transfer even though the training data of the target task is relatively small.
\end{itemize}

Here we can see that most of the previous approaches focus on measuring the similarity of the source and target tasks, i.e. try to assign the most related source tasks to the target one through various of metrics and use aggressive transfer algorithm to exploit the source knowledge. Just a few work \cite{kuzborskij2013stability} \cite{tommasi2010safety} addressed the problem that a sophisticated transfer algorithm should be designed to better exploit the source knowledge as well as avoid negative transfer. Therefore, in this thesis, we mainly focus on how to design a better transfer algorithm for transfer learning while certain source knowledge is assigned. 

To summarize, in this section, we provided an overview of the categories of transfer learning from two different views. From the relationships of the tasks, our two transfer learning scenarios belong to inductive transfer and transductive transfer learning respectively. In this thesis, we assume that we are not able to access to the source data, therefore, from the aspect of source knowledge, we use parameter transfer for both scenarios. In this thesis, without visiting any source data, it is difficult to measure the relationship between the source and target tasks. Therefore, avoiding negative transfer is also an important part we have to consider. Finally, we reviewed some methods that cab alleviate negative transfer.

\section{Related Work in Hypothesis Transfer Learning}
The main topic of this thesis is to investigate the problem of visual transfer learning under the HTL setting. Therefore, in this section, we show some work related to this topic and this thesis. We first review some work in fine-tuning the deep neural networks for inductive transfer learning related to our work in chapter \ref{sec:cnn}. Then we discuss some methods in hypothesis transfer learning, which is related to our work in chapter \ref{sec:pakdd}. The related work in chapter \ref{sec:aaai} is reviewed at last.

\subsection{Fine-tuning the Deep Net}
Since Deep Convolutional Neural Networks (CNNs) became the most powerful algorithm in object recognition task, fine-tuning the deep CNNs has become an popular and effective way to transfer the knowledge between different visual recognition tasks. The intuition of fine-tuning the deep CNNs for transfer learning is that, low level features, such as edges and lines, are universal for object recognition while high level features, which are the combinations of the low level features, are more specific for the designed task. Because deep CNNs can learn hierarchical features, from abstract low level features to detailed high level ones, by changing the combinations of the low level features in the pre-trained deep CNNs, the high level features can be learned effectively for the new recognition task \cite{farabet2013learning}.

\begin{figure}
	\centering
	\includegraphics[scale=.7]{relatedwork/fig/hierachy}
	\caption{Hierarchical Features of Deep Convolutional Neural Networks for face recognition.}
\end{figure}

Applying the pre-trained model from ImageNet dataset on other object recognition benchmark datasets shows some impressive results. Zeiler et al.  \cite{zeiler2014visualizing} applied their pre-trained model on Caltech-256 with just 15 instances per class and improved the previous state-of-the-art in which about 60 instances were used, by almost 10\%.
Chatfield et al.  \cite{Chatfield14} used their pre-trained model on VOC2007 dataset and outperformed the previous state-of-the-art by 0.9\%.
Agrawal et al. \cite{agrawal2014analyzing} show that even in the mid-level features, there are some grandmother cells, which can capture the high level features of specific objects. Hoffman et al. \cite{hoffman2013one} show that even with one labeled example per class, it is possible to fine-tune the pre-trained deep CNNs and obtain a good classifier for some new recognition tasks. Zhou et al. \cite{NIPS2014_Zhou} provided the state-of-the-art performance using the deep features on some scene benchmarks by fine-tuning the deep CNNs. Yosinski et al. \cite{yosinski2014transferable} investigated the transferability of the layers in deep CNNs and show that the target task can be benefited from pre-training even though the source and target tasks are distant. 

In this thesis, we also use the pre-trained deep CNNs to learn new categories for food recognition and investigate the affects of each layer in deep CNNs for knowledge transfer in chapter \ref{sec:cnn}.

\subsection{Hypothesis Transfer Learning with SVMs}
\input{relatedwork/hypothesis}

\subsection{Distillation for Knowledge Transfer}
\textbf{Distillation} \cite{hinton2015distilling} was developed frameworks for knowledge transfer and addressed the problem how to effectively transfer the knowledge from the source model directly. In chapter \ref{sec:aaai}, we propose a framework called GDSDA based on it to solve semi-supervised domain adaptation problem. In this part, we review the the principle of Distillation and some related work using this framework. The technical details will be introduced in chapter \ref{sec:aaai}.

Hinton et al. proposed Distillation to transfer the knowledge from a source neural network (or a whole ensemble of neural networks) to a single target one. In this setting, the capacity of the source neural network is large while the capacity of the target one is small. The capacity reflects the expressive power of a model and a model with larger capacity can fit complex data better. In statistical learning theory \cite{vapnik1999overview}, the relationship of the generalization error $e_{te}$ and training error $e_{tr}$ of a model can be expressed as follow:
\begin{equation}\label{eq:rw:general}
e_{te}\leq e_{tr}+2\sqrt{\frac{\log h +\log\frac{2}{\eta}}{2N}}
\end{equation}
Here, $h$ is the capacity (VC dimension) of the model and $N$ is the training set size. Inequation \ref{eq:rw:general} holds with a probability of $1-\eta$.
When we train the target model, we can let the it mimick the output of the source one on the training set. If both source and target models can achieve similar training error on the training set, the small target model can typically do much better on the test data due to its lower capacity. In this process, we actually don't need the true label of the training data. Instead, we only require the outputs of the source model of the training data. However, introducing the true label of the training data can further improve the performance of the target model.

Distillation is typically used for training the deep neural network for knowledge transfer between different models and tasks. Tzeng et al. \cite{Tzeng_2015_ICCV} proposed a CNN architecture for domain adaptation to leverage the knowledge from limited or no labeled data using the soft label. Urban et al. \cite{urban2016deep} use a small shallow net to mimick the output of a large deep net while using layer-wised distillation with $\ell_2$ loss of the outputs of student and teacher net. Similarly, Luo et al. \cite{luo2016face} use $\ell_2$ loss to train a compressed student model from the teacher model for face recognition. Gupta et al. \cite{Gupta_2016_CVPR} use supervision transfer to distill the knowledge from a trained CNN with unlabeled data or just a few labeled data.

The limitation of the previous work in Distillation is that it is difficult to balance the importance of the knowledge from source model and the true label from the training data to train the target model. Previous studies avoid this problem by either using brutal force searching or domain knowledge. In this thesis, we proposed a novel method that can autonomously balance the importance and extend it to the semi-supervised domain adaptation.






\section{Summary}
In this chapter, we reviewed the some concepts and work related to this thesis. We demonstrate the main methods related to our work and their limitations. In the following chapters, we will provide the technical details of this thesis. 
	\chapter{Effective Multiclass Transfer For Hypothesis Transfer Learning}\label{sec:pakdd}
\section{Introduction}
Domain adaptation for image recognition tries to exploit the knowledge from a source domain with plentiful data to help learn a classifier for the target domain with a different distribution and little labeled training data. In domain adaptation, the source and target domains share the same label but their data are drawn from different distributions.

Previous research \cite{ben2010theory,ben2007analysis} shows that without carefully measuring the distribution similarity between the source and target data, the source knowledge could not be exploited effectively or even hurt the learning process (called  \textit{negative transfer})\cite{pan2010survey}. 
However, as we are not able to access the source data in the Hypothesis Transfer Learning (HTL) \cite{kuzborskij2013stability} setting, how to effectively and safely exploit the knowledge from the source model could be an important issue in HTL, especially when target data is relatively small (Effectiveness issue). Moreover, the source models from different domains can be trained with different kinds of classifiers. For example, most models trained from ImageNet are deep convolutional neural networks while some models of the VOC recognition task could be SVMs or ensemble models. Therefore, a practical HTL algorithm should be compatible with different types of source classifiers (Compatibility issue). Previous work is limited to either leveraging the knowledge from a certain type of source classifiers \cite{tommasi2014learning,fei2006one} or low transfer efficiency in a small training set\cite{jie2011multiclass}. To the best of our knowledge, none of the previous work in HTL is able to solve these two issues at the same time.

In this chapter, we propose our method, called \textbf{Effective Multiclass Transfer Learning} (EMTLe), that can solve these two issues simultaneously. We perform comprehensive experiments on 4 real-world datasets from two benchmark datasets (3 from Office and 1 from Caltech256). We show that EMTLe can effectively transfer the knowledge with different types of source models and outperforms the baseline methods under the HTL setting. 

This work has been accepted by \textit{Pacific-Asia Conference on Knowledge Discovery and Data Mining, 2017}.

\section{Using the Source Knowledge as the Auxiliary Bias}\label{sec:prob}
%In this section, we introduce our strategy in EMTLe that can exploit the knowledge from different types of source classifiers. In general, for each example in the target domain, we use its output class probabilities from the source models as the auxiliary bias term to adjust the final prediction of the target model.

\begin{figure}
	\centering
	\includegraphics[scale=0.6]{pakdd/fig/mktl.png}
	\caption{Illustration of feature augmentation in MKTL. $f_i'$ is the output of the $i$-th source model and $\beta_{in}$ is the hyperparameter (need to be estimated) to weigh the augmented feature. $\phi_n(x)$ is augmented feature for the $n$-th binary model.}
	\label{fig:mktl}
\end{figure}
Some previous work such as MKTL\cite{jie2011multiclass} suggests that using the prediction from the source model as the source knowledge can greatly release the constraint of the type of the source model. However, with complex feature augmentation method, there are many hyperparameters to be estimated which makes it inefficient with small training set. In this paper, we adopt the idea of using the source model prediction as the transferable knowledge and propose our transfer strategy.

Suppose we have to recognize a image from one of the $N$ visual classes and there are $N$ experts each of who can only provide the probability of this image for one certain class (binary source model). After we make our decision for one example (prediction from target model), the experts provide their own decisions as well (probabilities from the source models). Their decisions can provide extra information regarding this example as the auxiliary bias and adjust our final prediction.
As each of the experts is a specialist in one class, we should weigh their decisions as well due to the bias of their predictions (see Figure \ref{fig:ab}). 

Unlike previous work\cite{aytar2011tabula,tommasi2014learning,yang2007adapting} which has to use the specific parameter of the source model as the source knowledge, our strategy is more compatible with different types of classifiers. Compared to MKTL\cite{jie2011multiclass}, we only have to estimate $N$ hyperparameters for the $N$-class problem while there are \mbox{$N\times N$} hyperparameters in MKTL (see Figure \ref{fig:mktl}). Therefore, it is easier to estimate the transfer parameters with our strategy and EMTLe can perform better especially when the size of the training set is small.
In addition, there are two advantages of our strategy: (1) It is an effective and easy way to align the knowledge from different types of source classifiers.
(2) The auxiliary bias term is naturally normalized in the same dimension as the class probabilities are always in the interval $[0,1]$.  As EMTLe can select more types of source classifiers, this makes it more practical in a real HTL scenario.

\begin{figure}
	\centering
	\includegraphics[scale=.7]{pakdd/fig/ab.png}
	\caption{Demonstration of using the source class probability as the auxiliary bias to adjust the output of the target model.}% $f'$ is a group of binary classifiers $\{f'_1,...,f'_4\}$ for each class and for each source model $f'_n$, we use the weight $\beta_n$ to control the knowledge transferred from this model.}
	\label{fig:ab}
\end{figure}


Here, the weight of each source model reflects the relatedness between the source model and our target domain. The more related they are, the better decision the source model can make and the larger weight we should apply to it. Specifically, in this paper, we call the weight \textit{transfer parameter}. Therefore, for any target data $D=\{x,y\}$ and the given source models $f'=\{f'_1,...,f'_N\}$, our goal is to find the target model $f$:
\begin{equation}\label{eq:low_opt}
f=\underset{f \in \mathcal{F}}{\arg \min}\ell\left(f+\beta f'|D,\beta\right)
\end{equation} 
where $\beta=[\beta_1,...,\beta_N]$ is the transfer parameter and $\ell(\cdot,\cdot)$ is the loss function to learn the target model.
It is obvious that assigning the proper transfer parameter to the source model can significantly improve the performance of our final prediction.
From Eq. \eqref{eq:low_opt} we can see that, once we have determined the value of the transfer parameter $\beta$, we are able to find the target model $f$ and solve the learning problem.
However, the transfer parameter in Eq.\eqref{eq:low_opt} is a hyperparameter and we cannot solve it directly. Therefore, we introduce our bi-level optimization method for transfer parameter estimation in the next section.









\section{Bi-level Optimization for Transfer Parameter Estimation}\label{sec:smitle}
As we discussed before, the transfer parameter in Eq. \eqref{eq:low_opt} is a hyperparameter that cannot be solved directly. 
Here we use bi-level optimization (\textbf{BO})\cite{Pedregosa16}, a popular method that is used in hyperparameter optimization to estimate the transfer parameter. In BO, the low-level optimization problem is to learn the target model and the high-level problem is another cross-validation (CV) hyperparameter optimization problem corresponding to the model learned at the low-level.

\begin{figure}[h]
	\centering
	\includegraphics[scale=.7]{pakdd/fig/bo.png}
	\caption{Bi-level Optimization problem for EMTLe.}% $f'$ is a group of binary classifiers $\{f'_1,...,f'_4\}$ for each class and for each source model $f'_n$, we use the weight $\beta_n$ to control the knowledge transferred from this model.}
	\label{fig:pakdd:bo}
\end{figure}

Suppose we use K-fold CV on the high-level problem. For the $i$-th fold CV, the target set $D$ is split into training set $D_i^{tr}$ and validation set $D_i^{val}$. The transfer parameter can be optimized with the following BO function:
\begin{equation}\label{eq:BO}
\begin{aligned}
\text{High level}\qquad&\beta=\underset{\beta}{\arg \min}\sum_i^K\mathcal{L}(f^{i}(\beta)|D_i^{val})\\
\text{Low level}\qquad&f^{i}(\beta)=\underset{f \in \mathcal{F}}{\arg \min}\ell\left(f+\beta f'|D_i^{tr},\beta\right) 
\end{aligned}
\end{equation} 
Here, $\ell(\cdot,\cdot)$ and $\mathcal{L}(\cdot,\cdot)$ are our low-level and high-level objective functions respectively. We can use any convex loss functions in Eq.\eqref{eq:BO} for optimization (e.g. SVM objective function). In this paper, we use the leave-one-out cross-validation (\textbf{LOOCV}) in the high-level problem. Previous research \cite{kuzborskij2013stability} suggests that LOOCV can increase the robustness of the estimated hyperparameter especially on the small dataset.
In previous studies\cite{maclaurin2015gradient,Pedregosa16}, BO is a non-convex problem and can only obtain the local optimal solution. However, we will show that problem \eqref{eq:BO} is strongly convex and we are able to obtain its optimal solution. 
\subsection{Low-level optimization problem}
To better illustrate our learning scenario, we define our learning process as follows. Suppose we have $N$ visual categories and 
can obtain $N$ source binary classifiers $f'=\{f'_1,...,f'_N\}$ from the source domain. We want to train a target function $f$ consisting of $N$ binary classifiers $f=\{f_1,...,f_N\}$ using the target training set $D$ and the source models $f'$.
Specifically, in our BO problem Eq. \eqref{eq:BO}, for the low-level optimization, we consider the scenario where we have to train $N$ binary linear target models $f_i = w_ix+b_i$ so that for any $\{x_i,y_i\}_{i=1}^l \in D$, the adjusted result satisfies $f(x)+f'(x)\beta = y$. Let $D^{\backslash i} = D\backslash\{x_i,y_i\}$.
Then, we use mean square loss in the low-level objective function to optimize each target model $f_n$ with any given transfer parameter $\beta$:
\begin{equation}\label{eq:bo_low}
\begin{aligned}
\text{Low-level:}\quad&f^{\backslash i}(\beta) : \underset{w,b}{\min} \sum_n^N\frac{1}{2}||w_n||^2+\frac{C}{2}\sum_je^2_{jn}\\
%\left(Y_{jn}-f_n(x_j)-\beta_n f_n'(x_j)\right)^2\\
\text{s.t.} &\qquad f_n(x) = w_nx+b_n; \quad x_j \in D^{\backslash i}\\
&\qquad e_{jn} = Y_{jn}-f_n(x_j)-\beta_n f_n'(x_j)
\end{aligned}
\end{equation}
Here, $Y$ is an encoded matrix of $y$ using the one-hot strategy where $Y_{in} =1$ if $y_i=n$ and 0 otherwise.

The reason why we use the objective function \eqref{eq:bo_low} is that it can provide an unbiased closed form Leave-one-out error estimation for each binary model $f_n$\cite{cawley2006leave}. As a result, the high-level problem becomes a convex problem and we are able to estimate our transfer parameter easier.

Let $K(X,X)$ be the kernel matrix and $C$ be the penalty parameter in Eq.\eqref{eq:bo_low}. We have:
\begin{equation}\label{eq:linear}
\psi=\left[ 
{K(X,X) + \frac{1}{C}{\rm{I}}} \right]
\end{equation}
Let $\psi^{-1}$ be the inverse of matrix $\psi$ and  $\psi_{ii}^{-1}$ is the $ith$ diagonal element of $\psi^{-1}$. $\hat{Y}_{in}$, the LOO estimation of binary model $f^{\backslash i}_n$ for sample $x_i$, can be written as\cite{cawley2006leave}:
\begin{equation} \label{eq:loo}
{\hat Y_{in}} = {Y_{in}} - \frac{{{\alpha _{in}}}}{{\psi_{ii}^{ - 1}}}\quad {\text{for}}\quad n = 1,...,N
\end{equation}
where the matrix $\boldsymbol{\alpha}=\{\alpha_{in}|i=1,...l;n=1,...,N\}$ can be calculated as:
\begin{equation}
\boldsymbol{\alpha} =\psi^{-1} Y - \psi^{-1} f'(X)diag(\boldsymbol{\beta})
\end{equation}

\subsection{High-level optimization problem} 
For the high level optimization problem, we use multi-class hinge loss \cite{crammer2002algorithmic} with $\ell_2$ penalty in our objective function.
\begin{equation}\label{eq:bo_high}
\begin{aligned}
\text{High-level:}\quad&\beta: \min \frac{{{\lambda}}}{2}\sum\limits_{n}^N {{{\left\| {{\beta _n}} \right\|}^2}}  +
\sum_i\xi_i\\
%\sum\limits_{i,n}\left[ {1 - {\varepsilon _{n{y_i}}} + {{\hat Y}_{in}} - {{\hat Y}_{i{y_i}}} - {\xi _i}} \right]\\
\text{s.t.} \qquad& 1 - {\varepsilon _{n{y_i}}} + {\hat Y_{in}} - {\hat Y_{i{y_i}}} \le {\xi_i}
\end{aligned}
\end{equation}
Here, $\varepsilon _{n{y_i}}=1$ if $n=y_i$ otherwise 0.
$\lambda$ is used to balance the $\ell_2$ penalty and our multi-class hinge loss. 
Compared to the previous work \cite{kuzborskij2013n,tommasi2014learning} which uses the multi-class hinge loss without the $\ell_2$ penalty, there are two main advantages for our high-level objective function: 
\begin{enumerate}
	\item When the training set is small, our LOOCV estimation could have a large variance. It is important to add the $\ell_2$ penalty to {reduce the variance and improve the generalization ability of the estimated transfer parameter}.
	\item It is clear that $\hat{Y}$ is a linear function w.r.t. $\beta$. With the $\ell_2$ penalty, the high-level optimization problem \eqref{eq:bo_high} becomes a strongly convex optimization problem w.r.t. the transfer parameter $\beta$.Therefore, we can obtain an $O({\log(t)}/{t})$ optimal solution with $t$ iterations using Algorithm \ref{alg:1} (see proof of Theorem \ref{th:1} in Appendix).
\end{enumerate}



\begin{algorithm}\label{alg:1}
       \caption{EMTLe}\label{alg:1}
        \begin{algorithmic}[1]
            \REQUIRE $\lambda, \psi,Y,f',T$,
            \ENSURE $\beta=\left\{\beta^1,...,\beta^n\right\}$
            \STATE $\beta^0 = 1$
            ,$\alpha' = \psi^{-1}Y,\alpha'' = \psi^{-1}f'$
            \FOR {$t=1$ to $T$}
                \STATE $\hat Y \leftarrow Y - {\left( {\psi^{-1} \circ I} \right)^{ - 1}}\left( \alpha' - \alpha''diag(\beta) \right)$
                %\STATE ${\Delta _\beta }=0$ 
                \FOR {$i=1$ to $l$}
                	\STATE ${\Delta _\beta }=\lambda\beta$ 
                	%\FOR {$r=1$ to $N$}
	                    \STATE $l_{ir} = \max(1 - {\varepsilon _{{y_i}r}} + {\hat Y_{ir}} - {\hat Y_{i{y_i}}})$
	                    \IF{$l_{ir}>0$}
	                            \STATE $\Delta _\beta^{{y_i}} \leftarrow \Delta _\beta^{{y_i}} - \frac{{{\alpha''_{i{y_i}}}}}{{{\psi^{-1}_{ii}}}}$%
	                            , $\Delta _\beta^{{r}} \leftarrow \Delta _\beta^{{r}} + \frac{{{\alpha''_{i{r}}}}}{{{\psi^{-1}_{ii}}}}$%
	                    \ENDIF
	                 %\ENDFOR %class ends   
                \ENDFOR %examples ends
                \STATE $\beta^t  \leftarrow \beta^{(t-1)}  - \frac{{{\Delta _\beta }}}{{\lambda\times {t} }}$
             \ENDFOR %iteration ends
        \end{algorithmic}
\end{algorithm} 
%\input{smitle.tex}

\section{Experiments}\label{sec:exp}
In this section, we show empirical results of our algorithm for different transferring situations on two image benchmark datasets: Office and Caltech.
\subsection{Dataset \& Baseline methods}
Office contains 31 classes from 3 subsets (Amazon, Dslr and Webcam) and Caltech contains 256 classes. We select 13 shared classes from two datasets\footnote{13 classes include: backpack, bike, helmet, bottle, calculator, headphone, keyboard, laptop, monitor, mouse, mug, phone and projector}. The input features of all examples are extracted using AlexNet\cite{krizhevsky2012imagenet}.
%Because the two subsets Dslr and Webcam are relatively small and don't have data for testing, we only use them as the source domain.
\begin{table}[htbp]
	\centering
	\caption{Statistics of the datasets and subsets}
	\begin{tabular}{|c|c|c|c|c|}
		\hline
		Dataset&Subsets&\# classes &\# examples & \# features\\\hline
		\multirow{3}{*}{Office} & Amazon (A) &13&1173 & 4096\\
		
		& Dslr (D) &13&224 & 4096\\
		& Webcam (W) &13&369 & 4096\\
		\hline
		Caltech256&Caltech (C)&13&1582&4096\\
		\hline
	\end{tabular}%
	\label{tab:class_info}%
\end{table}%
We compare our algorithm EMTLe with two kinds of baselines. The first one is the methods without leveraging any source knowledge (no transfer baselines), including two methods. \textbf{No transfer:} SVMs trained only on target data. Any transfer algorithm that performs worse than it suffers from negative transfer. \textbf{Batch:} We combine the source and target data, assuming that we have full access to all data, to train the SVMs. The result of the Batch method is expected to outperform other methods under the HTL setting as it can access the source data. The second kind of baseline consists of two previous transfer methods in HTL, \textbf{MKTL\cite{jie2011multiclass}} and \textbf{Multi-KT\cite{tommasi2014learning}}. Similar to EMTLe, both of them use the LOOCV method to estimate the relatedness of the source model and target domain, but they use their convex objective function without the $\ell_2$ penalty terms. We use linear kernel for all methods in all our experiments.
\subsection{Transfer from Single Source Domain}
In this subsection, following the experiment protocol in \cite{jie2011multiclass,tommasi2014learning} for a fair comparison, we perform 12 groups of experiments under the setting of HTL. 
For each experiment, one of the 4 (sub)datasets is selected as the source, while another dataset is used as the target. We evaluate the performance of EMTLe when all source models are of the same type. As Multi-KT can only leverage knowledge when the source model is SVM, All source models are trained with linear SVMs.
The size of each target dataset is varied from 1 to 5 to see how EMTLe and other baselines behave under the extremely small dataset. We use a heuristic way to set the value of $\lambda$ in Eq. \eqref{eq:bo_high}:
\begin{equation}
\lambda = 2e^{err_{n}-err_{s}}
\end{equation}
where $err_{n}$ and $err_{s}$ denote the performance of ``No transfer'' and the source model on the training set.
We perform each experiment 10 times and report the average result in Figure \ref{fig:exp}. 
\begin{figure}[th]
\centering
\subfloat[C$\rightarrow$A]{
    \includegraphics[width=0.30\textwidth]{pakdd/fig/caltechtoamazon.png}\label{a}
}
\subfloat[D$\rightarrow$A]{
    \includegraphics[width=0.30\textwidth]{pakdd/fig/dslrtoamazon.png}\label{b}
}
\subfloat[W$\rightarrow$A]{
	\includegraphics[width=0.30\textwidth]{pakdd/fig/webcamtoamazon.png}\label{c}
}\\
\subfloat[A$\rightarrow$C]{
	\includegraphics[width=0.30\textwidth]{pakdd/fig/amazontocaltech.png}\label{d}
}
\subfloat[D$\rightarrow$C]{
	\includegraphics[width=0.30\textwidth]{pakdd/fig/dslrtocaltech.png}\label{e}
}
\subfloat[W$\rightarrow$C]{
	\includegraphics[width=0.30\textwidth]{pakdd/fig/webcamtocaltech.png}\label{f}
}\\
\subfloat[A$\rightarrow$D]{
	\includegraphics[width=0.30\textwidth]{pakdd/fig/amazontodslr.png}\label{g}
}
\subfloat[C$\rightarrow$D]{
	\includegraphics[width=0.30\textwidth]{pakdd/fig/caltechtodslr.png}\label{h}
}
\subfloat[W$\rightarrow$D]{
	\includegraphics[width=0.30\textwidth]{pakdd/fig/webcamtodslr.png}\label{i}
}\\
\subfloat[A$\rightarrow$W]{
	\includegraphics[width=0.30\textwidth]{pakdd/fig/amazontowebcam.png}\label{j}
}
\subfloat[C$\rightarrow$W]{
	\includegraphics[width=0.30\textwidth]{pakdd/fig/caltechtowebcam.png}\label{k}
}
\subfloat[D$\rightarrow$W]{
	\includegraphics[width=0.30\textwidth]{pakdd/fig/dslrtowebcam.png}\label{l}
}
\caption{Recognition accuracy for HTL domain adaptation from a single source. 5 different sizes of target training sets are used in each group of experiments. A, D, W and C denote the 4 subsets in Table \ref{tab:class_info} respectively.}
\label{fig:exp}
\end{figure}

\textbf{Observation \& discussion:} EMTLe can significantly outperform other baselines especially with a small training set. %Moreover, in some groups of experiments, they even suffer from negative transfer on the small training set. 
As we have discussed above, when the training set is small, with the transfer parameter estimated by our $\ell_2$ penalty in our high-level objective functions, EMTLe has a strong generalization ability and performs better on the test data. As the training size increases, the variance of training data decreases and the affect of the $\ell_2$ penalty term become less significant. Therefore, EMTLe and the other two HTL baselines show similar performance. 
It is interesting to see that MKTL even falls into negative transfer even with 5 training examples per class in some experiments. We found that, MKTL is more sensitive to the variance of the training data. Its performance is not as stable as Multi-KT and EMTLe over the 10 experiments. Because MKTL needs to learn more hyperparameters than Multi-KT and EMTLe, even though the training size increases, it may not be able to obtain a good model. 
In some experiments, we can see that EMTLe can even outperform the Batch method which can access more information and is expected to outperform the other methods under the setting of HTL.

\subsection{Transfer from Multiple Source Domains}
As we mentioned, EMTLe can exploit knowledge from different types of source classifiers which could greatly extend our choice of the source domain under the HTL setting. In this subsection, we show that EMTLe can successfully transfer the knowledge from two different types of source classifiers. Meanwhile, MKTL and ``No Transfer" are used as our baseline. 

In this experiment, we assume that there is no single source domain that can cover all 13 classes in our target domain and we have to select source models from different source domains. Specifically, the 13 classes are selected from two different domains separately (6 from DSLR and 7 from Webcam) according to Table \ref{tab:class_gen}. Similar to our previous experiment configurations, we only use Caltech and Amazon as the target domains. We show the experiment results in Figure \ref{fig:exp2}.
% Table generated by Excel2LaTeX from sheet 'Sheet1'
\begin{table}[htbp]
	\centering
	\caption{The selected classes of the two source domains and the classifier type of the source model.}
	\begin{tabular}{|c|c|c|}
		\hline
		& class & classifier\\
		\hline
		DSRL& monitor,bike, helmet,calcu,headphone,projector & Logistic\\\hline
		Webcam&keyboard,mouse,phone,backpack,mug,bottle,laptop&SVMs\\ \hline
		
	\end{tabular}%
	\label{tab:class_gen}%
\end{table}%
\begin{figure}[h]
	\centering
	\subfloat[D+W $\rightarrow$ A]{
		\includegraphics[width=0.5\textwidth]{pakdd/fig/multi_amazon.png}\label{a2}
	}
	\subfloat[D+W $\rightarrow$ C]{
		\includegraphics[width=0.5\textwidth]{pakdd/fig/multi_caltech.png}\label{b2}
	}\\
	\caption{Recognition Accuracy for Multi-Model \& Multi-Source experiment on two target datasets. }
	\label{fig:exp2}
\end{figure}

\textbf{Observation \& discussion:} In our multi-source scenario, it is more difficult to leverage the knowledge from the source models as the models are trained from different domains separately. From the results we can see that, EMTLe can still exploit the knowledge from the source models despite the types of the source classifiers while MKTL can hardly leverage the source knowledge. EMTLe uses a simple way to leverage the source models and BO can help us better estimate the transfer parameter. However, MKTL uses a sophisticated feature augmentation and has more hyperparameters to estimate. Without sufficient training data, it is difficult for MKTL to measure the importance of each source model and exploit the knowledge from the models.






\section{Summary}
In this chapter, we propose a method, EMTLe that can effectively transfer the knowledge under the HTL setting. We focus on the effectiveness and compatibility issues for HTL problems. We propose our auxiliary bias strategy to let our model exploit the knowledge from different types of source classifiers. The transfer parameter of EMTLe is estimated by bi-level optimization method using our novel high-level objective function which allows our model to better exploit the knowledge from source models. Experiment results demonstrate that EMTLe can effectively transfer the knowledge even though the size of training data is extremely small.


	\chapter{Fast Generalized Distillation for Semi-supervised Domain Adaptation}
\section{Introduction}

\section{Previous Work}\label{sec:aaai:work}
As we use GD to solve SDA problem, we introduce related work in both GD and SDA areas.

In SDA, previous work tried to utilize the unlabeled data to improve the performance.  \cite{yao2015semi} introduced a framework named Semi-supervised Domain Adaptation with Subspace Learning (SDASL) to correct data distribution mismatch and leverage unlabeled data. \cite{Donahue_2013_CVPR} proposed a framework for adapting classifiers by ``borrowing" the source data to the target domain using a combination of available labeled and unlabeled examples. \cite{daume2010frustratingly} show that augmenting the feature space of the data can compensate the domain shift.
\cite{duan2009domain} proposed a method using the unlabeled data to measure the mismatch between different domains based on the maximum mean discrepancy. %\cite{reddi2015doubly} proposed 

There are also many studies related to GD for computer vision tasks. \cite{Sharmanska_2013_ICCV} proposed a Rank Transfer method that uses attributes, annotator
rationales, object bounding boxes, and textual descriptions as the privileged information for object recognition. \cite{Motiian_2016_CVPR} proposed {the information bottleneck method with privileged information (IBPI)} that leverage the auxiliary information such as supplemental visual features, bounding box annotations and 3D skeleton tracking data to improve visual recognition performance. \cite{Tzeng_2015_ICCV} proposed a CNN architecture for domain adaptation to leverage the knowledge from limited or no labeled data using the soft label. \cite{urban2016deep} used a small shallow net to mimick the output of a large deep net while using layer-wised distillation with $\ell_2$ loss of the outputs of the student and teacher net. Similarly, \cite{luo2016face} used $\ell_2$ loss to train a compressed student model from the teacher model for face recognition. 

Compared to previous work on SDA, our method only requires the output of the source models, which is more effective when the size of the source domain is relatively large and the source model is well-trained. Compared to other work in GD, our method GDSDA-SVM can effectively estimate the imitation parameter while previous work was limited to using either a brutal force search or domain knowledge.

\section{Generalized Distillation for Semi-supervised Domain Adaptation}\label{sec:aaai:gdda}
As previously mentioned, GDSDA is a paradigm using generalized distillation for semi-supervised domain adaptation. In this section, we first give a brief review of generalized distillation. Then we show the process of GDSDA and demonstrate the reason why GDSDA can work for the SDA problem. Finally, we show the importance of the imitation parameter. 
\begin{figure}\label{fig:gd}
	\centering
	\includegraphics[scale=.6]{aaai/figure/GD.png}
	\caption{Illustration of Generalized Distillation training process.}
\end{figure}
\subsection{An overview of Generalized Distillation and GDSDA}
\textit{Distillation} \cite{hinton2015distilling} and \textit{Learning Using Privileged Information} (LUPI) \cite{vapnik2015learning} are two paradigms that enable machines to learn from other machines. Both methods address the problem of how to build a student model that can learn from the advanced teacher models. Recently, \cite{lopez2015unifying} proposed a framework called \textit{generalized distillation} that unifies both methods and show that it can be applied in many scenarios.

In GD, the training data can be represented as a collection of the triples:
\[\{\left(x_1,x_1^*,y_1\right),\left(x_2,x_2^*,y_2\right) \dots \left(x_n,x_n^*,y_n\right)\}\]
$x^*$ is the privileged information for data $x$, which is only available in the training set and $y$ is the corresponding label. Therefore, the goal of GD is to train a model, called student model with the guidance of the privileged information to predict the unseen example pair $(x,y)$.

The process of generalized distillation is as follows: in step 1, a teacher model ${f}^{(t)}$ is trained using the input-output pairs $\{x^*_i,y_i\}_{i=1}^n$. In step 2, use ${f}^{(t)}$ to generate the soft label $s_i$ for each training example $x_i$ using the softmax function $\sigma$:
\begin{equation}\label{eq:softmax_T}
s_i=\sigma(f^{(t)}(x_i)/T)
\end{equation}
where $T$ is a parameter called temperature to control the smoothness of the soft label. In step 3, learn the student ${f}^{(s)}$ from the pairs $\{\left(x_i,y_i\right),\left(x_i,s_i\right)\}_{i=1}^n$ using:
\begin{equation}\label{eq:distill}
\begin{aligned}
f^{(s)}=&\underset{f^{(s)} \in \mathcal{F}^{(s)}}{\arg \min}\frac{1}{n}\sum_{i=1}^{n}\bigg[\lambda\ell\left(y_i,\sigma(f^{(s)}(x_i))\right)
+(1-\lambda)\ell\left(s_i,\sigma(f^{(s)}(x_i))\right)\bigg]\\
\end{aligned}
\end{equation}
Here, $\ell(\cdot,\cdot)$ is the loss function and $\lambda$ is the imitation parameter to balance the importance between the hard label $y_i$ and the soft label $s_i$.

GD can be used in many scenarios such as multi-task learning, semi-supervised learning, and reinforcement learning. In domain adaptation, when we consider the source model as the teacher and the predictions of the target data given by the source models as the privileged information,
GD can be naturally applied to SDA. This leads to \textit{Generalized Distillation Semi-supervised Domain Adaptation} (\textbf{GDSDA}). Moreover, in GDSDA, we also consider the multi-source scenario and extend the GD paradigm to fit this scenario. To be consistent with other work of domain adaptation, we use the \textit{source model} and the \textit{target model} to denote the teacher model and the student model.
\begin{figure}
	\centering
	\includegraphics[scale=.5]{aaai/figure/multi-GDDA.png}
	\caption{Illustration of GDSDA training process and our ``fake label" strategy.}\label{fig:GDSDA}
\end{figure}

Technically, when we apply GD to SDA, according to Eq. \eqref{eq:distill}, each example is assigned with a hard label $y$ (true label) and a soft label $s$ (class probabilities from the teacher). However, in SDA, we are not able to obtain the hard labels of the unlabeled data. Here we follow the GD work\cite{lopez2015unifying} and use the ``fake label" strategy to label the unlabeled data: for the labeled examples, we use \textit{one-hot} strategy to encode their labels while using all 0s as the label of the unlabeled examples (see Fig \ref{fig:GDSDA}). Thus, each example in the target domain is assigned with a label. It is arguable that the ``fake label" strategy would introduce extra noise and degrade the performance. However, we will show in our experiment that this noise can be well controlled by setting a proper value to the imitation parameter and we can still achieve improved performance (See the single source experiment).

Suppose we have $M-1$ source domains denoted as $D_s^{(j)}=\{X^{(j)},Y^{(j)}\}_{j=1}^{M-1}$ and the target domain $D_t=\{X,Y\}$ encoded with the ``fake label" strategy. The process of GDSDA is as follows:
\begin{enumerate}
	\item Train the source models $f^*_j$ for each of the $M-1$ domains with $\{X^{(j)},Y^{(j)}\}$.
	\item For each of the training example $x_i$ in the target domain, generate the corresponding soft label $y^*_{ij}$ with each of the source model $f^*_j$ and the temperature $T>0$.
	\item Learn the target model $f_t$ using the $(M+1)$-tuples $\{x_i,y_i,y^*_{i1},\dots,y^*_{i(M-1)}\}_{i=1}^L$ with the imitation parameters $\{\lambda_i\}^M_{i=1}$ using \eqref{eq:GDDA_abs}:
\end{enumerate} 
\begin{equation}\label{eq:GDDA_abs}
\begin{aligned}
f_t(\lambda)=&\underset{f_t \in \mathcal{F}}{\arg \min}\frac{1}{L}\sum_{i=1}^{L}\bigg[\lambda_1\ell\left(y_i,f_t(x_i)\right)+\sum_{j=1}^{M-1}\lambda_{j+1}\ell\left(y^*_{ij},f_t(x_i)\right)\bigg]\\
\text{s.t.} & \qquad \sum_i\lambda_i=1
\end{aligned}
\end{equation}
Compared to other studies on SDA where each example of the source domain was used, by either re-weighting \cite{Donahue_2013_CVPR,duan2012visual} or augmentation \cite{daume2010frustratingly}, GDSDA only requires the trained model from the source domain to generate the soft labels. Considering that it is more convenient to access the source model than each of the examples of the source domain, GDSDA can be more useful than those previous methods. For example, if we want to use ImageNet \cite{imagenet_cvpr09} as the source domain, it is almost impossible to access each of the millions of examples while there are many well trained models publicly available online that can be used for GDSDA. Also, GDSDA is able to handle the multi-class scenario while previous methods, such as SHFA\cite{duan2012learning} only solved the binary classification problem of SDA. Moreover, GDSDA is compatible with any type of source model that is able to output the soft label (i.e., the class probabilities).

\subsection{Why does GDSDA work}
In this section, we demonstrate the scenarios where GDSDA can work. Before we provide our analysis, we first introduce two basic assumptions for GDSDA: the \textit{assumption of distillation} and the \textit{assumption of the source model}.

\textbf{Assumption of Distillation:} The capacity (or VC dimension) of the target model $f_t$ is smaller than the capacity of source model $f^*$. This assumption is inherited from distillation \cite{lopez2015unifying}.
\textbf{Assumption of the source model:} The source model $f^*$ should work better than a target model $f'_t$ trained only with the hard labels. This assumption is common, especially in SDA where the labeled examples are often too few to build a good target model. For example, when we only have one labeled example from each class in the target training set, it is reasonable to assume that the source model trained from another domain can perform better than the model trained only with the target training data on the target task. Based on these two assumptions, we will show that GDSDA can effectively leverage the source model and transfer the knowledge between different domains under the SDA setting.

According to the ERM principle\cite{vapnik1999overview}, a simple model has better generalization ability than the complex one, if they both have the same training error.
As long as the target model $f_t$ can achieve similar training error to that of the source model $f^*$ on the target domain, considering the fact that the VC dimension of $f_t$ is smaller than $f^*$, we can expect that the target model has better generalization ability. This process can be achieved by letting the target model mimick the output of the source model on the training data.
It is worthy to notice that in this process, the target model only has to mimick the output of the source model (soft label) without considering the hard labels of the examples. In another word, GDSDA provide an effective way to utilize the unlabeled data.

Arguably, because of the domain shift, the source model is biased towards the source domain when we apply it to the target task. However, as suggested in \cite{hinton2015distilling}, we can use labeled data from the target domain to compensate for the domain shift and achieve a better performance on the target task with Eq. \eqref{eq:GDDA_abs}. Specifically, we use the imitation parameter $\lambda$ to control the relative importance between the soft label and the hard label, which in turn reflects the similarity between the source and target tasks. 
For example, in Figure \ref{fig:GDSDA}, when we set $\lambda_2=0$, we actually ignore the knowledge from source domain 1.
As a result, GDSDA can compensate for the domain shift under the setting of SDA (for more details, please see the experiment section).

\subsection{Key parameter: the imitation parameter}\label{sec:aaai:key}
From the above analysis, we can see that GDSDA can effectively transfer the knowledge from the source to the target domain. In this section, we demonstrate that the imitation parameter can greatly affect the performance of the target model.

In GDSDA, we must decide the values of 2 parameters, the temperature $T$ and the imitation parameter $\lambda$. The temperature $T$ controls the smoothness of the soft label and the imitation parameter $\lambda$ controls how much knowledge can be transferred from the source model. Previous work has addressed the importance of knowledge control in domain adaptation \cite{duan2012learning,duan2012visual}. Without carefully controlling the amount of knowledge transferred from the source domain, the target model may suffer from degraded performance or even negative transfer \cite{pan2010survey}.
How to choose the imitation parameter is crucial for GDSDA. In previous work, however, the imitation parameter was determined by either a brute force search \cite{lopez2015unifying} or background knowledge \cite{Tzeng_2015_ICCV}. Meanwhile, in real applications, it is common that  multiple source domains can be exploited. As suggested by \cite{tommasi2014learning}, learning from multiple related sources simultaneously can significantly improve the performance of the target model. However, these previous methods become more difficult to apply when there are multiple sources and imitation parameters to be determined.
For these reasons, it is ideal to find an approach that can determine the imitation parameter automatically.

\section{GDSDA-SVM}\label{sec:aaai:svm}
As previously mentioned, it is important to find an approach that can effectively determine the imitation parameter. In this section, we propose our method GDSDA-SVM which uses SVM as the base classifier and can effectively estimate the imitation parameter by minimizing the cross-validation error on the target domain.
\subsection{Distillation with multiple sources}
As suggested in \cite{vapnik2015learning}, the optimal imitation parameter should be the one that can minimize the training error on the target domain. Based on that, we propose our method GDSDA-SVM which can effectively estimate the imitation parameter.

Instead of using hinge loss in our GDSDA-SVM, we use Mean Squared Error (MSE) as our loss function for the following two reasons: (1) several recently studies \cite{ba2014deep,luo2016face,romero2014fitnets,urban2016deep} show that MSE is also an efficient measurement for the target model to mimick the output of the source model. (2) MSE can provide a closed form cross-validation error estimation which allows us to estimate the imitation parameter effectively. 

Suppose we have $L$ examples $\{\textbf{x}_j,\textbf{y}_j\}_{j=1}^L$ from $N$ classes in the target domain where $X\in R^{L\times d}, Y\in R^{L\times N}$. Meanwhile, there are $M-1$ source (teacher) models providing the soft labels $Y^*=\{\textbf{y}^*_{ij}|j=1,...,L;i=1,...,M-1\}$ for each of the $L$ examples.
For simplicity, we concatenate the hard label $Y$ and soft label $Y^*$ into a new label matrix $S$, where:
\[S=[Y;Y^*]=[S_1;...;S_M]; S_i \in R^{L \times N}\]
To solve this $N$-class classification problem, we adopt the One-vs-All strategy to build $N$ binary SVMs.
To build the $n$th binary SVM, we have to solve the following optimization problem: 
\begin{equation}\label{eq:multi-distill}
\begin{aligned}
\underset{w_n}{\min} \qquad & \frac{1}{2}{|| \textbf{w}_n ||^2} + C\sum_{j}{e_{jn}^2} \quad\\
s.t. \qquad& e_{jn} = \sum_i\lambda_iS_{ijn} - \textbf{w}_n\textbf{x}_j%;\sum_i\lambda_i=1;\\
%& \sum_i\lambda_i=1; \lambda_i \in [0,1]; i\in M;  j\in L\\
\end{aligned}  
\end{equation}
We use the KKT theorem \cite{cristianini2000introduction} and add dual sets of variables to the Lagrangian of the optimization problem:
\begin{equation}
\begin{aligned}
\mathcal{L}=&\frac{1}{2}{|| \textbf{w}_n ||^2} + C\sum_{j} {e_{jn}^2}\\
&+\sum_{j}\eta_{jn}\left(\sum_i\lambda_iS_{ijn} - \textbf{w}_n\textbf{x}_j-e_{jn}\right)%+\beta^{(n)}\left(\sum_i\lambda_i-1\right)
\end{aligned}
\end{equation}
To find the saddle point, 
\begin{equation}
\begin{aligned}
\frac{{\partial \mathcal{L}}}{{\partial \textbf{w}_n}}& =0 \rightarrow \textbf{w}_n = \sum_{j}\eta_{jn} {\textbf{x}_j}; &
\frac{{\partial \mathcal{L}}}{{\partial {e_{jn}}}} & =0 \rightarrow \eta_{jn} = 2C {e_{jn}}\\
\end{aligned}
\end{equation}
For each example $\textbf{x}_j$ and its constraint of label $S_{ijn}$, we have $e_{jn}  + \textbf{w}_n\textbf{x}_j= \sum_i\lambda_iS_{ijn}$. Replacing $\textbf{w}_n$ and $e_{jn}$,  we have:
\begin{equation}
\begin{aligned}
\sum_j\eta_{jn}\textbf{x}_j\textbf{x}_i+\frac{\eta_{in}}{2C}&=\sum_i\lambda_iS_{ijn}\\
\end{aligned}
\end{equation}


Here we use $\Omega$ to denote the matrix $\Omega=[K+\frac{\mathbf{I}}{2C}]$ where $K$ is the linear kernel matrix $K=\{\textbf{x}_i\textbf{x}_j|i,j\in 1\dots L\}$. Let
$\Omega^{-1}$ be the inverse of matrix $\Omega$ and $\Omega^{-1}_{jj}$ be the $j$th diagonal element of $\Omega^{-1}$. We have $\eta = \sum_i\lambda_i\Omega^{-1}S_i=\sum_i\lambda_i\eta_i'$. 
According to  \cite{cawley2006leave}, the Leave-one-out (LOO) estimation of the example $\textbf{x}_j$ for the $n$th binary SVM can be written as:
\begin{equation}\label{eq:yhat}
\hat{y}_{jn} = \sum_i\lambda_i\left(S_{ijn}-\frac{{\eta'}_{ijn}}{\Omega_{jj}^{-1}}\right)
\end{equation}

Now for any given $\lambda$, we have found an efficient way to estimate the LOO prediction of each binary target model for example $\textbf{x}_j$. In the following section, we will introduce how to find the optimal $\lambda_i$ for each of the source models. 
\subsection{Cross-entropy loss for imitation parameter estimation}
From the previous section, we have already found an effective solution to estimate the output of the SVM. The optimal imitation parameters can be found by solving the following optimization problem:
\begin{equation}\label{eq:loo_loss}
\begin{aligned}
\min \quad& L_c\left(\lambda\right)=\frac{1}{2}\sum_i^M||\lambda_i||^2+\frac{1}{L}\sum_{j,n}\ell\left(y_{in},\hat{y}_{jn}\left(\lambda\right)\right)\\
s.t. \quad& \sum\lambda_i=1
\end{aligned}
\end{equation}
Here we use the $\ell$-2 regularization term to control the complexity of $\lambda$s so that the target model can achieve better generalization performance. For the loss function $\ell(\cdot,\cdot)$, We choose the cross-entropy loss function.
\begin{equation}\label{eq:ce}
\begin{aligned}
\ell\left(y_{in},\hat{y}_{jn}\left(\lambda\right)\right)=y_{in}\log(P_{jn}) \qquad
P_{jn} = \frac{e^{\hat{y}_{jn}}}{\sum_{h} e^{\hat{y}_{jh}}}
\end{aligned}
\end{equation}
Cross-entropy pays less attention to a single incorrect prediction which reduces the affect of the outliers in the training data. Moreover, cross-entropy works better for the unlabeled data with our ``fake label" strategy. As we mentioned in our ``fake label" strategy, we use 0s to encode the hard labels of the unlabeled examples. From \eqref{eq:ce} we can see that cross-entropy loss can automatically ignore penalties of the unlabeled examples and reduce the noise introduced by our ``fake label" strategy. 
Let:
\begin{equation}\label{eq:mu}
\begin{aligned}
\mu_{ijn}:=S_{ijn}-\frac{{\eta}'_{ijn}}{\Omega_{jj}^{-1}} \qquad
\end{aligned}
\end{equation}
The derivative can be written as:
\begin{equation}\label{eq:p}
\begin{aligned}
\frac{\partial \ell(\lambda)}{\partial \lambda_i}&=\sum_n\mu_{ijn}\left(P_{jn}-{y}_{jn}\right)
\end{aligned}
\end{equation}
\begin{algorithm}[t]
	\caption{GDSDA-SVM}\label{alg:svm}
	\begin{algorithmic}
		\REQUIRE Input examples $X=\{\textbf{x}_1,...,\textbf{x}_L\}$, number of classes $N$, number of sources $M$, 3D label matrix, $S=[Y_1,Y_2,...,Y_{M}]$ with size $L\times M \times N$, temperature $T$ %optimization iteration $iter$
		\ENSURE Target model $f_t = Wx$
		\STATE $\Omega=[K+\frac{\mathbf{I}}{2C}]$
		\STATE Find the imitation parameter $\lambda$ with Algorithm \ref{alg:lambda}
		\STATE Generate new label $Y_{new}=\sum_i\lambda_iS_i$
		\STATE Calculate $\eta = \Omega^{-1}Y_{new}$
		\STATE Calculate $w_n = \sum_j \eta_{jn}x_j$
	\end{algorithmic}	
\end{algorithm}
\begin{algorithm}[t]
	\caption{$\lambda$ Optimization}\label{alg:lambda}
	\begin{algorithmic}
		\REQUIRE Input examples $X$, number of classes $N$, size of sources $M$, 3D label matrix $S$, temperature $T$, optimization iteration $iter$, Kernel matrix $\Omega$
		\ENSURE Imitation parameter $\lambda$
		\STATE Initialize $\lambda = \frac{1}{M}$, 
		
		\STATE Let $S_n$ be the label matrix of $S$ for class $n$
		\FOR{Each label $S_n$} 
		\STATE Calculate $\eta'_n=\Omega^{-1}S_n$ 
		\ENDFOR
		\STATE Calculate $\mu$ using \eqref{eq:mu}
		\FOR {$it \in \{1,...,iter\}$ }
		\STATE Compute $\hat{y}_{jn}$ and $P_{jn}$ with \eqref{eq:yhat}  and \eqref{eq:ce}
		\STATE $\Delta_{\lambda} \leftarrow 0$
		\FOR {each $\textbf{x}_j$ in $X$}
		\STATE $\Delta_{\lambda} = \Delta_{\lambda}+\sum_n\mu_{ijn}\left(P_{jn}-{y}_{jn}\right)$
		\ENDFOR
		\STATE $\Delta_{\lambda} =\Delta_{\lambda}/L$, $\lambda = \lambda - \frac{1}{it}(\Delta_{\lambda}+\lambda)$
		\STATE $\lambda = \lambda / \sum\lambda_i$
		%	    \State 
		\ENDFOR
	\end{algorithmic}	
\end{algorithm}
We describe GDSDA-SVM in Algorithm \ref{alg:svm}. As the optimization problem \eqref{eq:loo_loss} is strongly convex, it is easy to prove that Algorithm \ref{alg:lambda} can converge to the optimal $\lambda$ with the rate of $O(\log(t)/t)$ where $t$ is the optimization iteration. Due to the space limit, we are not able to provide the proof here. 

\section{Experiments}\label{sec:aaai:exp}
In this section, we show the empirical performance of our algorithm GDSDA-SVM on the Office benchmark dataset. Specifically, we provide the empirical results under two transfer scenarios: single source and multi-source transfer scenarios for GDSDA-SVM.

\textbf{Dataset:}
We use the domain adaptation benchmark dataset Office as our experiment dataset. 
There are 3 subsets in Office dataset, Webcam (795 examples), Amazon (2817 examples) and DSLR (498 examples), sharing 31 classes. We denote them as W, A and D respectively. In our experiments, we use DSLR and Webcam as the source domains and Amazon as the target domain.
We use the features extracted from Alexnet \cite{KrizhevskyNIPS12} FC7 as the input feature for both source and target domain. The source models are trained with multi-layer perception (MLP) on the whole source dataset. 

\subsection{Single Source for Office datasets}
In this experiment, we compare our algorithm under the scenario where the source model is trained from a single source dataset. Specifically, we have two groups of experiments, transferring from Webcam to Amazon and from DSLR to Amazon. As we mentioned, there are significantly fewer labeled examples than unlabeled ones in real SDA applications.
Therefore, in each group of experiment, there are only 31 labeled examples (1 per class) and some unlabeled examples (10, 15 and 20 per class) in the target domain.

To demonstrate the effectiveness of GDSDA-SVM, we show the performance of GDSDA using brute force to search the imitation parameter as the baseline. As there are two imitation parameters in this experiment, we use $\lambda_1$ and  $1-\lambda_1$ to denote the imitation parameter for hard and soft label respectively. Specifically, we search the imitation parameter $\lambda_1$ in the range $[0,0.1,...,1]$ with different temperature $T$. Meanwhile, we show the performance of the source model (denoted as ``Source") and the performance of a target model (denoted as ``No transfer" using LIBLINEAR\cite{fan2008liblinear}) trained with only labeled examples of the target domain on the target task. We run each experiment 10 times and report the average result. For GDSDA-SVM, as we are not able to tune the temperature $T$, we empirically set $T=20$ for all experiments in this subsection. The experimental results are shown in Figure \ref{fig:single1}. 

From the results of the brutal force search we can see that, the value of imitation parameter can greatly affect the performance of the target model.
As we expected, without using any true label information of the target data, i.e. $\lambda_1 = 0$, GDSDA can still slightly outperform the source model. This means GDSDA can effectively transfer the knowledge between different domains with the unlabeled data. As we increase the value of imitation parameter, i.e. considering the hard labels from the target domain, the performance of GDSDA can be further improved. As we mentioned before, even though our ``fake label" strategy would introduce extra noise, the noise can be limited by setting a proper value to imitation parameter and the target model can still achieve improved performance compared to the baselines.

\begin{figure}[h]
	\centering
	\includegraphics[scale=.50]{aaai/figure/cmp.png}
	\caption{D+W$\rightarrow$A, Multi-source results comparison.}\label{fig:multi}
\end{figure}
\begin{figure}[t]
	\centering
	\begin{tabular}{cccc}
		\subfloat[D $\rightarrow$ A, 10 unlabeled ]{\includegraphics[width=0.45\textwidth]{aaai/figure/dslrtoamazonlabeled1unlabeled10.png}}&
		\subfloat[D $\rightarrow$ A, 15 unlabeled ]{\includegraphics[width=0.45\textwidth]{aaai/figure/dslrtoamazonlabeled1unlabeled15.png}}\\
		\subfloat[D $\rightarrow$ A, 20 unlabeled ]{\includegraphics[width=0.45\textwidth]{aaai/figure/dslrtoamazonlabeled1unlabeled20.png}}&
		\subfloat[W $\rightarrow$ A, 10 unlabeled ]{\includegraphics[width=0.45\textwidth]{aaai/figure/webcamtoamazonlabeled1unlabeled10.png}}\\
		\subfloat[W $\rightarrow$ A, 15 unlabeled ]{\includegraphics[width=0.45\textwidth]{aaai/figure/webcamtoamazonlabeled1unlabeled15.png}}&
		\subfloat[W $\rightarrow$ A, 20 unlabeled ]{\includegraphics[width=0.45\textwidth]{aaai/figure/webcamtoamazonlabeled1unlabeled20.png}}\\
	\end{tabular}
	\caption{Experiment results on DSLR$\rightarrow$Amazon and Webcam$\rightarrow$Amazon when there are just one labeled examples per class. The X-axis denotes the imitation parameter of the hard label (i.e. $\lambda_1$ in Fig \ref{fig:GDSDA}) and the corresponding imitation parameter of the soft label is set to $1-\lambda_1$.
	}\label{fig:single1}
\end{figure}
Moreover, we can see that GDSDA-SVM can achieve competitive results compared to baselines using brutal force search in D$\rightarrow$A experiments. In W$\rightarrow$A experiments, it achieves the best performances on all 3 different unlabeled sizes. This indicates that we can efficiently (about 6 times faster than the brutal force search) obtain a good target model with GDSDA-SVM.
%\newpage
%\subsubsection{From DSLR to Amazon}


\subsection{Multi-Source for Office datasets}
In this experiment, we show the performance of GDSDA-SVM under the multi-source SDA scenario.
Specifically, we use Amazon as the target domain which can leverage the knowledge of two source models trained from Webcam and DSLR.
We use the similar settings as our single source experiment and perform 2 groups of experiments using 1 labeled and 2 labeled examples per class respectively. We use temperature $T=5$. The results of multi-source GDSDA-SVM are denoted as SVM\_Multi. Here we also include two single source GDSDA-SVMs obtained from the experiments above (SVM\_w and SVM\_d trained using Webcam and DSLR as the source respectively) as the baselines. Moreover, we show the best performance of the brutal force search model (SVM\_BF). For SVM\_BF, we search temperature in range $T=[1,2,5,10,20,50]$ and each imitation parameter in range $[0,0.1,...,1]$. The experiment results are shown in Figure \ref{fig:multi}.

From the results, we can see that, given 2 source models, SVM\_Multi can outperform any single source model trained with GDSDA. This indicates GDSDA-SVM can still exploit the knowledge even in the complex multi-source scenario. Even though SVM\_Multi performs slightly worse than the best result found by brutal force search in some experiments, considering their time consumption (GDSDA-SVM is around 30 times faster than brutal force search), SVM\_Multi still has its advantage in real applications.

\section{Summary}\label{sec:aaai:con}
In this paper, we propose a novel framework called \textit{Generalized Distillation Semi-supervised Domain Adaptation} (GDSDA) that can effectively leverage the knowledge from the source domain for SDA problem without accessing to the source data. To make GDSDA more effective in real applications, we proposed our method called GDSDA-SVM and show that GDSDA-SVM can effectively determine the imitation parameter for GDSDA. 
In our future work, we plan to use a more advanced hyperparameter optimization method, which can optimize the imitation parameter $\lambda$ and the temperature $T$ in GDSDA simultaneously, and expect
further performance improvement 
	\chapter{Learning Food Recognition Model with Deep Representation}

\section{Introduction}
%Deep Learning with Convolutional Neural Networks (CNNs) is the most popular method for image recognition and has been applied to solve many real problems. In this chapter, we investigate problem of using the pre-trained CNNs for transfer learning. In particular, we fine-tune the parameters of the pre-trained CNNs on two food image database and achieve the improved results. We also investigate the changes of parameter after fine-tuning and try to obtain some important experience on fine-tuning the deep CNNs.
 



\section{CNN layers:conv/pool/norm etc}
A CNN consists of a number of convolutional and subsampling layers optionally followed by fully connected layers. In this part, we introduce the layers used in our work.
\subsection{Convolutional Layer}
Convolutional Layer is the core building block of a Convolutional Network, and its output volume can be interpreted as holding neurons arranged in a 3D volume.
Natural images have the property of being "stationary", meaning that the statistics of one part of the image are the same as any other part. This suggests that the features that we learn at one part of the image can also be applied to other parts of the image, and we can use the same features at all locations.

Formally, given some original $h\times w$ input images $I$, we can train a small autoencoder from $a \times b$ kernel matrix. Also, we have to set other hyperparameters, stride $s$ and padding $p$. Stride defines the number of pixels the kernel should be moved in each step around the image $I$ and padding defines the number of rows/columns padded to the height and width of the original input (see Figure \ref{fig:cnn:conv}). 
Given a $a \times b$ kernel matrix $W^{(1)}$, bias $b^{(1)}$, padding $p$ and stride $s$, we can encode the original image $I$ as $f_{conv}=sigmod(W^{(1)}I_p+b^{(1)})$ for $I_p \in I$, giving us $f_{conv}$ (called feature map of $W^{(1)}$), a  $\left\lceil\frac{(h-a+2p)}{s}+1\right\rceil\times\left\lceil\frac{(w-b+2p)}{s}+1\right\rceil$ array of feature map. In general, for any specific input $I$ ($h \times w \times c$ array matrix) of a convolutional layer $L$, assuming we have $k$ such $a \times b \times c$ kernel matrix, its feature maps $f$ should be a $\left\lceil\frac{(h-a+2p)}{s}+1\right\rceil\times\left\lceil\frac{(w-b+2p)}{s}+1\right\rceil \times k$ array matrix with $f_{conv}^{(i)}=sigmod(W^{(i)}I_p+b^{(i)})$ for $I_p \in I$ and $i \in 1,\dots k$.

In real applications, small kernels ($3\times3$, $5\times5$ and $7\times7$) are preferred by many different CNN architectures \cite{krizhevsky2012imagenet} \cite{lecun1998gradient} \cite{simonyan2014very} \cite{zeiler2014visualizing}. Recent development of tiny $1\times1$ kernel shows an improvement on both accuracy and computational efficiency \cite{szegedy2014going}.
\begin{figure}
\centering
\includegraphics[scale=.6]{cnn/fig/conv.png}
\caption{Convolution operation with $3\times3$ kernel, stride 1 and padding 1. $\otimes$ denotes the convolutional operator.} \label{fig:cnn:conv}
\end{figure}

\subsection{Pooling Layer}
Pooling layer is widely used in all kinds of CNN architecture for dimensional reduction and computational efficiency. 
After obtaining features maps using convolutional layer, we need to use them for classification. However, applying the feature maps from convolutional layer for classification would be a computationally challenge. Consider for instance images of size $96\times96$ pixels, and suppose we have learned 400 features over $8\times8$ inputs. Each convolution results in an output of size $(96-8+1)\times(96-8+1)=7921$, and since we have 400 features, this results in a vector of $892\times 400=3,168,400$ features per example. Learning a classifier from over 3 millions features could lead to severe over-fitting.

Therefore, it is common to periodically insert a (Max) Pooling layer in-between successive convolutional layers in CNN architecture. Its function is to progressively reduce the spatial size of the representation to reduce the amount of parameters and computation in the network, and hence to also control overfitting. The pooling layer works independently on the channel dimension and resize the feature map spatially. For certain $h \times w \times c$ input array matrix, a $a \times b$ Pooling layer with stride $s$ and $p$ padding would output a $\left\lceil\frac{(h-a+2p)}{s}+1\right\rceil\times\left\lceil\frac{(w-b+2p)}{s}+1\right\rceil \times c$ matrix array.

In general, two kinds of pooling strategy, Max Pooling and Average Pooling, are commonly used in CNN architecture (see Figure \ref{fig:cnn:pool}). Average pooling was often used historically but has recently fallen out of favor compared to the max pooling operation, which has been shown to work better in practice \cite{malmaud2015s} \cite{szegedy2014going}. Max Pooling is been widely used in all kinds CNN architectures \cite{boureau2010theoretical} \cite{yang2009linear}. 
\begin{figure}
\centering
\includegraphics[scale=.8]{cnn/fig/pool.png}
\caption{$2\times2$ pooling layer with stride 2 and padding 0.}\label{fig:cnn:pool}
\end{figure}

\subsection{Fully Connected Layer}
Fully Connected (FC) Layer have full connections to all activations in the previous layer, as seen in regular Neural Networks. Recent work show that FC layers with Rectified Linear Units and Dropout can greatly improve the learning speed as well as avoid overfitting for deep CNNs \cite{hinton2012improving} \cite{nair2010rectified}.
\subsubsection{Rectified Linear Units (ReLUs) for Activation}
Rectified Linear Units can be considered as replacing each binary unit with sigmoid activation by an infinite number of copies that all have the same weights but have progressively more negative biases. This replacing procedure can be mathematically presented as:
\begin{equation}
\sum\limits_i^N {\sigma (x - i + 0.5)}  \approx \log (1 + {e^x})
\end{equation}
where $\sigma(x)$ is the sigmoid function. In practice, Rectified Linear Units use the function 
\begin{equation}
f(x) = \log (1 + {e^x}) \approx \max(x,0) 
\end{equation}\label{eq:cnn:relu}
as the activation function for approximation \cite{jarrett2009best}. With $max$ function, the derivatives of the active ($x>0$) and inactive neurons are 1 and 0 respectively.  As a result, ReLUs can speed up the learning procedure greatly and improve the performance.
\subsubsection{DropOut}
In FC layer, nodes are connected to each other and this leads to a large number of parameters. Generally, larger number of parameters means more power for Neural Networks and more easily prone to overfitting. Dropout is a technique for addressing this problem.
The key idea is to randomly drop units (along with their connections) from the neural network during training \cite{srivastava2014dropout}. Technically, dropout can be interpreted as adding extra noise into the training procedure. Without actually adding noise, FC layer with dropout is tolerant of higher level of noise (20 \%-50\%). Randomly dropping out the nodes, for any node in FC layer, it can't rely on the other nodes to adjust its result. By eliminating the co-adaptation of hidden units, dropout becomes a technique that can be applied to any general domain and improve the performance of neural nets.   
\begin{figure}
	\centering
	\begin{tabular}{cc}
		\subfloat[ Standard Neural Net ]{    \includegraphics[width=0.3\textwidth]{cnn/fig/net.png}}&
		\subfloat[After Dropout]{    \includegraphics[width=0.3\textwidth]{cnn/fig/dropnet.png}} \\
	\end{tabular}
\end{figure}
\section{Datasets}
In this section, we introduce some details about the two architectures and the food datasets used in our experiments.
\subsection{Models}
In this paper, AlexNet and GoogLeNet are their Caffe \cite{jia2014caffe} implementations and all the results for a specific CNN architecture are obtained from single model.

\textbf{AlexNet}
 contains 5 layers followed by the auxiliary classifier which contains 2 fully connected layers (FC) and 1 softmax layer. Each of the first two layers can be subdivided into 3 components: convolutional layer with rectified linear units (ReLUs), local response normalization layer (LRN) and max pooling layer. Layer 3 and layer 4 contain just convolutional layer with ReLUs while layer 5 is similar to the first two layers except for the LRN. For each of the fully connected layer, 1 ReLUs and 1 dropout \cite{srivastava2014dropout} layer are followed.

 \textbf{GoogLeNet}
  shows another trend of deep CNN architecture with lots of small receptive fields. There are 9 Inception modules in GoogLeNet and Figure \ref{incept} shows the architecture of a single inception module. Inspired by \cite{linNiN}, lots of $1\times 1$ convolutional layers are used for computational efficiency. Another interesting feature of GoogLeNet is that there are two extra auxiliary classifiers in intermediate layers. During the training procedure, the loss of these two classifiers are counted into the total loss with a discount weight 0.3, in addition with the loss of the classifier on top. More architecture details can be found from \cite{szegedy2014going}.
%\begin{figure}
%\centering
%\includegraphics[scale=.11]{cnn/fig/g_v.pdf}\\
%\end{figure}
\begin{figure}
  \centering
  % Requires \usepackage{graphicx}
  \includegraphics[scale=.7]{cnn/fig/inception.pdf}\\
  \caption{Inception Module. $n\times n$ stands for size $n$ receptive field, $n\times n\_reduce$ stands for the $1\times 1$ convolutional layer before the $n\times n$ convolution layer and $pool\_proj$ is another $1\times 1$ convolutional layer after the MAX pooling layer. The output layer concatenates all its input layers.}\label{incept}
\end{figure}

\subsection{Food Datasets}
Besides ImageNet dataset, there are many popular benchmark datasets for image recognition such as Caltech dataset and CIFAR dataset, both of which contain hundreds of classes. However, in this paper, we try to focus on a more specific area, food classification. Compared to other recognition tasks, there are some properties of the food (dishes) which make the tasks become a real challenge:
\begin{itemize}
  \item Food doesn't have any distinctive spatial layout: for other tasks like scene recognition, we can always find some discriminative features such as buildings or trees, etc;
  \item Food class is a small sub-category among all the categories in daily life, so the inter-class variation is relatively small; on the other hand, the contour of a food varies depending on many aspects such as the point of the view or even its components.
\end{itemize}
These properties make food recognition catastrophic for some recognition algorithms. Therefore, the training these two architectures on the food recognition task can reveal some important aspects of themselves and help people better understand them. In this paper, we use two image datasets Food-256 \cite{kawano14c}\footnote{Dataset can be found http://foodcam.mobi/dataset.html} and Food-101 \cite{bossard2014food}\footnote{Dataset can be found http://www.vision.ee.ethz.ch/datasets\_extra/food-101}. It is worthy to mention that PFID dataset is also a big public image database for classification, but their images are collected in a laboratory condition which is considerably not applicable for real recognition task.

\textbf{Food-256 Dataset.}
This is a relatively small dataset containing 256 kinds of foods and 31644 images from various countries such as French, Italian, US, Chinese, Thai, Vietnamese, Japanese and Indonesia. The distribution among classes is not even and the biggest class (vegetable tempura) contains 731 images while the smallest one contains just 100 images. For this small dataset, we randomly split the data into training and testing set, using around 80\% (25361 images) and 20\% (6303 images) of the original data respectively and keep the class distribution in these two sets uniform. The collector of this dataset also provides boundary box for each image to separate different foods and our dataset is cropped according to these boundary boxes.

\textbf{Food-101 Dataset.}
This dataset contains 101-class real-world food (dish) images which were taken and labeled manually. The total number of images is 101,000 and there are exactly 1000 images for each class. Also, each class has been divided into training and testing set containing 750 images and 250 images respectively by its collector. The testing set is well cleaned manually while the training set is not well cleaned on purpose. This noisy training set is more similar to our real recognition situation and it is also a good way to see the effect of the noise on these two architectures.

\subsection{Data Argumentation}
In this section, we introduce some data argumentation methods in our work to enrich our training data. Previous work shows that with intensive argumentation for the training data, the performance of CNN model can be improved \cite{wu2015deep}. 
Data argumentation is an efficient way to enrich the data. There are also some techniques that can applied to enlarge the dataset such as subsampling and mirroring. The original images are firstly resized to $256\times 256$ pixels. We crop the 4 corners and center for each image according to the input size of each model and flap the 5 cropped images to obtain 10 crops. For the testing set, the prediction of an image is the average prediction of the 10 crops (see Figure \ref{fig:cnn:crop}).

\begin{figure}
	\centering
	\begin{tabular}{cccc}
	\multicolumn{2}{c}{\subfloat[Original image]{    \includegraphics[width=0.15\textheight]{cnn/fig/crop00.png}}}&
	\subfloat[Center]{    \includegraphics[width=0.15\textheight]{cnn/fig/crop4.png}}&
	\subfloat[Center mirror]{    \includegraphics[width=0.15\textheight]{cnn/fig/crop9.png}}  
		\\	
	
	\subfloat[Up-left]{    \includegraphics[width=0.15\textheight]{cnn/fig/crop0.png}}&
	\subfloat[Up-left mirror]{    \includegraphics[width=0.15\textheight]{cnn/fig/crop5.png}} & \subfloat[Up-right]{    \includegraphics[width=0.15\textheight]{cnn/fig/crop1.png}}&
	\subfloat[Up-right mirror]{    \includegraphics[width=0.15\textheight]{cnn/fig/crop6.png}}\\
	
	\subfloat[Bottom-left]{    \includegraphics[width=0.15\textheight]{cnn/fig/crop2.png}}&
	\subfloat[Bottom-left mirror]{    \includegraphics[width=0.15\textheight]{cnn/fig/crop7.png}} & \subfloat[Bottom-right]{    \includegraphics[width=0.15\textheight]{cnn/fig/crop3.png}}&
	\subfloat[Bottom-right mirror]{    \includegraphics[width=0.15\textheight]{cnn/fig/crop8.png}}\\

	\end{tabular}
	\caption{Crop area from original image}\label{fig:cnn:crop}
\end{figure}

Before cropping subsamples from the original image, we also use other argumentation methods such as color casting and vignette etc., to enrich our data and make our model less sensitive to lighting changes and other invariance (see Figure \ref{fig:cnn:argu}). 

Compared to color shifting in \cite{krizhevsky2012imagenet}, we use color casting to alter the intensities of the RGB channels in training images. For each image, we firstly use a random boolean parameter to determine whether its R, G, and B channel should be changed. For any channel that should be changed, we add a random integer ranging from $[-20 , 20]$ to this specific channel. We also apply vignetting effect to the original image. In our implementation, we apply a 2D Gaussian kernel on the original image for vignetting. The two parameter $\sigma_x$ and $\sigma_y$ are randomly chosen from $[160,200)$. We also apply some geometric transformation such as stretching and rotation, on the original image for data argumentation. In summary, we enriched the data by 11 times, 3 times color shifting, 2 times vignetting, 4 times stretching and 1 time rotation and plus the original image.






\begin{figure}
  \centering
  \subfloat[Original image]{    \includegraphics[width=0.3\textwidth]{cnn/fig/org.png}  }
  \subfloat[Red casting]{    \includegraphics[width=0.3\textwidth]{cnn/fig/red.png}  }
  \subfloat[Green casting]{    \includegraphics[width=0.3\textwidth]{cnn/fig/green.png}  }\\
  \subfloat[Blue casting]{    \includegraphics[width=0.3\textwidth]{cnn/fig/blue.png}  }
  \subfloat[RGB casting]{    \includegraphics[width=0.3\textwidth]{cnn/fig/rgb.png}  }
  \subfloat[Vignette]{    \includegraphics[width=0.3\textwidth]{cnn/fig/v1.png}  }\\
  \subfloat[More vignette]{    \includegraphics[width=0.3\textwidth]{cnn/fig/v2.png}  }
  \subfloat[Horizontal stretch]{    \includegraphics[width=0.3\textwidth]{cnn/fig/s1.png}  }
  \subfloat[More horizontal stretch]{    \includegraphics[width=0.3\textwidth]{cnn/fig/s2.png}  }\\
  \subfloat[Vertical stretch]{    \includegraphics[width=0.3\textwidth]{cnn/fig/s4.png}  }
  \subfloat[More vertical stretch]{    \includegraphics[width=0.3\textwidth]{cnn/fig/s3.png}  }
  \subfloat[Rotation]{    \includegraphics[width=0.3\textwidth]{cnn/fig/rota.png}  }\\

  \caption{Different data argumentation methods}\label{fig:cnn:argu}
\end{figure}

\section{Experimental Discuss}
Training a CNN with millions of parameters on a small dataset could easily lead to horrible overfitting. But the idea of supervised pre-training on some huge image datasets could preventing this problem in certain degree. Compared to other randomly initialized strategies with certain distribution, supervised pre-training is to initialize the weights according to the model trained from a specific task. Indeed, initialization with pre-trained model has certain bias as there is no single dataset including all the invariance for natural images \cite{agrawal2014analyzing}, but this bias can be reduced as the pre-trained image dataset increases and the fine-tuning should be benefit from it.
\subsection{Pre-training and Fine-tuning}
We conduct several experiments on both architectures and use different training initialization strategies for both Food-256 and Food-101 datasets. The scratch models are initialized with Gaussian distribution for AlexNet and Xavier algorithm for GoogLeNet%, which automatically determines the scale of initialization based on the number of input and output neurons
 \cite{glorot2010understanding}. These two initializations are used for training the original models for the ImageNet task. The ft-last and fine-tuned models are initialized with the weights pre-trained from ImageNet dataset. For the ft-last model, we just re-train the fully connected layers while the whole network is fine-tuned for the fine-tune model.
\begin{table}[htbp]
  \centering
  \caption{Top-5 Accuracy in percent on fine-tuned, ft-last and scratch model for two architectures}
    \begin{tabular}{|l|cc|cc|}
    \hline
          & \multicolumn{2}{c|}{AlexNet} & \multicolumn{2}{c|}{GoogLeNet} \\  \hline 
     & Food-101   & Food-256   & Food-101   & Food-256 \\\hline
    Fine-tune & \textbf{88.12} & \textbf{85.59} & \textbf{93.51} & \textbf{90.66} \\\hline
    Ft-last &76.49	&79.26&	82.84	&83.77\\\hline
    Scratch & 78.18 & 75.35 & 90.45 & 81.20 \\\hline
    \end{tabular}%
  \label{tab:ft}%
\end{table}%


% Table generated by Excel2LaTeX from sheet 'Sheet1'
\begin{table}[htbp]
  \centering
  \caption{Accuracy compared to other method on Food-256 dataset in percent}
    \begin{tabular}{|c|C{3cm}|c|c|}
    \hline
     & fv+linear \cite{Kawano:2014} & GoogLeNet & AlexNet \\\hline
     
    Top1  & 50.1& \textbf{70.13} & 63.82 \\\hline
    Top5  & 74.4  & \textbf{90.66} & 85.59\\\hline
    \end{tabular}%
  \label{tab:256}%
\end{table}%

% Table generated by Excel2LaTeX from sheet 'Sheet1'
\begin{table*}[htbp]
  \centering
  \caption{Top-1 accuracy compared to other methods on Food-101 dataset in percent}
    \begin{tabular}{|c|C{3cm}|C{3cm}|c|c|}
		\hline
          & RFDC\cite{bossard2014food} & MLDS($\approx$\cite{singh2012unsupervised}) & GoogLeNet & AlexNet \\\hline

    Top1 accuracy & 50.76 & 42.63& \textbf{78.11 }& 66.40 \\\hline

    \end{tabular}%
    \label{tab:101}
\end{table*}%
\begin{figure*}[htbp]
  \centering
  % Requires \usepackage{graphicx}
  \includegraphics[scale=0.5]{cnn/fig/sashimi.png}\\
  \caption{Visualization of some feature maps of different GoogLeNet models in different layers for the same input image. 64 feature maps of each layer are shown. Conv1 is the first convolutional layer and Inception\_5b is the last convolutional layer. }
   \label{fig:sashimi}
\end{figure*}
From Table \ref{tab:ft} we can see that fine-tuning the whole network can improve the performance of the CNN for our task. Compared to other traditional computer vision methods (see Table \ref{tab:256} and \ref{tab:101}), GoogLeNet outperforms the other methods with large margins and we provide the state-of-the-art performance of these two food image datasets.

In Figure \ref{fig:sashimi} we visualize the feature maps of the pre-trained GoogLeNet model and fined-tuned GoogLeNet model with the same input image for some layers. We can see that the feature maps of the lower layer are similar as the lower level features are similar for most recognition tasks.
Then we can see that the feature maps in the high-level are different which leads to totally different recognition results.
Since only the last layer (auxiliary classifier) of the ft-last model is optimized, we can infer that the higher level features are more important which is consistent with our intuition. Also from Table \ref{tab:ft}, it is interesting to see that for the Food-101 task, the accuracy of  the scratch models outperforms the pre-trained models. Since Food-101 is a relatively large dataset with 750 images per class while Food-256 dataset is an imbalanced small one, this indicates that it is difficult to obtain a good deep CNN model while the data is insufficient.

From Table \ref{tab:ft} we can see that GoogLeNet always performances better than AlexNet on both datasets. This implies that the higher level features of GoogLeNet are more discriminative compared to AlexNet and this is due to the special architecture of its basic unit, Inception module. Table \ref{tab:cosg} and \ref{tab:cosa} show the weights' cosine similarity of each layer between the fine-tuned models and their pre-trained models. From the results we can see that the weights in the low layer are more similar which implies that these two architectures can learn the hierarchical features. As the low level features are similar for most of the tasks, the difference of the objects is determined by high-level ones which are the combination of these low level features. Also from Table \ref{tab:cosa}, we can observe that, the weights of the pre-trained and fine-tuned models are extremely similar in AlexNet . This can be caused by the size of receptive filed. Since ReLUs are used in both architectures, vanishing gradients do not exist. Rectified activation function is mathematically given by:
      \begin{equation}\label{relu}
        h = \max ({w^T}x,0) = \left\{ {\begin{array}{*{20}{c}}
{{w^T}x}&{{w^T}x > 0}\\
0&{else}
\end{array}} \right.
      \end{equation}

    The ReLU is inactivated when its input is below 0 and its partial derivative is 0 as well. Sparsity can improve the performance of the linear classifier on top, but on the other hand, sparse representations make the network more difficult to train as well as fine-tune. The derivative of the filter is $\frac{{\partial J}}{{\partial w}} = \frac{{\partial J}}{{\partial y}}\frac{{\partial y}}{{\partial w}} = \frac{{\partial J}}{{\partial y}}*x$ where $\frac{{\partial J}}{{\partial y}}$ denotes the partial derivative of the activation function, $y=w^Tx$ and $x$ denotes the inputs of the layer. The sparse input could lead to sparse filter derivative for back propagation which would eventually prevent the errors passing down effectively. Therefore, the filters of the fine-tuned AlexNet is extremely similar. Compared to large receptive field used in AlexNet, the inception module in GoogLeNet employs 2 additional $n\times n\_reduced$ convolutional layers before the $3\times 3$ and $5\times 5$ convolutional layers (see Figure \ref{incept}). Even though the original purpose of these two $1\times 1$ convolutional layer is for computational efficiency, these 2 convolutional layers tend to squeeze their sparse inputs and generate a dense outputs for the following layer. We can see from Table \ref{tab:sparse} that the sparsity of the $n\times n\_reduce$ layers are denser than other layers within the inception module. This makes the filters in the following layer more easily to be trained for transfer learning and generate efficient sparse representations.
  %\item The pooling strategy. In AlexNet, max pooling is applied to all the pooling layers between several convolution layers. During back propagation, the max pooling layer always passes the error to the place where it came from. Since it only came from one place of the receptive field, the back propagation error is sparse and keeps the most filters unchanged. In GoogLeNet, even though, there is a max pooling layer within every inception module, there are other 3 back propagation errors, from $5\times 5\_reduce$ and $3\times 3\_reduce$ that can parse dense back propagation errors to the previous inception module.


\begin{table*}[htbp]
  \centering
  \caption{Cosine similarity of the layers in inception modules between fine-tuned models and pre-trained model for GoogLeNet}
    \begin{tabular}{|r|cccccc|}
	\hline
    \multicolumn{7}{|c|}{food256} \\\hline

          & \multicolumn{1}{l}{1x1} & \multicolumn{1}{l}{3x3\_reduce} & \multicolumn{1}{l}{3x3} & \multicolumn{1}{l}{5x5\_reduce} & \multicolumn{1}{l}{5x5} & \multicolumn{1}{l|}{pool\_proj } \\\hline
    inception\_3a & 0.72  & 0.72  & 0.64  & 0.67  & 0.73  & 0.69 \\
    inception\_3b & 0.59  & 0.64  & 0.53  & 0.70  & 0.60  & 0.56 \\
    inception\_4a & 0.46  & 0.53  & 0.54  & 0.50  & 0.67  & 0.38 \\
    inception\_4b & 0.55  & 0.58  & 0.63  & 0.52  & 0.69  & 0.41 \\
    inception\_4c & 0.63  & 0.64  & 0.63  & 0.57  & 0.68  & 0.52 \\
    inception\_4d & 0.60  & 0.62  & 0.60  & 0.58  & 0.68  & 0.50 \\
    inception\_4e & 0.60  & 0.61  & 0.67  & 0.61  & 0.68  & 0.50 \\
    inception\_5a & 0.51  & 0.53  & 0.58  & 0.48  & 0.60  & 0.39 \\
    inception\_5b & 0.40  & 0.44  & 0.50  & 0.41  & 0.59  & 0.40 \\  \hline
    \multicolumn{7}{|c|}{food101} \\ \hline
          & \multicolumn{1}{l}{1x1 } & \multicolumn{1}{l}{3x3\_reduce} & \multicolumn{1}{l}{3x3} & \multicolumn{1}{l}{5x5\_reduce} & \multicolumn{1}{l}{5x5} & \multicolumn{1}{l|}{pool\_proj } \\\hline
    inception\_3a & 0.71  & 0.72  & 0.63  & 0.67  & 0.73  & 0.68 \\
    inception\_3b & 0.56  & 0.63  & 0.50  & 0.71  & 0.60  & 0.53 \\
    inception\_4a & 0.43  & 0.50  & 0.50  & 0.47  & 0.62  & 0.36 \\
    inception\_4b & 0.48  & 0.52  & 0.57  & 0.50  & 0.67  & 0.35 \\
    inception\_4c & 0.57  & 0.61  & 0.59  & 0.53  & 0.63  & 0.47 \\
    inception\_4d & 0.54  & 0.58  & 0.53  & 0.54  & 0.64  & 0.44 \\
    inception\_4e & 0.53  & 0.54  & 0.61  & 0.55  & 0.62  & 0.42 \\
    inception\_5a & 0.43  & 0.47  & 0.53  & 0.45  & 0.57  & 0.34 \\
    inception\_5b & 0.36  & 0.39  & 0.46  & 0.38  & 0.52  & 0.37 \\
    \hline
    \end{tabular}%
  \label{tab:cosg}%
\end{table*}%


\begin{table*}[htbp]
  \centering
  \caption{Cosine similarity of the layers between fine-tuned models and pre-trained model for AlexNet}
    \begin{tabular}{|r|ccccccc|}
    \hline
          & conv1 & conv2 & conv3 & conv4 & conv5 & fc6   & fc7 \\
	\hline
    food256 & 0.997 & 0.987 & 0.976 & 0.976 & 0.978 & 0.936 & 0.923 \\
    food101 & 0.996 & 0.984 & 0.963 & 0.960 & 0.963 & 0.925 & 0.933 \\
    \hline
    \end{tabular}%
  \label{tab:cosa}%
\end{table*}%

% Table generated by Excel2LaTeX from sheet 'google'
\begin{table*}[htbp]
  \centering
  \caption{Sparsity of the output for each unit in GoogLeNet inception module for training data from Food101 in percent}
    \begin{tabular}{|r|cccccc|}
    \hline
          & 1x1  & 3x3\_reduce & 3x3  & 5x5\_reduce & 5x5  & pool\_proj  \\
	\hline
    inception\_3a & $69.3\pm 1.3$  & $69.6 \pm 1.1$  & $80.0\pm  1.0$& $64.1\pm  2.2$& $75.8\pm  1.6$& $76.2\pm 5.4$\\
    inception\_3b & $92.8 \pm 0.9$&$ 76.5 \pm 0.9$& $94.7\pm 0.9 $&$ 71.6 \pm 2.3 $&$ 94.4\pm 0.5 $&$ 94.7 \pm 1.6$\\
    inception\_4a & $90.9 \pm 0.9$& $70.0\pm 1.2 $& $93.8\pm 1.1 $& $63.3\pm 4.0 $& $91.9\pm 1.8 $& $95.1\pm 2.0$\\
    inception\_4b & $71.9 \pm 1.6$& $67.5\pm 1.2$ & $75.4\pm  1.0$& $58.5 \pm 2.6$& $78.9\pm  1.6$& $85.6\pm 3.6$\\
    inception\_4c & $75.1 \pm 2.4$& $72.6 \pm 1.3$& $81.0\pm 2.0$ & $66.3\pm 6.1 $& $79.7 \pm 3.6$& $88.1\pm 3.3$\\
    inception\_4d & $87.3 \pm 2.7$& $78.0 \pm 2.2$& $88.0\pm 1.6$& $67.9\pm 3.1 $& $88.9\pm 2.8 $& $93.0\pm 2.2$\\
    inception\_4e & $91.8\pm  1.1$& $62.3\pm 2.2 $& $91.0\pm 2.5 $& $49.5 \pm 3.7$& $94.0 \pm 1.0$& $92.3\pm 1.5$\\
    inception\_5a & $78.7 \pm 1.6$& $66.5\pm  1.7$& $82.3\pm 2.6 $& $59.9\pm 3.2 $& $86.4\pm 2.3 $& $87.1\pm 2.6$\\
    inception\_5b & $88.2\pm 2.3 $& $86.8 \pm 1.6$&$ 83.3\pm 4.4$ & $84.0\pm 3.1 $& $81.4\pm 5.3$  & $94.7\pm 1.5$\\
    \hline
    \end{tabular}%
  \label{tab:sparse}%
\end{table*}%

The unique structure of the Inception module guarantees that the sparse outputs from previous layer can be squeezed with the $1\times 1$ convolutional layers and feed to convolutional layers with bigger receptive field to generate sparser representation. The squeeze action promises the back propagation error can be transferred more efficiently and makes the whole network more flexible to fit different recogntion tasks.

\subsection{Learning across the datasets}
From the previous experiments we can see that pre-training on the ImageNet dataset can improve the performance of the deep convolutional neural network on our specific area. In this part, we will discuss the generalization ability within the food recognition problem.  Zhou et al. trained AlexNet for Scene Recognition across two datasets with identical categories \cite{NIPS2014_Zhou}. But for more complex situation, such as two similar datasets with a little overlapped categories, we are very interested in exploring whether deep CNN can still successfully handle. Therefore, we conduct the following experiment to stimulate a more challenging real world problem: transferring the knowledge from the fine-tuned Food-101 model to a target set, Food-256 dataset. To make the experiment more practical, we limit the number of samples per category from Food-256 for training, because if we want to build a our own model using deep CNN for a specific task, the resource is always limited and it is exhausted to collect hundreds of labeled images for each category.

\begin{table*}[htbp]
	\centering
	\begin{tabular}{|c|cc|cc|}
		\hline
		& \multicolumn{2}{c|}{AlexNet} & \multicolumn{2}{c|}{GoogLeNet} \\
		\hline
		instances per class & ImageNet  & Food101\_ft    &  ImageNet  & Food101\_ft \\ \hline
		20    & 68.80  & {75.12} & 74.54 & {77.77} \\
		30    & 73.15 & {77.02} & 79.21 & {81.06} \\
		40    & 76.04 & {80.23} & 81.76 & {83.52} \\
		50    & 78.90  & {81.66} & 84.22 & {85.84} \\
		all    & 85.59 &  {87.21} & {90.66 }&   {90.65}     \\
		\hline
	\end{tabular}%
	\caption{Top5 Accuracy for transferring from Food101 to subset of Food256 in percent}
	\label{tab:cross}%
\end{table*}%

The Food-101 and Food-256 datasets share about 46 categories of food even though the images in the same category may vary across these two datasets. The types of food in Food-101 are mainly western style while most types of food in Food-256 are typical Asian foods. We compare the top-5 accuracy trained from different size of subset for Food-256 on different pre-trained model and the results are shown in Table \ref{tab:cross}.
%The ImageNet columns denote  the pre-trained model trained only on ImageNet images and the Food101\_ft columns denote the pre-trained model trained on ImageNet images and then fine-tuned on Food-101.
The ImageNet columns denote using the model pre-trained from ImageNet dataset as the pre-trained model and Food101\_ft columns denote using the fine-tuned Food-101 model (the same one in Table \ref{tab:ft}) as the pre-trained model.

From the result of Table \ref{tab:cross} we can see that, with this further transfer learning, both CNNs can achieve around 95\% of the accuracy trained on full dataset while just utilizing about half of them (50 per class, 12800 of 25361 images). This indicates that when there is not enough labeled data, with its strong generalization ability, deep CNN trained from general task can still achieve satisfying result and perform even better when an additional relevant dataset is involved. This encouraging result may attract more people to use deep CNN for their specific task and continue to explore the potential of the existing architecture as well as designing new ones.



\section{Summary}
	\chapter{Conclusion}
Large object recognition task can be effectively solved with deep CNNs, but learning from a small size of data is still challenging. Transfer learning becomes a popular way to solve the small data regime by leveraging knowledge from learned tasks. 
In this thesis, we investigate the visual transfer learning problem in two scenarios under the setting where the source data is absent. Transfer learning under this setting is common and investigating the transfer learning problem in the absence of the source data is meaningful for the practical problems. The main contributions of this thesis are as follows:
\begin{itemize}
	\item In chapter \ref{sec:pakdd}, we investigated the supervised domain adaption problem under the HTL setting. We proposed EMTLe that can leverage the knowledge from the source model. Compared to previous methods, EMTLe can better leverage the source knowledge and achieve improved performance.
	\item In chapter \ref{sec:aaai}, we proposed a framework called GDSDA for semi-supervised domain adaptation, which can leverage the knowledge from the source model in the semi-supervised learning scenario. To make GDSDA more practical, we then proposed GDSDA-SVM as an example that uses SVM as the classifier in GDSDA. Experimental results show that GDSDA can effectively leverage the source knowledge for for semi-supervised domain adaptation problem. 
	\item Finally, we investigated the problem of fine-tuning the deep CNNs for food recognition tasks. We compared the performances of two CNNs architectures and found that GoogLeNet is more suitable as the pre-trained model for transfer learning.
\end{itemize}

Visual transfer learning in the absence of the source data is challenging and important in many real transfer learning scenarios. How to better leverage the knowledge from the source data and void negative transfer at the same time is still an open question. In this thesis, we provided a few methods for this problem. In our future work, we plan to use deep neural network for visual transfer task. There are still many challenges in apply deep neural network for transfer learning. An important issue of applying deep transfer learning is to solve the problem of overfitting. Possible solutions could be eliminating the redundant ``nodes" in deep neural networks while keep those informative ``nodes" to reduce the size of the net and therefore, can better avoid overfitting problem.

	
	
	
	%% This adds a line for the Bibliography in the Table of Contents.
	\addcontentsline{toc}{chapter}{Bibliography}
	%% ***   Set the bibliography style.   ***
	\bibliographystyle{plain} % (change according to your preference)
	%%% ***   Set the bibliography file.   ***
	\bibliography{research}{}
	%% ***   NOTE   ***
	%% If you don't use bibliography files, comment out the previous line
	%% and use \begin{thebibliography}...\end{thebibliography}.  (In that
	%% case, you should probably put the bibliography in a separate file
	%% and \include or \input it here).
	
	%Appendices.
	\begin{appendices}
		\chapter{Proofs of Theorems}
\section{Proof of Theorem1}
\begin{proof}
For simplification, let $\delta_i=1$ if $i=N+1$ and 0 otherwise, and  ${\theta _{ij}} = {\alpha ''_{ij}}\left( {1 - {\delta _j}} \right)/\psi_{ii}^{ - 1}$. Eq. \eqref{eq:train_loss} can be written as:
\begin{equation}\label{eq:loss_simple}
%\begin{split}
{\xi _i}(\gamma ,\beta )=\mathop {\max }\limits_{n} \bigg \{ {\varepsilon _{n{y_i}}} - 1 + \frac{{\left( {{{\alpha '}_{i{y_i}}} - {{\alpha '}_{in}}} \right)}}{{\psi _{ii}^{ - 1}}} + {\theta _{in}}{\gamma _n} 
- {\theta _{i{y_i}}}{\gamma _{{y_i}}} + \left( {{\delta _n} - {\delta _{{y_i}}}} \right)\sum\limits_k {\frac{{{{\alpha ''}_{ik}}{\beta _k}}}{{\psi_{ii}^{ - 1}}}}  \bigg\}
%\end{split}
\end{equation}
When $\mathbf{\gamma}=\mathbf{\beta} = \mathbf{0}$, from Eq. \eqref{eq:loss_simple} we can get:
\begin{equation*}
{\bar \xi _i} = \mathop {\max }\limits_n \left[ { {\varepsilon _{n{y_i}}}-1 + \frac{{\left( {{{\alpha '}_{i{y_i}}} - {{\alpha '}_{in}}} \right)}}{{\psi _{ii}^{ - 1}}}} \right]
\end{equation*}
To obtain the optimal value of $\gamma$ and $\beta$, we have to seek the saddle point of the Lagrangian problem in \eqref{eq:dual} by finding the minimum for the prime variables $\left\{ \gamma, \beta, \xi \right\}$ and the maximum for the dual variables $\eta $. To find the minimum of the primal problem, we require:
\begin{equation*}
\frac{{\partial L}}{{\partial {\xi _i}}} = 1 - \sum\limits_n {{\eta _{in}}}  = 0 \to \sum\limits_n {{\eta _{in}}}  = 1
\end{equation*}
Similarly, for $\gamma$ and $\beta$, we require:
\begin{eqnarray}\label{eq:opt_gama}
\frac{{\partial L}}{{\partial {\gamma _n}}} &=& {\lambda _1}{\gamma _n} + \sum\limits_i {{\eta _{in}}{\theta _{in}}}  - \sum\limits_{i,n = {y_i}} {\left( {\sum\limits_q {{\eta _{iq}}} } \right){\theta _{in}}{\gamma _n}}  \nonumber\\
&=_1 &{\lambda _1}{\gamma _n} + \sum\limits_i {{\eta _{in}}{\theta _{in}}}  - \sum\limits_i {{\varepsilon _{n{y_i}}}{\theta _{in}}}  = 0  \nonumber\\
&\Rightarrow & \gamma _n^* = \frac{1}{{{\lambda _1}}}\sum\limits_i {\left( {{\varepsilon _{n{y_i}}} - {\eta _{in}}} \right){\theta _{in}}}
\end{eqnarray}
In $=_1$ we use the facts that $\sum_n\eta_{in}=1$ and use $\varepsilon_{ny_i}$ to replace it.
\begin{eqnarray}\label{eq:opt_beta}
\frac{{\partial L}}{{\partial {\beta _n}}} &=& {\lambda _2}{\beta _n} + \left[ {\sum\limits_{i,n} {\frac{{{\eta _{in}}{{\alpha ''}_{in}}}}{{\psi_{ii}^{ - 1}}}\left( {{\delta _n} - {\delta _{{y_i}}}} \right)} } \right] = 0 \nonumber \\
&\Rightarrow &\beta _n^* = \frac{1}{{{\lambda _2}}}\sum\limits_{i,n} {\frac{{{\eta _{in}}{{\alpha ''}_{in}}}}{{\psi _{ii}^{ - 1}}}\left( {{\delta _{{y_i}}} - {\delta _n}} \right)}
\end{eqnarray}
As the strong duality holds,the primal and dual objectives coincide. Plug Eq \eqref{eq:opt_gama} and \eqref{eq:opt_beta} into Eq. \eqref{eq:dual}, we have:
\begin{equation*}
\sum\limits_{i,n} {{\eta _{in}}\left[ {1 - {\varepsilon _{n{y_i}}} + {{\hat Y}_{in}}\left( {\gamma^* ,\beta^* } \right) - {{\hat Y}_{i{y_i}}}\left( {\gamma^* ,\beta^* } \right) - {\xi _i^*}} \right]}=0
\end{equation*}
Expand the equation above, we have:
\begin{eqnarray}\nonumber
\sum\limits_{i,n} {{\eta _{in}}\left[ { {\varepsilon _{n,{y_i}}}-1 + \frac{{\left( {{{\alpha '}_{i{y_i}}} - {{\alpha '}_{in}}} \right)}}{{\psi_{ii}^{ - 1}}} - {\xi _i}} \right]} \nonumber
= {\lambda _1}\sum\limits_r {{{\left\| {\gamma _r^*} \right\|}^2}}  + {\lambda _2}\sum\limits_r {{{\left\| {\beta _r^*} \right\|}^2}}  \ge 0\nonumber
\end{eqnarray}
Rearranging the above, we obtain:
\begin{eqnarray}\label{eq:link1}
\sum\limits_{i,n} {{\eta _{in}}\left[ { {\varepsilon _{n,{y_i}}} -1+ \frac{{\left( {{{\alpha '}_{i{y_i}}} - {{\alpha '}_{in}}} \right)}}{{\psi_{ii}^{ - 1}}}} \right]} 
 \ge \sum\limits_{i,n} {{\eta _{in}}{\xi _i}}  = \sum\limits_i {{\xi _i}}
\end{eqnarray}
The left-hand side of Inequation \eqref{eq:link1} can be bounded by:
\begin{eqnarray}
&&\sum\limits_{i,n} {{\eta _{in}}\left[ { {\varepsilon _{n{y_i}}}-1 + \frac{{\left( {{{\alpha '}_{i{y_i}}} - {{\alpha '}_{in}}} \right)}}{{\psi_{ii}^{ - 1}}}} \right]} \nonumber\\ &&\le \sum\limits_i {\left( {\sum\limits_n {{\eta _{in}}\mathop {\max }\limits_r \left\{ { {\varepsilon _{r{y_i}}} -1 + \frac{{\left( {{{\alpha '}_{i{y_i}}} - {{\alpha '}_{ir}}} \right)}}{{\psi_{ii}^{ - 1}}}} \right\}} } \right)}  \nonumber\\
&&= \sum\limits_i {\left( {\sum\limits_n {{\eta _{in}}{{\bar \xi }_i}} } \right)}  = \sum\limits_i {\bar \xi_i }
\end{eqnarray}
\end{proof} 
\section{Proof of Theorem2}\label{app:cross}
\begin{theorem}[Extension of \cite{cawley2006leave}]
\textit{Given a dataset $D=\{(x_i,y_i)|i=1,...,l\}$, the solution of a LS-SVM on $D$ can be written as:}
	
	\begin{equation}\label{eq:app:orgmatrix}
	\left[ {\begin{array}{*{20}{c}}
		{K  + \frac{1}{C}{\rm I}}\\
		1^T
		\end{array}\begin{array}{*{20}{c}}
		1\\
		0
		\end{array}} \right]\left[ {\begin{array}{*{20}{c}}
		\alpha \\
		b
		\end{array}} \right] = \left[ \begin{array}{l}
	y\\
	0
	\end{array} \right]
	\end{equation}
	\textit{Assume that $D^{(n)} = \{(x_i,y_i)|i=1,...,n\}$ is a subset of $D$ and $D\backslash D^{(n)}$ is the complement of $D^{(n)}$ in $D$ , The unbiased leave out error of a LS-SVM trained from $D\backslash D^{(n)}$ on $D^{(n)}$ can be estimated as:}
	
	\begin{equation*}%\label{eq:nout}
	ERR_{leave-out} = \left( {{S_n} - sS_{(l - n + 1)}^{ - 1}{s^T}} \right){\left[ {{\alpha _1},...,{\alpha _n}} \right]^T}
	\end{equation*}
	\textit{Where $\left[\alpha_1,...,\alpha_n\right]$ is the first $n$ rows of $\alpha$ in \eqref{eq:app:orgmatrix}. $S_n$, $s$ and $S_{(l - n + 1)}$ are the square blocks of matrix:}
	
	\begin{equation*}
	\left[ {\begin{array}{c|c}
		{{S_{n }}} &s\\ \hline
		{{s^T}}&{{S_{(l - n + 1)}}}
		\end{array}} \right] =\left[ {\begin{array}{*{20}{c}}
		{K  + \frac{1}{C}{\rm I}}\\
		1^T
		\end{array}\begin{array}{*{20}{c}}
		1\\
		0
		\end{array}} \right]
	\end{equation*}
\end{theorem}
\begin{proof}
	Following the result of Eq. \eqref{eq:app:orgmatrix} and noticing that the matrix of the left hand in Eq. \eqref{eq:app:orgmatrix} is symmetrical, it can be written as follow:
	
	\begin{equation}\label{eq:app:block}
	\left[ {\begin{array}{*{20}{c}}
		{K  + \frac{1}{C}{\rm I}}\\
		1^T
		\end{array}\begin{array}{*{20}{c}}
		1\\
		0
		\end{array}} \right] = \left[ {\begin{array}{c|c}
		{{S_{n }}} &s\\ \hline
		{{s^T}}&{{S_{(l - n + 1)}}}
		\end{array}} \right] 
	\end{equation}
	Where $S_{n} \in R^{n \times n}$, $s \in R^{n\times (l-n+1)}$ and $S_{(l - n + 1)} \in R^{(l - n + 1) \times (l - n + 1) }$.
	
	For the first round of the cross validation situation, assume the first $n$ examples are used as the validation set. In this case, let $\left[ {{\alpha ^{ - 1}},{b^{ - 1}}} \right] \in R^{l-n+1}$ denote the optimal parameters for a LS-SVM $f^{-1}$ trained on the rest of the samples and they can be found by:
	
	\begin{equation}\label{eq:app:ab,l-n+1}
	\left[ \begin{array}{l}
	{\alpha ^{ - 1}}\\
	{b^{ - 1}}
	\end{array} \right]{\rm{ = S}}_{(l - n + 1)}^{ - 1}{\left[ {{y_{n + 1}},...,{y_l},0} \right]^T}
	\end{equation}
	The prediction of $f^{-1}$ on the validation set $\hat{Y} = \left[\hat{y}_1,...,\hat{y}_n\right]$ is given by:
	
	\begin{equation}\label{eq:app:predict}
	\left[ {\hat {{y}}_1,...,\hat {{y}}}_n \right]^T = s\left[ {\begin{array}{*{20}{c}}
		\alpha^{-1} \\
		b^{-1}
		\end{array}} \right] = sS_{(l - n + 1)}^{ - 1}{\left[ {{y_{n + 1}},...,{y_l},0} \right]^T}
	\end{equation}
	Moreover, the last $l-n+1$ rows in Eq. \eqref{eq:app:orgmatrix} can be represented as $\left[ {\begin{array}{*{20}{c}}{{s^T}}&{{S_{l - n + 1}}}\end{array}} \right] \left[ {\alpha ,b}\right] ^T= \left[ {{y_{n + 1}},...,{y_l},0} \right]^T$. So
	
	\begin{equation}\label{eq:app:yestimate}
	\begin{array}{l}
	\hat{Y}={\left[ {\hat {{y}}_1,...,\hat {{y}}}_n \right]^T} = sS_{(l - n + 1)}^{ - 1}\left[ {\begin{array}{*{20}{c}}
		{{s^T}}&{{S_{l - n + 1}}}
		\end{array}} \right]\left[ {\begin{array}{*{20}{c}}
		\alpha \\
		b
		\end{array}} \right]\\
	= sS_{(l - n + 1)}^{ - 1}{s^T}{\left[ {{\alpha _1},...,{\alpha _n}} \right]^T} + s{\left[ {{\alpha _{n + 1}},...,{\alpha _l},b} \right]^T}
	\end{array}
	\end{equation}
	Then first $n$ rows in Eq. \eqref{eq:app:orgmatrix} can be represented as:
	
	\begin{equation}\label{eq:app:ytrue}
	Y={\left[ {{y_1},...{y_n}} \right]^T} = \left[ {\begin{array}{*{20}{c}}
		{{S_n}}&s
		\end{array}} \right]\left[ {\begin{array}{*{20}{c}}
		\alpha \\
		b
		\end{array}} \right] = {S_n}{\left[ {{\alpha _1},...,{\alpha _n}} \right]^T} + s{\left[ {{\alpha _{n + 1}},...,{\alpha _l},b} \right]^T}
	\end{equation}
	Thus, combining \eqref{eq:app:yestimate} and \eqref{eq:app:ytrue}, we have the following equation:
	
	\begin{equation}
	\begin{array}{l}
	\hat Y = Y - {S_n}{\left[ {{\alpha _1},...,{\alpha _n}} \right]^T} + sS_{(l - n + 1)}^{ - 1}{s^T}{\left[ {{\alpha _1},...,{\alpha _n}} \right]^T}\\
	= Y - \left( {{S_n} - sS_{(l - n + 1)}^{ - 1}{s^T}} \right){\left[ {{\alpha _1},...,{\alpha _n}} \right]^T}
	\end{array}
	\end{equation}
	According to block matrix inversion lemma
	
	\begin{equation}%\label{}
	{\left[ {\begin{array}{*{20}{c}}
			{{S_n}}\\
			{{s^T}}
			\end{array}\begin{array}{*{20}{c}}
			s\\
			{{S_{(l - n + 1)}}}
			\end{array}} \right]^{ - 1}} = \left[ {\begin{array}{*{20}{c}}
		{{\kappa ^{ - 1}}}&{ - {\kappa ^{ - 1}}sS_{(l - n + 1)}^{ - 1}}\\
		{ - {S_{(l - n + 1)}}{s^T}{\kappa ^{ - 1}}}&{S_{(l - n + 1)}^{ - 1} + S_{(l - n + 1)}^{ - 1}{s^T}{\kappa ^{ - 1}}sS_{(l - n + 1)}^{ - 1}}
		\end{array}} \right]
	\end{equation}
	Where $\kappa  = \left( {{S_n} - sS_{(l - n + 1)}^{ - 1}{s^T}} \right)$. We have:
	
	\begin{equation}%\label{eq:nout}
	Y - \hat Y = {\kappa}{\left[ {{\alpha _1},...,{\alpha _n}} \right]^T}
	\end{equation}
\end{proof}
%\chapter{Tables}
%\section{Configuration of GoogLeNet}
%\begin{landscape}
%% Table generated by Excel2LaTeX from sheet 'new  2'
\begin{table}
  \centering
  \caption{Configuration of GoogLeNet}
    \begin{tabular}{|c|c|c|c|c|c|c|c|c|c|}
    \hline
    type  & patch size / stride & output size & depth & \#1x1 & \#3x3 reduce & \#3x3 & \#5x5 reduce & \#5x5 pool & proj \\\hline
    convolution & 7x7/2 & 112x112x64 & 1     &       &       &       &       &       &  \\\hline
    max pool & 3x3/2 & 56x56x64 & 0     &       &       &       &       &       &  \\\hline
    convolution & 3x3/1 & 56x56x192 & 2     &       & 64    & 192   &       &       &  \\\hline
    max pool & 3x3/2 & 28x28x192 & 0     &       &       &       &       &       &  \\\hline
    inception(3a) &       & 28x28x256 & 2     & 64    & 96    & 128   & 16    & 32    & 32 \\\hline
    inception(3b) &       & 28x28x480 & 2     & 128   & 128   & 192   & 32    & 96    & 64 \\\hline
    maxpool & 3x3/2 & 14x14x480 & 0     &       &       &       &       &       &  \\\hline
    inception(4a) &       & 14x14x512 & 2     & 192   & 96    & 208   & 16    & 48    & 64 \\\hline
    inception(4b) &       & 14x14x512 & 2     & 160   & 112   & 224   & 24    & 64    & 64 \\\hline
    inception(4c) &       & 14x14x512 & 2     & 128   & 128   & 256   & 24    & 64    & 64 \\\hline
    inception(4d) &       & 14x14x528 & 2     & 112   & 144   & 288   & 32    & 64    & 64 \\\hline
    inception(4e) &       & 14x14x832 & 2     & 256   & 160   & 320   & 32    & 128   & 128 \\\hline
    maxpool & 3x3/2 & 7x7x832 & 0     &       &       &       &       &       &  \\\hline
    inception(5a) &       & 7x7x832 & 2     & 256   & 160   & 320   & 32    & 128   & 128 \\\hline
    inception(5b) &       & 7x7x1024 & 2     & 384   & 192   & 384   & 48    & 128   & 128 \\\hline
    avg pool & 7x7/1 & 1x1x1024 & 0     &       &       &       &       &       &  \\\hline
    dropout (40\%) &       & 1x1x1024 & 0     &       &       &       &       &       &  \\\hline
    linear &       & 1x1x1000 & 1     &       &       &       &       &       &  \\\hline
    softmax &       & 1x1x1000 & 0     &       &       &       &       &       &  \\\hline
    \end{tabular}%
  \label{tab:cnn:googlenet}%
\end{table}%

%\end{landscape}
%
\chapter{Detailed Experiment Results}
\section{Detailed result on MNIST}
%\section{Detailed result on MNIST}
\subsection{No Noise}
\begin{figure}[h]
	\centering
	\begin{tabular}{cc}
		\includegraphics[width=0.45\textwidth]{appendix/tables/MNIST_Rate_0_class_1.jpg} & 
		\includegraphics[width=0.45\textwidth]{appendix/tables/MNIST_Rate_0_class_2.jpg} \\
		class 1 & class 2\\
		\includegraphics[width=0.45\textwidth]{appendix/tables/MNIST_Rate_0_class_3.jpg} & 
		\includegraphics[width=0.45\textwidth]{appendix/tables/MNIST_Rate_0_class_4.jpg} \\
		class 3 & class 4\\
	\end{tabular}
\end{figure}
\begin{figure}[h]
	\centering
	\begin{tabular}{cc}
		\includegraphics[width=0.45\textwidth]{appendix/tables/MNIST_Rate_0_class_5.jpg} & 
		\includegraphics[width=0.45\textwidth]{appendix/tables/MNIST_Rate_0_class_6.jpg} \\
		class 5 & class 6\\
		\includegraphics[width=0.45\textwidth]{appendix/tables/MNIST_Rate_0_class_7.jpg} & 
		\includegraphics[width=0.45\textwidth]{appendix/tables/MNIST_Rate_0_class_8.jpg} \\
		class 7 & class 8\\
		\includegraphics[width=0.45\textwidth]{appendix/tables/MNIST_Rate_0_class_9.jpg} & 
		\includegraphics[width=0.45\textwidth]{appendix/tables/MNIST_Rate_0_class_10.jpg} \\
		class 9 & class 10\\
	\end{tabular}
\end{figure}
\clearpage



\subsection{0.3 Noise}
\begin{figure}[h]
	\centering
	\begin{tabular}{cc}
		\includegraphics[width=0.45\textwidth]{appendix/tables/MNIST_Rate_3_class_1.jpg} & 
		\includegraphics[width=0.45\textwidth]{appendix/tables/MNIST_Rate_3_class_2.jpg} \\
		class 1 & class 2\\
		\includegraphics[width=0.45\textwidth]{appendix/tables/MNIST_Rate_3_class_3.jpg} & 
		\includegraphics[width=0.45\textwidth]{appendix/tables/MNIST_Rate_3_class_4.jpg} \\
		class 3 & class 4\\
	\end{tabular}
\end{figure}
\begin{figure}[h]
	\centering
	\begin{tabular}{cc}
		\includegraphics[width=0.45\textwidth]{appendix/tables/MNIST_Rate_3_class_5.jpg} & 
		\includegraphics[width=0.45\textwidth]{appendix/tables/MNIST_Rate_3_class_6.jpg} \\
		class 5 & class 6\\
		\includegraphics[width=0.45\textwidth]{appendix/tables/MNIST_Rate_3_class_7.jpg} & 
		\includegraphics[width=0.45\textwidth]{appendix/tables/MNIST_Rate_3_class_8.jpg} \\
		class 7 & class 8\\
		\includegraphics[width=0.45\textwidth]{appendix/tables/MNIST_Rate_3_class_9.jpg} & 
		\includegraphics[width=0.45\textwidth]{appendix/tables/MNIST_Rate_3_class_10.jpg} \\
		class 9 & class 10\\
	\end{tabular}
\end{figure}
\clearpage



\subsection{0.5 Noise}
\begin{figure}[h]
	\centering
	\begin{tabular}{cc}
		\includegraphics[width=0.45\textwidth]{appendix/tables/MNIST_Rate_5_class_1.jpg} & 
		\includegraphics[width=0.45\textwidth]{appendix/tables/MNIST_Rate_5_class_2.jpg} \\
		class 1 & class 2\\
		\includegraphics[width=0.45\textwidth]{appendix/tables/MNIST_Rate_5_class_3.jpg} & 
		\includegraphics[width=0.45\textwidth]{appendix/tables/MNIST_Rate_5_class_4.jpg} \\
		class 3 & class 4\\
	\end{tabular}
\end{figure}
\begin{figure}[h]
	\centering
	\begin{tabular}{cc}
		\includegraphics[width=0.45\textwidth]{appendix/tables/MNIST_Rate_5_class_5.jpg} & 
		\includegraphics[width=0.45\textwidth]{appendix/tables/MNIST_Rate_5_class_6.jpg} \\
		class 5 & class 6\\
		\includegraphics[width=0.45\textwidth]{appendix/tables/MNIST_Rate_5_class_7.jpg} & 
		\includegraphics[width=0.45\textwidth]{appendix/tables/MNIST_Rate_5_class_8.jpg} \\
		class 7 & class 8\\
		\includegraphics[width=0.45\textwidth]{appendix/tables/MNIST_Rate_5_class_9.jpg} & 
		\includegraphics[width=0.45\textwidth]{appendix/tables/MNIST_Rate_5_class_10.jpg} \\
		class 9 & class 10\\
	\end{tabular}
\end{figure}
	\end{appendices}
	
	%CV only relevant stuff... not full CV.
	\addcontentsline{toc}{chapter}{Curriculum Vitae}
	\chapter*{Curriculum Vitae}
	\begin{table}[ht]
		\begin{tabular}{ll}
			\textbf{Name:} & \firstname{} \lastname\\\\
			\textbf{Post-Secondary} &  University of Western Ontario\\
			\textbf{Education and}& London, ON\\
			\textbf{Degrees:}& 2012 - 2017 Ph.D.\\\\
			& China University of Geosciences (Wuhan)\\
			& Wuhan, Hubei, China\\
			& 2009 - 2012 M.Sc\\\\
			& China University of Geosciences (Wuhan)\\
			& Wuhan, Hubei, China\\
			& 2005 - 2009 B.Sc\\\\
			%\textbf{Honours and}& \\
			%\textbf{Awards:}& 2006-2007\\\\
			\textbf{Related Work}& Teaching Assistant\\
			\textbf{Experience:}& The University of Western Ontario\\
			& 2012 - 2017\\\\
			
			& Research Assistant\\
			& The University of Western Ontario\\
			& 2012 - 2017\\\\
		\end{tabular}
	\end{table}
	\subsubsection*{Publications:}
	\textbf{Ao, Shuang}, and Li, Xiang and Ling, Charles. "Fast Generalized Distillation for Semi-supervised Domain Adaptation." Thirtieth-First AAAI Conference on Artificial Intelligence, 2017.\\\\
	\textbf{Ao, Shuang}, and Li, Xiang and Ling, Charles. "Effective Multiclass Transfer For Hypothesis Transfer Learning." Pacific-Asia Conference on Knowledge Discovery and Data Mining, 2017. (Accepted)\\\\
	\textbf{Ao, Shuang}, and Ling, Charles. "Adapting new categories for food recognition with deep representation." 2015 IEEE International Conference on Data Mining Workshop (ICDMW).\\\\
	Luo, Yan and Ling, Charles and \textbf{Ao, Shuang}. Mobile-based Food Classification For Type-2 Diabetes Using Nutrient and Textual Features. 2014 IEEE International Conference on Data Science and Advanced Analytics.
	
\end{document}

