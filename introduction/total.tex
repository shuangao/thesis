\chapter{Introduction}\label{sec:intro}
With the explosive image resources people uploaded every day, image recognition becomes a very hot topic and has draw many attentions in recent years. Every year, there are many inspiring results in the ImageNet Large Scale Visual Recognition Challenge (ILSVRC). 
With development of recognition technology, many IT companies want to use image recognition techniques to serve their customers and many interesting applications have been developed, such as HowOld from Microsoft and Im2Calories from Google.

In order to successfully capture the diversity of different objects around us, many recognition models contain thousands or sometimes even millions of parameters and require large amount of training images to tune these parameters as well.
As the visual recognition system become increasingly successful in many general recognition tasks, people wish that it can solve the recognition problems in many new and complicated areas which are paid less attention before.
Unfortunately, for some real applications, it is often difficult and cost to collect large set of training images. Moreover, most algorithms require that the training examples should be aligned with a prototype, which is commonly done by hand. In many real applications, collecting and fully annotating these images can be extremely expensive and could have a significant impact on the over cost of the whole system. As a result, the challenge can come from these two aspects
\begin{enumerate}
	\item Abundant new categories/scenario. In some tasks, it is required for the system to learn new categories or scenarios. The recognition algorithm should be more dynamic to adapt the new categories and scenarios.
	\item Constraint on the training images. As these tasks are lack of attention before, the training data is rare. The annotation process of these data could be both expensive and time-consuming. 
\end{enumerate} 

On the other hand, recognition is one of the most important part of our human visual system. We can recognize various kinds of materials (apple, orange, grape), objects (vehicles, buildings) and natural scenes (forests, mountain). At the age of six, human can recognize about $10^4$ object categories\cite{biederman1987recognition}. 
Our human can learn and recognize a new object with just a glance, which means we can capture the diversity of forms and appearances of a objects with just a handful examples. This remarkable ability is obtained by effectively leveraging the learned knowledge and applying it to the new tasks. It could be ideal if there is a visual recognition model can have such ability, which is referred to as \textbf{Visual Transfer Learning} (VTL). 

VTL has been increasing popular with the success of modern image recognition algorithms. In VTL, the source domain is referred to the one we have already learned and the target domain is the one we actually want to learn. In VTL, measuring the relatedness of the source and target tasks is important for the transfer process.
In previous studies of VTL, people assume that the source data are always available and can be freely accessed. Therefore, the relatedness of the source and target domain can be effectively measured by comparing the data in two domains. However, this assumption is rarely hold in real applications. Source data are often a subject of legal, technical and contractual constraints between data owners and data customers. Beyond privacy and disclosure obligations, customers are often reluctant to share their data. When operating customer care, collected data may include information on recent technical problems which is a highly sensitive topic that companies are not willing to share.
On the other hand, sharing the model trained from the source data, i.e. the source model, instead of the data can avoid these obligations and is more common in real applications. VTL under this situation is usually called \textbf{Hypothesis Transfer Learning} (HTL) \cite{kuzborskij2013stability} and the source model is called hypothesis.

In this thesis, we investigate the VTL problem under the HTL setting. 3 methods are proposed under the HTL setting for different learning scenarios.

\section{Overview for Image Recognition}\label{sec:intro:over}
In this section, we review the major procedures for image recognition. 
A general image recognition method consists of three parts: image preprocess, feature extraction and classification.

\begin{figure}
	\centering
	\includegraphics[scale=.8]{introduction/fig/IRflow.png}
	\caption{Major procedure for image recognition.}\label{fig:intro:irflow}
\end{figure}
\subsection{Preprocess}
The digital camera can capture the optical property of an object through its optical sensor and generate raw digital data.
After receiving the raw data of a image from the sensor, preprocess is to generate a new image from the source image. This new image is similar to the source image, but differs from it considering certain apsects, e.g. the new image has smoother edge, better contrast and less noise. 
Here, some \textit{pixel operations} and \textit{local operations} are used to improve the contrast and remove the noise.  

Another important operation of preprecess is segmentation according to the object, i.e. finding the region of interest. Images used for recognition should be aligned, making the target object appear in the central of the image and remove those irrelevant area.

The result of preprocess has great impact on the final result of the recognition. Clear and noise free images can make the feature extraction more effective and significantly improve the final classification accuracy.

\subsection{Feature Extraction}
 Feature extraction is a type of dimensionality reduction that efficiently represents interesting parts of an image as a compact feature vector. The feature vector is then used for either training the classifier or recognition. Therefore, feature extraction is the most important part for image recognition. The quality of the features extracted from a image have great impact on the recognition result. There are two major streams for feature extraction: the hand engineered method and representation learning method.
\subsubsection{Hand Engineered Feature}
Hand engineered features are typically low level and local features.
Low level features are extracted according to some optical properties of an image. These features are low level / local features. There is a widely agreement that local features are an efficient tool for object representation due to their robustness with respect to occlusion and geometrical transformations \cite{van2006coloring}. Common low level hand engineered features include Histogram of Oriented Gradients (HOG) \cite{dalal2005histograms}, Scale Invariant Feature Transform (SIFT) \cite{lowe1999object}, Speeded Up Robust Features (SURF) \cite{bay2006surf}, Local Binary Patterns (LBP) \cite{ojala2002multiresolution}, and color histograms \cite{birchfield1998elliptical}. Feature descriptors obtain from these low level features refer to a pattern or distinct structure found in an image, such as a point, edge, or small image patch. They are usually associated with an image patch that differs from its immediate surroundings by texture, color, or intensity. What the feature actually represents does not matter, just that it is distinct from its surroundings.
\begin{figure}[h]
	\centering
	\includegraphics[scale=.6]{introduction/fig/sift.jpg}
	\caption{Feature extraction using SIFT.}\label{fig:intro:sift}
\end{figure}
These low level features can be used directly for recognition. However, since they just represent certain local properties of an image and are not discriminative enough for recognition, discriminative high level features can be further learned by combining the low level features. 

\subsubsection{Representation Learning}
Representation learning is mainly described by Deep Learning 
algorithms\cite{krizhevsky2012imagenet} or Auto Encoders\cite{bengio2007scaling}. The ideas is to learn a group of filters that are able to capture various kinds of features to discern one category of images from the another category with some supervised or unsupervised algorithm. Typically in representation learning, features are learned hierarchically from low-level features to high level ones. 
Learning representation from an image can start from either low level features (Auto Encoders) or raw pixels of an image (Deep Learning). It is generally considered that learning the good feature representations is the most important part in image recognition.
\begin{figure}
	\centering
	\includegraphics[scale=.3]{introduction/fig/sparsecoding.png}
	\caption{General Scheme of Auto Encoders. L1 is the input layer, possibly raw-pixel intensities. L2 is the compressed learned latent representation and L3 is the reconstruction of the given L1 layer from L2 layer. AutoEncoders tries to minimize the difference between L1 and L3 layers with some sparsity constraint.}\label{fig:intro:sparse}
\end{figure}

\textbf{Auto Encoders} are widely used to combine different types of low level feature. The outputs of the Auto Encoders are some latent representations. These latent representations are learned from the given images that have lowest possible reconstruction error. Even though the high level representations from Auto Encoders are learned by minimizing the reconstruction errors, they are still not robust enough to handle all kinds of variance of the objects.

\textbf{Deep Learning} is the most popular approach for learning representations. It has been widely used for all kinds of image recognition tasks and achieved the state-of-the-art performance on some large scale image recognition tasks, such as ILSVRC and The PASCAL Visual Object Classes Challenge (PASCAL VOC). 
Convolutional Neural Networks (CNN) is the most popular deep learning model for the image recognition tasks\footnote{Deep CNNs are sometimes considered as the end-to-end classifier while learning the feature representation and discriminative classifier simultaneously. However, the feature representation learned from deep CNNs can still achieve good results with other classifiers and here we consider it as a feature extractor rather than a classifier.}. The first deep CNN that had great success on image recognition is the LeNet proposed by Y.LeCun in 1989 \cite{lecun1989backpropagation}. Backpropagation was applied to Convolutional Neural networks with adaptive connections. This combination, incorporating with Max-Pooling and speeding up on graphics cards has become an important part for  many modern, competition-winning, feedforward, visual Deep Learners. Deep CNNs have been widely used as the feature extractor for all kinds of images recognition tasks and proven to be the most powerful method for feature extraction.
\begin{figure}
	\centering
	\includegraphics[scale=.3]{introduction/fig/alexnet.png}
	\caption{The architecture of ALEXNET (adapted from \cite{krizhevsky2012imagenet}).}\label{fig:intro:alex}
\end{figure}

\subsection{Classification}
After extracting feature representation from the images, a classifier should be used to train a recognition model as well as for predicting the new coming images. A supervised model is always used for training the recognition model. Discriminative classifier such as Support Vector Machine (SVM) is widely used as the classifier for recognition \cite{cristianini2000introduction}. As we mentioned before, in order to capture different variances of the images for one category, the size of the feature representation for a image is usually very large. In order to avoid overfitting, the size of the training set should be at least the same size of the size of the feature representation as well. Some classifiers such as Bayesian method or decision tree require to consider the correlations between each feature and the class labels and suffer from the large feature dimension. However, Discriminative Models\cite{bottou2010large} are more convenient for training and can be effectively optimized with stochastic gradient descent which is suitable for very large training set. 

However, obtaining a good recognition model requires abundant data when we learning a model from scratch. With the limited training data, it is difficult to achieve a good classification performance. Transfer learning is an effective way to solve this problem by utilizing the knowledge from previous tasks. In this thesis, we focus on how to transfer the knowledge from the source domain for recognition tasks. The methods proposed in this thesis are mainly focus on the stage of classification. %More specifically, we investigate the problem of how to build classifier to leverage the source knowledge when we can only visit the source model. 




\section{Approaches in Visual Transfer Learning and the Limitations}
\subsection{Importance of learning self-defined categories}
Nowadays, more and more companies provide individual service for their clients. Personalized recommendation system has been widely used in many electronic commerce, such as Amazon and eBay. The requirement of personalized service is growing every year. Personalized system means it is possible to handle the variance of the request from individuals. For image recognition, many interesting applications have been proposed. However, unlike other personalized system, it is difficult to train personalized recognition system for individuals. There could be enormous variances of one object and it is always difficult for a model to capture these variances.

Even though, the state-of-the-art recognition system can do as well as a human, its great success is achieved based on the fact that millions of labeled training images are used for training. Selecting large amount of images for the new categories is always a tedious job.
Moreover, an important property of self-defined category images is that the source of the data is inherently scarce. Therefore, it is not possible to obtain abundant training data. For example, users of Google's im2calories can track their nutrition of every meal by taking a pictures of their foods via image recognition techniques. The system firstly check the category of each food item and find their nutritions in the database. However, the existing model can only recognize the general food categories which means it is not possible for a user to track the nutrition of his/her daily home-made meals. These home-made foods can be similar to some existing food categories (home-made veggie burger can be similar to burger in McDonald), but they may have different nutritions. This scenario can be applied to learning any exclusive category in our life. Therefore, a model that can learn these self-defined categories can be important.

\begin{figure}
	\centering
	\includegraphics[scale=.6]{introduction/fig/scenario.jpg}
	\caption{Different nutrition facts between the burgers in McDonald and home-made.}\label{fig:intro:scenario}
\end{figure}

Our human is good at learning the new objects. For our human, all the information acquired is stored in our memory. These information are organized according to the properties. When we see a new concept, we don't treat it isolated, but connect it to certain previous knowledge we stored in our memory. By comparing a new concept with the organized information in our memory, we can capture the property of a new concept effectively. When referring to visual tasks, several examples can be given to show this cognitive ability. For instance, when we describe the animal "zebra", we would probably say that: zebra is a horse with while and black striped coats. People who never see a zebra could instantly have a rough idea of a zebra. 

This means to learn a new category effectively, we should be able to make use of the gained knowledge instead of learn it from scratch. This process is commonly refer to as transfer learning. Traditional machine learning methods work under the common assumption: training data and testing data are drawn from the same feature space and same distribution. When considering adding classifying the data from the new categories, the distribution of the data changes and a new model should be rebuilt from newly collected data.Those data from the original distribution is called source data and data from the new distribution is called target data. Transfer learning is used to utilize the source knowledge from the source data to help training the new model to classify the target data. 

%\hl{We can keep learning new categories throughout our whole life and become more proficient as we learn more. Human doesn't need large amount of training data to adapt a new object. In most scessnario, we can still recognize the new object well by a few examples, e.g. taking a glancing at the object. This indicates that it is possible for a recognition model to adapt by just a few training examples from the new categories.}
In this thesis, we try to use transfer learning approaches to solve the problem of learning self-defined categories. We can find many examples in knowledge engineering where transfer learning does benefit the learning process \cite{pan2010survey}. This implies that, by leveraging the learned knowledge properly, we can learn the self-defined categories with a few examples. 
In this thesis, we also assume that the self-defined categories are not isolated and similar to certain existed categories. Therefore, we can use leverage the knowledge from those existed categories to help us learn the self-defined categories. Then, we split the self-defined categories into two groups according to their relationship to the exited categories: 
\begin{itemize}
	\item One source category, a category that is very similar to one existed category. For example, my veggie burger can be very similar to general burger and less similar to other categories.
	\item Multi-source category, a category that doesn't have an explicit similarity to one category but share some properties with several existed categories (see figure \ref{fig:intro:multi} for instance).
\end{itemize} 
Due to different properties of these two groups, we should design different strategies to adapt them.

\begin{figure}
	\centering
	\includegraphics[scale=.6]{introduction/fig/multiple.jpg}
	\caption{Multi-source category case: an okapi can be roughly described as the combination of a body of a horse, legs of the zebra and a head of giraffe.}\label{fig:intro:multi}
\end{figure}

\subsection{Assumption of the source knowledge}
In transfer learning, the source knowledge can be presented in two different ways \cite{pan2010survey}: 
\begin{itemize}
	\item Instance transfer learning. Even though the source data can not be re-used directly, certain part of the data can be used incorporating with a few labeled target data to train the model.
	\item Parameter transfer learning. Instead of utilizing the source data, parameter transfer learning approaches re-use the parameters of the model trained from source data (called the source model). 
\end{itemize}

In this thesis, we try to explore a method to learn self-defined categories via the parameter transfer learning approach and try to utilizing the source data in a hard way, by assuming that the source data is access prohibited and we can only access the trained model from the source data.
This assumption is made based on the following two facts: (1) In some situations, we may not be able to access the source data and only the model trained from the source data is available. There are many credential datasets. Therefore, it is not always possible to fully access the source data. (2) The source data could be very large and it could be tedious to determine which part of the data could benefit the transfer learning. On the other hand, the trained model from the source data can be as informative as the source data itself. For example, the information extracted from the support vectors (SVs) of a SVM model trained from a dataset could be as much as the information of the whole dataset. Some results from recent work also show that it is possible to obtain a good model by even just utilize the source models and learn new categories from a few examples \cite{fei2006one} \cite{tommasi2010safety} \cite{tommasi2014learning}.

In this section, we discuss the scenario of the learning self-defined category problem. We conclude that learning self-defined categories can be achieved by transfer learning. We also make an assumption that we can only access the model trained from the source data and the source data itself is prohibited. \hl{In the next section, we discuss the challenges of our problem.}

%In this thesis, we propose a novel transfer learning method, called \textbf{SMITLe} (\textbf{S}afety \textbf{M}ulticlass \textbf{I}ncremental \textbf{T}ransfer \textbf{Le}arning) that is able to learn new categories from a few examples.
\section{Main Contribution}
In this section, we discuss several major challenges that we may encounter when solving the learning self-defined category problem.
To solve the learning self-defined category problem, there are the following major problems:
\begin{enumerate}
	\item What's the classifier and parameter we should use to train the model that can transfer the knowledge from the source data effectively?
	\item How are the knowledges transferred from the source task to target one and control the amount of the knowledge that should be transferred? 
	\item How to guarantee the result of transfer model? 
\end{enumerate}

The first challenge is to determine the classifier we use for the source data as well as the target data. There are two criterion for us to choose the classifier: (1) This classifier should be popular and has been used in many tasks. We would expect our method to be general enough and can be applied to many different scenarios. (2) Because we can only access the trained source model, this source model should be able to restore as much information as its training data to make sure that there is enough knowledge to be leveraged for the transferred model.  
%Therefore, in this thesis we choose \textbf{S}upport \textbf{V}ector \textbf{M}achines (SVM) as the classifier to train the source model as well as the target one. \cite{cristianini2000introduction}. SVM has been widely used in various of machine learning and image recognition tasks due to its performance and generalization ability. Another advantage of using SVM as the classifier is that a SVM model can provide as much information as its train data \hl{(we will discuss it later!!!!)}.
Then we have to choose the parameters of the model to be used for transfer learning. As we discussed above, the parameter should be informative enough to represent the source data. 
%In this thesis, we choose the hyperplane parameter $W$ of the SVM model as the transfer parameter.

The second challenge of out problem is to determine the way of the knowledge transferred from source model to target one. To learn a new category by transfer learning, according to our human experience, we would firstly select several known categories that are similar to the new one and then describe the new category as a combination of these known ones (see figure \ref{fig:intro:multi}). It is worthy to note that despite of the knowledge from the source models, we will also extract some new knowledge from the examples of the new category. This indicates that how to combine those knowledge from learned categories (e.g. the parameters of the source models) and the knowledge from the new category is the key for our task. 

The last challenge is to guarantee the performance of our transfer method. A baseline criterion is that after leveraging the knowledge from the source model, the performance of the transfer model to distinguish the self-defined categories will be improved or at least won't get worse when there is little useful knowledge in the source model. This is an important criterion for transfer learning. We should always expect to leverage knowledge of the source model to help training the target model. However, inappropriately leveraging the knowledge of the source models would decrease the performance of the target model. For example, we could never expect to obtain a good model to distinguish apple and other fruits by relying too much on the knowledge of distinguishing human and animal.
\section{Summary}
In this chapter, we briefly introduced the problems that are solved in this thesis. We first demonstrate the procedure for image recognition and introduce some previous work for visual transfer learning. Then we pointed out the limitations of the previous work and proposed our methods under two transfer scenarios.