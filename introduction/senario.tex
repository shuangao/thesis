As we mentioned before, research of visual transfer learning focuses on designing classifiers that can leverage the source knowledge effectively. In this section, we briefly review methods for the visual transfer learning and show the limitation of the previous work. 
\subsection{Intuition for Visual Transfer Learning}
The intuition of visual transfer learning comes from our human recognition mechanism. For our human, all the information acquired is stored in our memory. These information are organized according to the properties. When we see a new concept, we don't treat it isolated, but connect it to certain previous knowledge we stored in our memory. By comparing a new concept with the organized information in our memory, we can capture the property of a new concept effectively. When referring to visual tasks, several examples can be given to show this cognitive ability. For instance, when we describe the animal "zebra", we would probably say that: zebra is a horse with while and black striped coats. People who never see a zebra could instantly have a rough idea of a zebra. 

\begin{figure}
	\centering
	\includegraphics[scale=.6]{introduction/fig/multiple.jpg}
	\caption{An intuitive description for human to learn new concept: an okapi can be roughly described as the combination of a body of a horse, legs of the zebra and a head of giraffe.}\label{fig:intro:multi}
\end{figure}

This means to learn a concept effectively, we should be able to make use of the gained knowledge instead of learn it from scratch. This process is commonly refer to as transfer learning. Traditional machine learning methods work under the common assumption: training data and testing data are drawn from the same feature space and same distribution. When considering adding classifying the data from the new categories, the distribution of the data changes and a new model should be rebuilt from newly collected data. Those data from the original distribution is called source data, and data from the new distribution is called target data. Transfer learning is used to utilize the source knowledge from the source data to help training the new model to classify the target data. 

\subsection{Approaches for Visual Transfer Learning}
Successfully leveraging the source knowledge can greatly improve the performance of the target model. In general, the more related the source and target domain are, the more useful the source knowledge is and the more benefit the target model can get. Leveraging unrelated knowledge cannot help to improve the performance of the target model or even hurt the it. Therefore, the key issue for visual transfer learning is to identify the relatedness of the source and target domain. The major approaches for Visual Transfer Learning consists of two main direction: Distribution Similarity Measurement and Instance Reuse.

\begin{itemize}
	\item \textbf{Distribution Similarity Measurement}. A straight-forward approach to identify the relatedness of the source and target domain is to measure their similarity directly. Measuring the data discrepancy such as \textbf{Maximum Mean Discrepancy} (MMD) \cite{duan2009domain}, has been a popular way to identify the source and target domain. MMD reflects the distance of two data distributions in the Reproducing Kernel Hilbert Space (RKHS) \cite{aronszajn1950theory}.
	\item \textbf{Instance Reuse} \cite{lim2012transfer}. Another simple way to use the source knowledge is to ``borrow" some of the data from the source domain and use it to build the target model together with the target data. This approach can directly increase the size of the data in the target domain and effectively improve the performance of the target model.
	\textbf{Feature Transformation} \cite{duan2012learning} can overcome the data distribution mismatch in different domains and project the data into the same augmented space and thus can increase the training data for the target task as well (see Figure \ref{fig:intro:trans}).
\end{itemize}

\begin{figure}
	\centering
	\includegraphics[scale=.3]{introduction/fig/transformation.png}
	\caption{Feature transformation. Transform the data in different domains into a augmented feature space.}\label{fig:intro:trans}
\end{figure}

\subsection{Limitation of Previous Methods}
From the review of the approaches for Visual Transfer Learning we can see that
most previous methods require to access to the source data to obtain the source knowledge for the knowledge transfer process. However, in many practical problems, these previous approaches may not be as convenient as we thought due to the following reasons:

\begin{itemize}
	\item \textbf{Data accessibility}. The source data may not be able to access for some tasks. For example, clinical database is not allowed to access for general publics due to the privacy. Disclosure obligations and will to share the databases are also two important reasons that make the source data inaccessible.
	\item \textbf{Size of the source data}. Besides data accessibility, many previous methods \cite{daume2009frustratingly}\cite{duan2012learning} require to access to each of the individual source instance to obtain the source knowledge which is ineffective for many large source domain. For example, it is almost impossible to measure the MMD for some large source domain which contains hundreds of thousands instances.
\end{itemize}

These previous methods are greatly limited to leverage the knowledge from certain areas and tasks. Even though we are not able to access to the source data, we can still obtain the source knowledge by obtaining the source model. Using the source model instead of the source data can successfully avoid the two issues discussed above. Source model can contain as much knowledge as the source data while without containing any information regarding to the information of the individual instance. Therefore, the owner of the source data don't have to worry about the leak of the privacy. For those large source dataset such as ILSVRC containing millions of images, a trained source model is normally a few megabytes and public available for general publics. Therefore, leveraging the source knowledge from source model instead of the source data itself is more practical for real visual transfer learning applications.
%%%%%%%%%%%%%%%%%%%%%%%%%%%%%%%%%%%%%%%%%%%%%%%%%%%%%%%%%%%%%%%%%%%%%%%%%%%%%%%%%
