In this section, we discuss several major challenges that we may encounter when solving the learning self-defined category problem.
To solve the learning self-defined category problem, there are the following major problems:
\begin{enumerate}
	\item What's the classifier and parameter we should use to train the model that can transfer the knowledge from the source data effectively?
	\item How are the knowledges transferred from the source task to target one and control the amount of the knowledge that should be transferred? 
	\item How to guarantee the result of transfer model? 
\end{enumerate}

The first challenge is to determine the classifier we use for the source data as well as the target data. There are two criterion for us to choose the classifier: (1) This classifier should be popular and has been used in many tasks. We would expect our method to be general enough and can be applied to many different scenarios. (2) Because we can only access the trained source model, this source model should be able to restore as much information as its training data to make sure that there is enough knowledge to be leveraged for the transferred model.  
%Therefore, in this thesis we choose \textbf{S}upport \textbf{V}ector \textbf{M}achines (SVM) as the classifier to train the source model as well as the target one. \cite{cristianini2000introduction}. SVM has been widely used in various of machine learning and image recognition tasks due to its performance and generalization ability. Another advantage of using SVM as the classifier is that a SVM model can provide as much information as its train data \hl{(we will discuss it later!!!!)}.
Then we have to choose the parameters of the model to be used for transfer learning. As we discussed above, the parameter should be informative enough to represent the source data. 
%In this thesis, we choose the hyperplane parameter $W$ of the SVM model as the transfer parameter.

The second challenge of out problem is to determine the way of the knowledge transferred from source model to target one. To learn a new category by transfer learning, according to our human experience, we would firstly select several known categories that are similar to the new one and then describe the new category as a combination of these known ones (see figure \ref{fig:intro:multi}). It is worthy to note that despite of the knowledge from the source models, we will also extract some new knowledge from the examples of the new category. This indicates that how to combine those knowledge from learned categories (e.g. the parameters of the source models) and the knowledge from the new category is the key for our task. 

The last challenge is to guarantee the performance of our transfer method. A baseline criterion is that after leveraging the knowledge from the source model, the performance of the transfer model to distinguish the self-defined categories will be improved or at least won't get worse when there is little useful knowledge in the source model. This is an important criterion for transfer learning. We should always expect to leverage knowledge of the source model to help training the target model. However, inappropriately leveraging the knowledge of the source models would decrease the performance of the target model. For example, we could never expect to obtain a good model to distinguish apple and other fruits by relying too much on the knowledge of distinguishing human and animal.