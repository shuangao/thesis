In last section, we introduced the multi-class LOO-CV loss. We can see that it is a function of the transfer parameter. Therefore, in this section, we propose a novel objective function based on the multi-class LOO-CV loss. We use sub-gradient descent to optimize this objective function to get the optimal solution. 

As we discussed in the last section, the optimal transfer parameter is the one that minimize the LOO-CV error. We can see that the multi-class LOO-CV loss is an extension of binary hinge loss and therefore, it is convex. We propose the following objective function based on the multi-class LOO-CV loss:

\begin{equation}\label{eq:single:opti}
\begin{aligned}
& \textbf{min}
& & L(\gamma)=\frac{{{\lambda }}}{2}\sum\limits_{n = 1}^N {{{\left\| {{\gamma _n}} \right\|}^2}}  + \sum\limits_{i = 1}^l {{\epsilon_{i} }}   \\
& \textbf{s.t.}
& & 1 - {\varepsilon _{n{y_i}}} + {\hat y_{in}}\left( {\gamma  } \right) - {\hat y_{i{y_i}}}\left( {\gamma } \right) \le {{\epsilon_{i} }};\\
& & &\lambda \ge 0
\end{aligned}
\end{equation}
Besides the multi-class LOO-CV loss, we add an extra L2 regularization term in objective function \eqref{eq:single:opti}. Compared to other methods such as \cite{tommasi2010safety} and \cite{kuzborskij2013n}, which optimize the multi-class loss directly with certain constraint for the transfer parameters, the L2 regularization term has the following advantages:
\begin{itemize}
	\item The regularization term can limit the value of the transfer parameter. Our objective function try to make a balance between the knowledge transferred from the source and the LOO-CV error made on the target data. For example, if increase the amount of the source knowledge can improve the performance of our model, we can expect a decrease on the LOO-CV loss. However, because we the LOO-CV error is estimated from a small dataset, it is more likely to prune to overfitting. Overfitting on a small dataset would lead to poor generalization ability. As a result, it would lead to decreased performance. In \cite{kuzborskij2013stability}, it shows that the upper bound of the generalization error of a transfer model can be written as:
	\begin{equation}
		Err \leq Err_{loo}+\mathcal{O}(||\gamma||^2)
	\end{equation}
	Therefore, limit the value of the transfer parameter can improve the generalization ability of our transfer model.
	\item Because of the L2 regularization term, our objective function is a strongly convex function. Due to its strong convexity, we can use sub-gradient descent \cite{boyd2004convex} to search for the optimal transfer parameter. We show that we can find the $\epsilon$-accurate solution by using only $\mathcal{O}\left(\frac{\ln \epsilon}{\epsilon}\right)$ iterations. In each iteration, we only have to solve 
\end{itemize}
\begin{theorem}[Converge Property]\label{th:single:converg}
	Let $L(\gamma)$ be a $\lambda$-strongly convex function in \eqref{eq:single:opti} and $\gamma^*$ be its optimal solution. Let $\gamma_1,...,\gamma_{T+1}$ be a sequence such that $\gamma_1 \in B$ and for $t>1$, we have $\gamma_{t+1} = \gamma_t - \eta_t \Delta_t$ , where $\Delta_t$ is the sub-gradient of $L(\gamma_t)$ and $\eta_t = 1/(\lambda t)$. Assume we have $||\Delta_t|| \leq G$ for all $t$. Then we have:
	
	\begin{equation}
	L(\gamma_{T+1}) \leq L(\gamma^*)+\frac{G^2(1+\ln (T))}{2\lambda T}
	\end{equation}
\end{theorem}
We show the detail proof in Appendix \ref{app:converg}.

By adding a dual set of variables, one for each constraint, we get the Lagrangian of the optimization problem \eqref{eq:single:opti}:
\begin{equation}\label{eq:single:dual}
\begin{aligned}
\textbf{max}\qquad {}& L\left( {\gamma ,\epsilon ,\eta } \right) =\frac{{{\lambda }}}{2}\sum\limits_{n = 1}^N {{{\left\| {{\gamma _n}} \right\|}^2}}  + \sum\limits_{i = 1}^l {{\epsilon _i}} \\
&+ \sum\limits_{i,n} {{\eta _{i,n}}\left[ {1 - {\varepsilon _{n{y_i}}} + {{\hat Y}_{in}}\left( {\gamma  } \right) - {{\hat Y}_{i{y_i}}}\left( {\gamma } \right) - {\epsilon _i}} \right]}  \\
\textbf{s.t.} \qquad {} & \forall i,n \quad {} {\eta _{i,n}} \ge 0
\end{aligned}
\end{equation}
The sub-gradient of can be written as:

\begin{equation*}
{\Delta _\gamma }=\begin{cases}
\boldsymbol{0}&{y_i}=n\\
\\
\left[ {0,..,\frac{{\alpha ''}_{in}}{\mathcal{S} _{ii}^{ - 1}},.., - \frac{{\alpha ''}_{i{y_i}}}{\mathcal{S} _{ii}^{ - 1}},..,0} \right]&{y_i},n = 1,...N\\
\end{cases}
\end{equation*}
We call this algorithm Safety MultI-class  Transfer Learning (SMITLe) and show the algorithm details in Alg. \ref{alg:1}. 
\begin{algorithm}\label{alg:1}
       \caption{SMITLe}\label{alg:1}
        \begin{algorithmic}[1]
            \REQUIRE $\lambda, \mathcal{S},\alpha',\alpha'',T$,
            \ENSURE $\gamma=\left\{\gamma^1,...,\gamma^n\right\}$
            \STATE $\gamma^0 \leftarrow 1$
            \FOR {$t=1$ to $T$}
                \STATE $\hat Y \leftarrow Y - {\left( {\mathcal{S} \circ I} \right)^{ - 1}}\left( \alpha' - \alpha''D(\gamma) \right)$
                \STATE ${\Delta _\gamma }=0, {\Delta _\beta }=0$
                \FOR {$i=1$ to $l$}
                	\STATE ${\Delta _\gamma }\leftarrow {\Delta _\gamma }+\lambda_1\gamma$ 
                	\FOR {$r=1$ to $N+1$}
	                    \STATE $l_{ir} = 1 - {\varepsilon _{{y_i}r}} + {\hat Y_{ir}} - {\hat Y_{i{y_i}}}$
	                    \IF{$l_{ir}>0$}
	                            \STATE $\Delta _\gamma^{{y_i}} \leftarrow \Delta _\gamma^{{y_i}} - \frac{{{\alpha''_{i{y_i}}}}}{{{\mathcal{S}^{-1}_{ii}}}}$%
	                    \ENDIF
	                 \ENDFOR %class ends   
                \ENDFOR %examples ends
                \STATE $\gamma^t  \leftarrow \gamma^{(t-1)}  - \frac{{{\Delta _\gamma }}}{{\lambda\times {t} }}$
             \ENDFOR %iteration ends
        \end{algorithmic}
\end{algorithm}

In summary, in this section, we proposed a novel objective function based on LOO-CV error. By adding the L2 regularization terms we show that we can use sub-gradient descent to find the $\epsilon$-accurate solution by using only $\mathcal{O}\left(\frac{\ln \epsilon}{\epsilon}\right)$ iterations. In the next section, we show some empirical results on the benchmark dataset.
